\documentclass[12pt, oneside]{article}

\usepackage{amssymb,amsmath,pifont,amsfonts,comment,enumerate,enumitem}
\usepackage{currfile,xstring,hyperref,tabularx,graphicx,wasysym}
\usepackage[labelformat=empty]{caption}
\usepackage[dvipsnames,table]{xcolor}
\usepackage{multicol,multirow,array,listings,tabularx,lastpage,textcomp,booktabs}

% NOTE(joe): This environment is credit @pnpo (https://tex.stackexchange.com/a/218450)
\lstnewenvironment{algorithm}[1][] %defines the algorithm listing environment
{   
    \lstset{ %this is the stype
        mathescape=true,
        frame=tB,
        numbers=left, 
        numberstyle=\tiny,
        basicstyle=\rmfamily\scriptsize, 
        keywordstyle=\color{black}\bfseries,
        keywords={,procedure, div, for, to, input, output, return, datatype, function, in, if, else, foreach, while, begin, end, }
        numbers=left,
        xleftmargin=.04\textwidth,
        #1
    }
}
{}
\lstnewenvironment{java}[1][]
{   
    \lstset{
        language=java,
        mathescape=true,
        frame=tB,
        numbers=left, 
        numberstyle=\tiny,
        basicstyle=\ttfamily\scriptsize, 
        keywordstyle=\color{black}\bfseries,
        keywords={, int, double, for, return, if, else, while, }
        numbers=left,
        xleftmargin=.04\textwidth,
        #1
    }
}
{}

\newcommand\abs[1]{\lvert~#1~\rvert}
\newcommand{\st}{\mid}

\newcommand{\A}[0]{\texttt{A}}
\newcommand{\C}[0]{\texttt{C}}
\newcommand{\G}[0]{\texttt{G}}
\newcommand{\U}[0]{\texttt{U}}

\newcommand{\cmark}{\ding{51}}
\newcommand{\xmark}{\ding{55}}


\usepackage{enumitem}

\begin{document}
\begin{flushright}
\StrBefore{\currfilename}{.}
\end{flushright}

\section*{This week's highlights}
\begin{itemize}
\item Distinguish between and use as appropriate each of structural induction, mathematical induction, and strong induction
\item Evaluate which proof technique(s) is appropriate for a given proposition
\item Trace and/or construct a proof by contradiction
\end{itemize}

\section*{Lecture videos}
Week 7 Day 1
\href{https://youtube.com/playlist?list=PLML4QilACLk5c1QSKhlVQktPUMhXeXPUo}{YouTube playlist}

Week 7 Day 2
\href{https://youtube.com/playlist?list=PLML4QilACLk62CVQOOJAjgI03qis1LasU}{YouTube playlist}


\newpage
\section*{Wednesday February 17}

For which nonnegative integers  $n$ can we make change for $n$ with coins of 
value $5$ cents and $3$ cents?


Restating: We can make change for \underline{\phantom{$3$~~,~~ $5$~~,~~ $6$}}, we
cannot make change for \underline{\phantom{$1$~~,~~ $2$~~,~~ $4$~~,~~ $7$}}, and 
\[
\underline{\phantom{\forall n  \in  \mathbb{Z}^{\geq 8}  \exists x \in \mathbb{N}  \exists y \in \mathbb{N}  (  5x+3y =  n)\qquad \qquad}} \star
\]


\vfill

\fbox{\parbox{\textwidth}{%

{\bf New! Proof by Strong Induction} (Rosen 5.2 p337)

To prove that a universal quantification over the set of all integers greater than or equal to some  base integer $b$ holds,  pick a  fixed nonnegative integer  $j$ and then: \hfill 

\begin{tabularx}{\textwidth}{lX}
    Basis Step: & Show the statement holds for $b$, $b+1$, \ldots, $b+j$. \\
    Recursive Step: & Consider an arbitrary integer $n$ greater than or  equal to  $b+j$, assume
    (as the {\bf strong  induction hypothesis})  that the property holds  for {\bf each of} $b$, $b+1$, \ldots, $n$, 	
    and use  this and
    other facts to  prove that  the property holds for $n+1$.
\end{tabularx}
}} 

\vfill

\begin{center}
\begin{tabular}{cll}
$\mathbb{N}$  &  The set of  natural numbers & $\{ 0, 1, 2, 3, \ldots \}$ \\
%$\mathbb{Z}$  &  The set of  integers & $\{ \ldots -3, -2, -1, 0, 1, 2, 3, \ldots \}$ \\
%$\mathbb{Z}^+$ & The set of positive integers  & $\{ 1, 2, 3,  \ldots  \}$ \\
$\mathbb{Z}^{\geq b}$ & The set of integers greater than  or equal  a  basis element $b$ & $\{ b, b+1, b+2, b+3,  \ldots  \}$ \\
\end{tabular}
\end{center}

\vfill
\newpage

\setlength{\columnseprule}{0.4pt}
\begin{multicols}{2}
{\bf  Proof of $\star$ by mathematical induction} ($b=8$)

{\bf Basis step}:  WTS property is true about  $8$
\vspace{50pt}

{\bf Recursive step}: Consider an  arbitrary  $n \geq 8$.
Assume (as the  IH) that  there are nonnegative integers
$x, y$ such that $n =  5x +  3y$.  WTS
that there are nonnegative integers $x', y'$ such
that  $n+1 = 5x' +  3y'$.  We consider two cases, 
depending on  whether  any  $5$ cent coins
are used for $n$.

\vspace{1in}

{\it Case 1}:  Assume $x \geq  1$.
Define $x' = x-1$ and $y'=y+2$ (both in  $\mathbb{N}$ by case assumption).

\vspace{-15pt}

Calculating:
\begin{align*}
5x' +  3y' &\overset{\text{by def}}{=}  5(x-1) +  3(y+2)  = 5x -  5 +3y +   6  \\
&\overset{\text{rearranging}} = (5x+3y) -5  + 6\\
& \overset{\text{IH}}{=} n-5+6 =  n+1
\end{align*}

\vspace{1in}

{\it  Case 2}: Assume $x = 0$.  Therefore  $n  = 3y$,  so 
since  $n \geq 8$, $y \geq 3$. Define $x' = 2$ and $y'=y-3$
(both in $\mathbb{N}$ by case assumption).
Calculating:
\begin{align*}
5x' +  3y' &\overset{\text{by def}}{=}  5(2) +  3(y-3)  = 10  +3y -9  \\
&\overset{\text{rearranging}} = 3y +10 -9 \\
&\overset{\text{IH and case}}{=} n+10-9 =  n+1
\end{align*}

\vspace{1in}

\columnbreak


{\bf Proof of $\star$ by strong induction} ($b=8$ and $j=2$)

{\bf Basis step}:  WTS property is true about  $8, 9, 10$
\vspace{50pt}

{\bf Recursive step}: Consider an  arbitrary  $n \geq 10$.
Assume (as the  IH) that the property is true about  each of $8, 9, 10, \ldots, n$.  
WTS
that there are nonnegative integers $x', y'$ such
that  $n+1 = 5x' +  3y'$.

\vspace{200pt}
\end{multicols}


\vfill

\newpage
{\bf  Representing positive integers}


{\bf Theorem}: Every positive integer is a sum of (one or more) distinct powers of $2$. {\it  
binary expansions exist!}



{\bf Proof by strong induction}, with $b=1$ and $j=0$.


{\bf Basis step}:  WTS property is true about  $1$.


{\bf Recursive step}: Consider an arbitrary integer $n \geq 1$.
Assume (as the IH) that the property is true about  each of $1, \ldots, n$.  
WTS that the property is true about  $n+1$.


\vfill

{\bf Definition} (Rosen p257):  An integer $p$ greater than $1$ is called {\bf prime} if the only positive factors of 
$p$ are $1$ and $p$. A positive integer that is greater than $1$ and is not prime is called composite. 



{\bf Theorem} (Rosen p336): Every positive integer {\it greater than 1} is a product of (one or more) primes.

{\bf Proof by strong induction}, with $b=2$ and $j=0$.

{\bf Basis step}:  WTS property is true about  $2$.
\vspace{20pt}

{\bf Recursive step}: Consider an arbitrary integer $n \geq 2$.
Assume (as the IH) that the property is true about  each of $2, \ldots, n$.  
WTS that the property is true about  $n+1$.


{\bf Case 1}: 


{\bf Case 2}: 

\vfill

\newpage
\section*{Friday February 19}
\vspace{-20pt}
{\bf Definition} (Rosen p. 257):  An integer $p$ greater than $1$ is called {\bf prime} means 
the only positive factors of  $p$ are $1$ and $p$.  A formal definition of the predicate $Pr$ over the domain $\mathbb{Z}$ which evaluates to T exactly when the input is prime is: $Pr(x) = (x > 1) \land \forall a( ~ (~ a > 0 \land F(a,x) ~) \to (a = 1 \lor a = x) ~)$

\fbox{\parbox{\textwidth}{%

{\bf New! Proof by Contradiction} (Rosen 1.7  p86)

To prove that a statement $p$ is true, pick another statement $r$ and once we show
that $\neg p  \to (r \wedge  \neg r)$ then  we can conclude that  $p$ is  true.

{\it Informally} The statement we care about can't possible be false, so it must be true.
}} 



{\bf Prove} or {\bf  disprove}:  There is a least prime number.


\vfill





{\bf Prove} or {\bf  disprove}: There is a greatest integer. 

{\it Approach 1, De Morgan's and universal generalization}: 

\vfill

{\it Approach 2, proof by contradiction}: 

\vfill

\vfill

{\it Extra examples}: Prove or disprove that $\mathbb{N}$,  $\mathbb{Q}$ each have a
least and a greatest element. Prove that there is no greatest prime number.


\newpage

The {\bf set  of rational numbers}, $\mathbb{Q}$  is defined as 
\[
\left\{ \frac{p}{q} \mid p \in \mathbb{Z}  \text{ and  } q  \in \mathbb{Z} \text{ and } q \neq  0 \right\}
\text{~~~~or, equivalently,~~~~}
\{ x  \in  \mathbb{R} \mid \exists p \in \mathbb{Z}  \exists q \in \mathbb{Z}^+ ( p =  x \cdot q) \}
\]

{\it Extra practice}: Use the definition of set equality to prove that the definitions above  give the same set.

\vspace{20pt}

{\bf Goal}:  The square root of $2$ is not a rational number.  In other words: $\neg \exists x \in \mathbb{Q} ( x^2 -  2 = 0)$

{\bf Attempted proof}: The definition of the set of rational numbers is the collection of fractions $p/q$ where $p$ is an integer and $q$ is a nonzero integer. Looking for a {\bf witness} $p$ and $q$, we can write the square root of $2$ as the fraction 
$\sqrt{2 }/1$, where $1$ is a nonzero integer. Since the numerator is not in the domain, this witness is not allowed, and we have shown that the square root of $2$ is not a fraction of integers (with nonzero denominator). Thus, the square root of $2$ is not rational.


{\it The problem in the above attempted proof is that} \underline{\phantom{it only considers one candidate witness
and does not prove that no witnesses exist.}}


{\bf Proof}: 



\vfill






\vfill

{\bf Lemma 1:} For every two integers $p$  and  $q$, not both zero, $gcd\left( \frac{p}{gcd(p,q) },  \frac{q}{gcd(p,q)} \right) =  1$.


{\bf Lemma 2:} For every two integers $a$ and  $b$, not both zero, with  $gcd(a,b) = 1$, it is not the case that both $a$
is  even and $b$ is even.


{\bf Lemma 3:} For every integer  $x$, $x$ is  even if and only if $x^2$  is even.


\fbox{\parbox{\textwidth}{%

{\bf Greatest common divisor} (Rosen 4.3 p265) Let $a$ and $b$ be integers, not both zero. The largest integer $d$ such that 
$d$ is a  factor of $a$ and $d$ is a factor of  $b$ is called the greatest common divisor of $a$ and $b$ 
and is denoted by $gcd(a, b)$.
}
}




\newpage
\section*{Review quiz questions}
\begin{enumerate}
\item {\bf Wednesday} Consider the following statement: For $n > 0$, the sum of the first $n$ positive integers, also written as $\sum_{i=1}^n 
i$, is equal to $$\left(\dfrac{n \cdot (n + 1)}{2}\right)$$.

For example, when $n = 3$, the statement would mean that $1+2+3 = \left(\dfrac{3 \cdot (3 + 1)}{2}\right)$.

Consider the following (start to) a proof of this statement:

Basis Step: Choose $n = 1$ as the basis step. Applying the formula from the original statement, we find that $\dfrac{1 \cdot (1 + 1)}{2}$ is equal to $1$. Since the total sum of the first positive integer is $1$, these are equal and the basis step is complete.

Recursive Step: Consider an arbitrary $k \geq 1$.  Towards a direct proof, we assume (as the induction hypothesis) that the sum of the first $k$ positive integers is $\left(\dfrac{k \cdot (k + 1)}{2}\right)$. We want to show that the sum of the first $k + 1$ positive integers is $$\left( \dfrac{(k + 1) \cdot ((k + 1) + 1)}{2}\right)$$.

[Proof would continue here...]

\begin{enumerate}
   \item In a recursive definition of the function that gives the sum of the first $n$ positive integers, the domain is 
    \begin{enumerate}
        \item $\mathbb{N}$
        \item $\mathbb{Z}^+$
        \item $\mathbb{Z}$
    \end{enumerate}

   \item In a recursive definition of the function that gives the sum of the first $n$ positive integers, the basis step is
    \begin{enumerate}
        \item $\sum_{i=1}^1 i = 1$
        \item $\sum_{i=1}^n 1 = n$
        \item None of the above.
    \end{enumerate}

   \item In a recursive definition of the function that gives the sum of the first $n$ positive integers, the recursive step is
    \begin{enumerate}
        \item If $n$ is a positive integer, $\sum_{i=1}^n i = n$
        \item If $n$ is a positive integer, $\sum_{i=1}^n i =\left(  \sum_{i=1}^n i \right)+ 1$
        \item If $n$ is a positive integer, $\sum_{i=1}^{n+1} i =\left(  \sum_{i=1}^n i \right) + 1$
        \item If $n$ is a positive integer, $\sum_{i=1}^{n+1} i =\left(  \sum_{i=1}^n i \right) + n$
        \item If $n$ is a positive integer, $\sum_{i=1}^{n+1} i =\left(  \sum_{i=1}^n i \right)+ (n+1)$
    \end{enumerate}

    \item The proof technique used here is:
    \begin{enumerate}
        \item Structural induction
        \item Mathematical induction
        \item Strong induction
    \end{enumerate}
    
    \item Which of these is both true, and a useful next step in the proof?
    
    \begin{enumerate}
        \item By the induction hypothesis, we know that $$\left( \dfrac{(k + 1) \cdot ((k + 1) + 1)}{2}\right) = \left(\dfrac{k \cdot (k + 1)}{2}\right)$$
        \item By the induction hypothesis, we know that $$\left( \dfrac{(k + 1) \cdot ((k + 1) + 1)}{2}\right) > \left(\dfrac{k \cdot (k + 1)}{2}\right)$$
        \item By the induction hypothesis and the definition of the statement we're proving, we know that $$\left(\dfrac{k \cdot (k + 1)}{2}\right) + k$$ is the sum of the first $k + 1$ positive integers.
        \item By the induction hypothesis and the definition of the statement we're proving, we know that $$\left(\dfrac{k \cdot (k + 1)}{2}\right) + k + 1$$ is the sum of the first $k + 1$ positive integers.
        \item None of the above
    \end{enumerate}
\end{enumerate}


\item {\bf Wednesday} Recall that an integer $p$ greater than $1$ is called {\bf prime} if the only positive factors of 
$p$ are $1$ and $p$. Fill in the following blanks in the proof of the {\bf Theorem} (Rosen p336): 
Every positive integer {\it greater than 1} is a product of (one or more) primes.

{\bf Proof}: We proceed by \underline{~~~BLANK for part (a)~~~}.
\begin{itemize}
\item[] Basis step: We want to show that the property is about $2$. Since $2$ is itself prime,
it is already written as a product of (one) prime.
\item[] Recursive step: Consider an arbitrary integer $n \geq 2$.  Assume, as the induction hypothesis,
that \underline{~~~BLANK for part (c)~~~}. We want to show that $n+1$ can be written 
as a product of primes.  There are two cases to consider: $n+1$ is itself prime or it is composite.
In the first case, we assume $n+1$ is prime and then immediately it is written as a product
of (one) prime so we are done.  In the second case, we assume that $n+1$ is composite
so there are integers $x$ and $y$ where $n+1 = xy$ and each of them is between $2$ and $n$
(inclusive).  Therefore, the induction hypothesis applies to each of $x$ and $y$ so each 
of these factors of $n+1$ can be written as a product of primes.  Multiplying these products together, 
we get a product of primes that gives $n+1$, as required.  Since both cases give the necessary
conclusion, the proof by cases for the recursive step is complete.
\end{itemize}
\begin{enumerate}
\item  The proof technique used here is:
    \begin{enumerate}
        \item Structural induction
        \item Mathematical induction
        \item Strong induction
    \end{enumerate}
\item How many basis steps are used in this proof?
\newpage
\item What is the induction hypothesis?
\begin{enumerate}
\item That $n$ can be written as a product of (one or more) primes.
\item That each integer between $0$ and $n$ (inclusive) can be written as a product of (one or more) primes.
\item That each integer between $1$ and $n+1$ (inclusive) can be written as a product of (one or more) primes.
\item That each integer between $2$ and $n$ (inclusive) can be written as a product of (one or more) primes.
\item That each integer between $3$ and $n+1$ (inclusive) can be written as a product of (one or more) primes.
\end{enumerate}
\end{enumerate}



\item {\bf Friday} {\it Goals for this question: recognize that we can prove the same statement
in different ways.  Trace proofs and justify why they are valid.}


 By definition, an integer $n$ is {\bf even} means that there is an integer $a$ such that $n = 2a$; 
an integer $n$ is {\bf odd} means that there is an integer $a$ such that $n = 2a+1$.  Equivalently, 
an integer $n$ is {\bf even} means $n ~\textbf{ mod }~2 = 0$; an integer $n$ is {\bf odd} means $n ~\textbf{ mod }~2 = 1$.  Also, an integer is even if and only if it is not odd.

{\it You can refer to any of the above definitions and claims in your proofs.}

Below are two proofs of the same statement: fill in the blanks with the 
expressions below.

{\bf Claimed statement}:  \textbf{(a)}$\underline{\phantom{\hspace{1.3in}}}$
\begin{quote}

{\bf Proof 1}: Using De Morgan's law for quantifiers, 
we can rewrite this statement as a universal of the negation of the body of the statement.
Towards a proof by universal generalization, let $x$ be an arbitrary element of $\mathbb{Z}$. Then we need to show that
$$\textbf{(b)}\underline{\phantom{\hspace{1.3in}}}$$

We proceed by contradiction to show that $$(x \textrm{ is odd} \land x^2 \textrm{ is even}) \to \textbf{(c)}\underline{\phantom{\hspace{1.3in}}}$$
We assume by direct proof that $(x \textrm{ is odd} \land x^2 \textrm{ is even})$. Then, $(x^2 \textrm{ is even})$ follows directly from this assumption, so by definition 
of conjunction, we must show that $(x^2 \textrm{ is not even})$ to complete the proof.
From the assumption, we have that $(x \textrm{ is odd})$.  Applying the definition of odd, $x = 2k + 1$ for some $k \in \mathbb{Z}$. Then $x^2 = 4k^2 + 4k + 1$. We can rewrite the right hand side to $2(2k^2 + 2k) + 1$. This shows that $x^2$ is odd by the definition of odd, since choosing $j = 2k^2 + 2k$ gives us $j \in \mathbb{Z}$ with $x^2 = 2j + 1$. Since a number is either even or odd and not both, and $x^2$ is odd, then it must not be even. 
This concludes the proof, as we have assumed the negation of the original statement and deduced a contradiction
from this assumption.
\end{quote}

\newpage
\begin{quote}{\bf Proof 2}: 

    \begin{tabular}{l p{2.5in}}
    1. \begin{tabular}{l}
        \textbf{To Show} $\forall x \in \mathbb{Z} \neg (x \textrm{ is odd} \land x^2 \textrm{ is even})$\\
    \end{tabular}
    & Rewriting statement using De Morgan's law for quantifiers
 \\ 
   2. \begin{tabular}{l}
        \textbf{Choose arbitrary} $x \in\mathbb{Z}$ \\
        \textbf{To Show} \textbf{(d)}$\underline{\phantom{\hspace{1.3in}}}$\\
    \end{tabular}
    & By \textbf{(e)}$\underline{\phantom{\hspace{1.3in}}}$\\
 \\
    3. \begin{tabular}{l}
        \textbf{To Show}
         $x \textrm{ is odd} \to \neg  (x^2 \textrm{ is even})$
    \end{tabular}
    & Rewrite previous {\bf To Show} using logical equivalence
    \\
    4. \begin{tabular}{l}
        \textbf{Assume } $x \textrm{ is odd}$ \\
        \textbf{To Show } $\neg  (x^2 \textrm{ is even})$ \\
    \end{tabular}
    & By \textbf{(f)}$\underline{\phantom{\hspace{1.3in}}}$\\
    \\    
    5. \begin{tabular}{l}
        \textbf{To Show} $x^2 \textrm{ is odd}$
    \end{tabular}
    & Rewrite previous {\bf To Show} using definition of even, odd
    \\
    6. \begin{tabular}{l}
        \textbf{Use the witness} $k$, an integer,\\
        where $x = 2k+1$\\
    \end{tabular}
     & By existential definition of $x$ being odd \\
    \\
    7. \begin{tabular}{l}
        \textbf{Choose the witness} \\
        $j = 2k^2 + 2k$, an integer\\
        \textbf{To Show} $x^2 = 2j+1$
    \end{tabular}
     & Show this new {\bf To Show} is true to prove the existential definition of $x^2$ being odd\\
    \\
    8.\begin{tabular}{l}
        \textbf{To Show} $(2k+1)^2  = 2j+1$
    \end{tabular}
    & Rewrite previous {\bf To Show} using definition of $k$
    \\
    9.  \begin{tabular}{l}
        \textbf{To Show} $(2k+1)^2  = 2(2k^2 + 2k) + 1$
    \end{tabular}
    & Rewrite previous {\bf To Show} using definition of $j$
    \\
    10. \begin{tabular}{l}
        \textbf{To Show } $T$ \\
    \end{tabular}
     & By algebra: multiplying out the LHS; factoring the RHS\\
    QED & Because we got to $T$ only by rewriting \textbf{To Show} to equivalent statements, using valid proof techniques and definitions. \\
    \end{tabular}
\end{quote}


Consider the following expressions as options to fill in the two proofs above. Give your answer as one of the numbers below for each blank a-c. You may use some numbers for more than one blank, but each letter only uses one of the expressions below.

\begin{multicols}{2}
\begin{enumerate}[label=\roman*.]
    \item $\exists x \in \mathbb{Z} \, (x \textrm{ is odd} \land x^2 \textrm{ is even})$
    \item $\neg \exists x \in \mathbb{Z} \, (x \textrm{ is odd} \land x^2 \textrm{ is even})$
    \item $\exists x \in \mathbb{Z} \, (x \textrm{ is odd} \land x \textrm{ is even})$
    \item $\neg \exists x \in \mathbb{Z} \, (x \textrm{ is odd} \land x \textrm{ is even})$
    \item $\exists x \in \mathbb{Z} \, (x^2 \textrm{ is odd} \land x^2 \textrm{ is even})$
    \item $\neg \exists x \in \mathbb{Z} \, (x^2 \textrm{ is odd} \land x^2 \textrm{ is even})$
    \item $(x^2 \textrm{ is even} \land x^2 \textrm{ is not even})$
    \item $\neg (x \textrm{ is odd} \land  x^2 \textrm{ is even})$
    \item $(x \textrm{ is odd} \land  x^2 \textrm{ is even})$
    \item $(x \textrm{ is odd} \land  x \textrm{ is not odd})$
    \item $\neg (x \textrm{ is odd} \land  x \textrm{ is not odd})$
    \item $x^2 \textrm{ is even}$
    \item $x^2 \textrm{ is odd}$
    \item universal generalization
    \item proof by cases
    \item direct proof
    \item proof by contraposition
    \item proof by contradiction
\end{enumerate}
\end{multicols}

\item {\bf Friday}  {\it  Goals for this question: Reason through
multiple nested quantifiers. Fluently use the definition and properties of the set of rationals. 
}

Recall the definition of the set of rational numbers, $\mathbb{Q} = \left\{ \frac{p}{q} \mid p \in \mathbb{Z}  \text{ and  } q  \in \mathbb{Z} \text{ and } q \neq  0 \right\}$.
We define the set of {\bf irrational} numbers $\overline{\mathbb{Q}} = \mathbb{R} - \mathbb{Q}
= \{ x \in \mathbb{R} \mid x \notin \mathbb{Q} \}$.
\begin{multicols}{2}
\begin{enumerate}[label=(\roman*)]
\item $\forall x \in \mathbb{Q} ~\forall y \in \mathbb{Q}~ \exists z \in \mathbb{Q} ~( x + y = z)$
\item $\forall x \in \mathbb{Q} ~\forall y \in \mathbb{Q}~ \exists z \in \mathbb{Q} ~( x + z = y)$
\item $\forall x \in \mathbb{Q} ~\forall y \in \mathbb{Q}~ \exists z \in \mathbb{Q} ~( x \cdot y = z)$
\item $\forall x \in \mathbb{Q} ~\forall y \in \mathbb{Q} ~\exists z \in \mathbb{Q} ~( x \cdot z = y)$
\item $\forall x \in \overline{\mathbb{Q}}~ \forall y \in \overline{\mathbb{Q}}~ \exists z \in \overline{\mathbb{Q}} ~( x + y = z)$
\item $\forall x \in \overline{\mathbb{Q}}~ \forall y \in \overline{\mathbb{Q}}~ \exists z \in \overline{\mathbb{Q}}~( x + z = y)$
\item $\forall x \in \overline{\mathbb{Q}} ~\forall y \in \overline{\mathbb{Q}}~ \exists z \in \overline{\mathbb{Q}} ~( x \cdot y = z)$
\item $\forall x \in \overline{\mathbb{Q}} ~\forall y \in \overline{\mathbb{Q}}~ \exists z \in \overline{\mathbb{Q}}~( x \cdot z = y)$
\end{enumerate}
\end{multicols}

\begin{enumerate}
\item Which of the statements above (if any) could be {\bf disproved} using the counterexample 
$x = \frac{1}{2}$, $y= \frac{3}{4}$?
\item Which of the statements above (if any) could be {\bf disproved} using the counterexample 
$x = \sqrt{4}$, $y= \sqrt{3}$?
\item Which of the statements above (if any) could be {\bf disproved} using the counterexample 
$x = 0$, $y= 3$?
\item Which of the statements above (if any) could be {\bf disproved} using the counterexample 
$x = \sqrt{2}$, $y= 0$?
\item Which of the statements above (if any) could be {\bf disproved} using the counterexample 
$x = \sqrt{2}$, $y= -  \sqrt{2}$?
\end{enumerate}

{\it Hint: we proved in class that $\sqrt{2} \notin \mathbb{Q}$. You may also use the facts
that $\sqrt{3} \notin \mathbb{Q}$ and $-\sqrt{2} \notin \mathbb{Q}$.

Bonus - not to hand in: prove these facts; that is, prove that $\sqrt{3} \notin \mathbb{Q}$ and $-\sqrt{2} \notin \mathbb{Q}$. }


\end{enumerate}
\end{document}
