\documentclass[12pt, oneside]{article}

\usepackage{amssymb,amsmath,pifont,amsfonts,comment,enumerate,enumitem}
\usepackage{currfile,xstring,hyperref,tabularx,graphicx,wasysym}
\usepackage[labelformat=empty]{caption}
\usepackage[dvipsnames,table]{xcolor}
\usepackage{multicol,multirow,array,listings,tabularx,lastpage,textcomp,booktabs}

% NOTE(joe): This environment is credit @pnpo (https://tex.stackexchange.com/a/218450)
\lstnewenvironment{algorithm}[1][] %defines the algorithm listing environment
{   
    \lstset{ %this is the stype
        mathescape=true,
        frame=tB,
        numbers=left, 
        numberstyle=\tiny,
        basicstyle=\rmfamily\scriptsize, 
        keywordstyle=\color{black}\bfseries,
        keywords={,procedure, div, for, to, input, output, return, datatype, function, in, if, else, foreach, while, begin, end, }
        numbers=left,
        xleftmargin=.04\textwidth,
        #1
    }
}
{}
\lstnewenvironment{java}[1][]
{   
    \lstset{
        language=java,
        mathescape=true,
        frame=tB,
        numbers=left, 
        numberstyle=\tiny,
        basicstyle=\ttfamily\scriptsize, 
        keywordstyle=\color{black}\bfseries,
        keywords={, int, double, for, return, if, else, while, }
        numbers=left,
        xleftmargin=.04\textwidth,
        #1
    }
}
{}

\newcommand\abs[1]{\lvert~#1~\rvert}
\newcommand{\st}{\mid}

\newcommand{\A}[0]{\texttt{A}}
\newcommand{\C}[0]{\texttt{C}}
\newcommand{\G}[0]{\texttt{G}}
\newcommand{\U}[0]{\texttt{U}}

\newcommand{\cmark}{\ding{51}}
\newcommand{\xmark}{\ding{55}}


\usepackage{enumitem}

\begin{document}
% \begin{flushright}
% \StrBefore{\currfilename}{.}
% \end{flushright}

\section*{Mathematical Induction}


{\bf Invariant}: A property that is true about our algorithm no matter what. \hfill Rosen p375

{\bf Theorem}: Statement that can be shown to be true, usually an important one. \hfill Rosen p81

\begin{quote}
 Less important theorems can be called {\bf proposition}, {\bf fact}, {\bf result}.

 A less important theorem that is useful in proving a theorem is called a {\bf lemma}.
 
 A theorem that can be proved directly after another one has been proved is called a {\bf corollary}
\end{quote}







{\bf Theorem}: A robot on an infinite 2-dimensional integer grid starts at $(0,0)$ and at each step moves
to diagonally adjacent grid point. This robot can / cannot {\footnotesize({\it circle one})} reach $(1,0)$.


{\bf Definition} The set of positions the robot can visit  $P$ is defined by:
\[
\begin{array}{ll}
    \textrm{Basis Step: } & (0,0) \in P \\
    \textrm{Recursive Step: } & \textrm{If } (x,y) \in P  \textrm{, then } 
    \phantom{(x+1, y+1), (x+1, y-1), (x-1, y-1), (x-1, y+1)} \textrm{ are also in } P
\end{array}
\]

{\bf Lemma}: $\forall (x,y) \in P( (x+y \textrm{ is an even integer})~)$


Proof of theorem using lemma: To show is $(1,0) \notin P$. Rewriting the lemma to explicitly 
restrict the domain of the universal, 
we have $\forall (x,y) ~(~ (x,y) \in P  \to (x+y \textrm{ is an even integer})~)$.  Since
the universal is true, 
$ (~ (1,0) \in P \to (1+0 \textrm{ is an even integer})~)$ is a true statement.
Evaluating the conclusion of this conditional statement: 
By definition of long division, since $1 = 0 \cdot 2 + 1$ (where $0 \in \mathbb{Z}$ and 
$1 \in \mathbb{Z}$ and $0 \leq 1 < 2$ mean that $0$ is the quotient and $1$ is the remainder), $1 ~\textrm{\bf mod}~ 2 = 1$ which is not $0$ 
so the conclusion is false.  A true conditional with a false conclusion must have a false hypothesis.
Thus, $(1,0) \notin P$, QED. $\square$



Proof of lemma by structural induction:

{\bf Basis Step}


{\bf Recursive Step}.  Consider arbitrary $(x,y) \in P$.  To show is:
\[
(x+y \text{ is an even integer}) \to (\text{sum of coordinates of next position is even integer})
\]
Assume {\bf as the induction hypothesis, IH} that: 
\vfill


\newpage

\fbox{\parbox{\textwidth}{%

{\bf ``New"! Proof by Mathematical Induction} (Rosen 5.1 p329)

To prove a universal quantification over the set of  all integers greater than  or  equals some  base integer $b$:

\vspace{-10pt}

\begin{itemize}
\item[] {\bf Basis Step}:  Show the statement holds for $b$. 
\item[]  {\bf Recursive Step}:  Consider an arbitrary integer $n$ greater than or  equal to  $b$, assume
    (as the {\bf induction hypothesis})  that the property holds  for $n$, and use  this and
    other facts to  prove that  the property holds for $n+1$.
\end{itemize}

\vspace{-10pt}

}}


\vspace{-20pt}


\begin{center}
\begin{tabular}{|p{3.5in}|p{3.5in}|}
\hline
Recall that the set of linked lists of natural numbers $L$

\vspace{-10pt}

\begin{itemize}
\item[] Basis Step: $[] \in L$

\vspace{-10pt}

\item[] Recursive Step: If $l \in L$ and $n \in \mathbb{N}$ then $(n, l) \in L$
\end{itemize}

\vspace{-20pt}

&
Recall that length of a linked list of natural numbers $L$, $\textit{length}: L \to \mathbb{N}$ is defined by:

\vspace{-10pt}

\begin{itemize}
\item[] Basis step: $length([]) = 0 $

\vspace{-10pt}

\item[] Recursive step: If $l \in L$ and $n \in \mathbb{N}$ then $length((n, l)) = 1+ length(l)$
\end{itemize}

\vspace{-20pt}

\\
\hline
\end{tabular}
\end{center}

\vspace{-10pt}

Prove or disprove: $\forall n \in \mathbb{N} ~\exists l \in L ~(~length(l) = n~)$


\vfill
\vfill
\vfill

\newpage

\setlength{\columnseprule}{0.4pt}
\begin{multicols}{2}
{\bf  Proof of $\star$ by mathematical induction} ($b=8$)

{\bf Basis step}:  WTS property is true about  $8$
\vspace{50pt}

{\bf Recursive step}: Consider an  arbitrary  $n \geq 8$.
Assume (as the  IH) that  there are nonnegative integers
$x, y$ such that $n =  5x +  3y$.  WTS
that there are nonnegative integers $x', y'$ such
that  $n+1 = 5x' +  3y'$.  We consider two cases, 
depending on  whether  any  $5$ cent coins
are used for $n$.

\vspace{1in}

{\it Case 1}:  Assume $x \geq  1$.
Define $x' = x-1$ and $y'=y+2$ (both in  $\mathbb{N}$ by case assumption).

\vspace{-15pt}

Calculating:
\begin{align*}
5x' +  3y' &\overset{\text{by def}}{=}  5(x-1) +  3(y+2)  = 5x -  5 +3y +   6  \\
&\overset{\text{rearranging}} = (5x+3y) -5  + 6\\
& \overset{\text{IH}}{=} n-5+6 =  n+1
\end{align*}

\vspace{1in}

{\it  Case 2}: Assume $x = 0$.  Therefore  $n  = 3y$,  so 
since  $n \geq 8$, $y \geq 3$. Define $x' = 2$ and $y'=y-3$
(both in $\mathbb{N}$ by case assumption).
Calculating:
\begin{align*}
5x' +  3y' &\overset{\text{by def}}{=}  5(2) +  3(y-3)  = 10  +3y -9  \\
&\overset{\text{rearranging}} = 3y +10 -9 \\
&\overset{\text{IH and case}}{=} n+10-9 =  n+1
\end{align*}

\vspace{1in}

\columnbreak


{\bf Proof of $\star$ by strong induction} ($b=8$ and $j=2$)

{\bf Basis step}:  WTS property is true about  $8, 9, 10$
\vspace{50pt}

{\bf Recursive step}: Consider an  arbitrary  $n \geq 10$.
Assume (as the  IH) that the property is true about  each of $8, 9, 10, \ldots, n$.  
WTS
that there are nonnegative integers $x', y'$ such
that  $n+1 = 5x' +  3y'$.

\vspace{200pt}
\end{multicols}


\vfill


\end{document}
