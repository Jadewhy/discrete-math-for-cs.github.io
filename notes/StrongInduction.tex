\documentclass[12pt, oneside]{article}

\usepackage{amssymb,amsmath,pifont,amsfonts,comment,enumerate,enumitem}
\usepackage{currfile,xstring,hyperref,tabularx,graphicx,wasysym}
\usepackage[labelformat=empty]{caption}
\usepackage[dvipsnames,table]{xcolor}
\usepackage{multicol,multirow,array,listings,tabularx,lastpage,textcomp,booktabs}

% NOTE(joe): This environment is credit @pnpo (https://tex.stackexchange.com/a/218450)
\lstnewenvironment{algorithm}[1][] %defines the algorithm listing environment
{   
    \lstset{ %this is the stype
        mathescape=true,
        frame=tB,
        numbers=left, 
        numberstyle=\tiny,
        basicstyle=\rmfamily\scriptsize, 
        keywordstyle=\color{black}\bfseries,
        keywords={,procedure, div, for, to, input, output, return, datatype, function, in, if, else, foreach, while, begin, end, }
        numbers=left,
        xleftmargin=.04\textwidth,
        #1
    }
}
{}
\lstnewenvironment{java}[1][]
{   
    \lstset{
        language=java,
        mathescape=true,
        frame=tB,
        numbers=left, 
        numberstyle=\tiny,
        basicstyle=\ttfamily\scriptsize, 
        keywordstyle=\color{black}\bfseries,
        keywords={, int, double, for, return, if, else, while, }
        numbers=left,
        xleftmargin=.04\textwidth,
        #1
    }
}
{}

\newcommand\abs[1]{\lvert~#1~\rvert}
\newcommand{\st}{\mid}

\newcommand{\A}[0]{\texttt{A}}
\newcommand{\C}[0]{\texttt{C}}
\newcommand{\G}[0]{\texttt{G}}
\newcommand{\U}[0]{\texttt{U}}

\newcommand{\cmark}{\ding{51}}
\newcommand{\xmark}{\ding{55}}


\usepackage{enumitem}

\begin{document}
% \begin{flushright}
% \StrBefore{\currfilename}{.}
% \end{flushright}

% \section*{Wednesday February 17}

\section*{Strong Induction}

For which nonnegative integers  $n$ can we make change for $n$ with coins of 
value $5$ cents and $3$ cents?


Restating: We can make change for \underline{\phantom{$3$~~,~~ $5$~~,~~ $6$}}, we
cannot make change for \underline{\phantom{$1$~~,~~ $2$~~,~~ $4$~~,~~ $7$}}, and 
\[
\underline{\phantom{\forall n  \in  \mathbb{Z}^{\geq 8}  \exists x \in \mathbb{N}  \exists y \in \mathbb{N}  (  5x+3y =  n)\qquad \qquad}} \star
\]


\vfill

\fbox{\parbox{\textwidth}{%

{\bf New! Proof by Strong Induction} (Rosen 5.2 p337)

To prove that a universal quantification over the set of all integers greater than or equal to some  base integer $b$ holds,  pick a  fixed nonnegative integer  $j$ and then: \hfill 

\begin{tabularx}{\textwidth}{lX}
    Basis Step: & Show the statement holds for $b$, $b+1$, \ldots, $b+j$. \\
    Recursive Step: & Consider an arbitrary integer $n$ greater than or  equal to  $b+j$, assume
    (as the {\bf strong  induction hypothesis})  that the property holds  for {\bf each of} $b$, $b+1$, \ldots, $n$, 	
    and use  this and
    other facts to  prove that  the property holds for $n+1$.
\end{tabularx}
}} 

\vfill

\begin{center}
\begin{tabular}{cll}
$\mathbb{N}$  &  The set of  natural numbers & $\{ 0, 1, 2, 3, \ldots \}$ \\
%$\mathbb{Z}$  &  The set of  integers & $\{ \ldots -3, -2, -1, 0, 1, 2, 3, \ldots \}$ \\
%$\mathbb{Z}^+$ & The set of positive integers  & $\{ 1, 2, 3,  \ldots  \}$ \\
$\mathbb{Z}^{\geq b}$ & The set of integers greater than  or equal  a  basis element $b$ & $\{ b, b+1, b+2, b+3,  \ldots  \}$ \\
\end{tabular}
\end{center}

\vfill
\newpage

\setlength{\columnseprule}{0.4pt}
\begin{multicols}{2}
{\bf  Proof of $\star$ by mathematical induction} ($b=8$)

{\bf Basis step}:  WTS property is true about  $8$
\vspace{50pt}

{\bf Recursive step}: Consider an  arbitrary  $n \geq 8$.
Assume (as the  IH) that  there are nonnegative integers
$x, y$ such that $n =  5x +  3y$.  WTS
that there are nonnegative integers $x', y'$ such
that  $n+1 = 5x' +  3y'$.  We consider two cases, 
depending on  whether  any  $5$ cent coins
are used for $n$.

\vspace{1in}

{\it Case 1}:  Assume $x \geq  1$.
Define $x' = x-1$ and $y'=y+2$ (both in  $\mathbb{N}$ by case assumption).

\vspace{-15pt}

Calculating:
\begin{align*}
5x' +  3y' &\overset{\text{by def}}{=}  5(x-1) +  3(y+2)  = 5x -  5 +3y +   6  \\
&\overset{\text{rearranging}} = (5x+3y) -5  + 6\\
& \overset{\text{IH}}{=} n-5+6 =  n+1
\end{align*}

\vspace{1in}

{\it  Case 2}: Assume $x = 0$.  Therefore  $n  = 3y$,  so 
since  $n \geq 8$, $y \geq 3$. Define $x' = 2$ and $y'=y-3$
(both in $\mathbb{N}$ by case assumption).
Calculating:
\begin{align*}
5x' +  3y' &\overset{\text{by def}}{=}  5(2) +  3(y-3)  = 10  +3y -9  \\
&\overset{\text{rearranging}} = 3y +10 -9 \\
&\overset{\text{IH and case}}{=} n+10-9 =  n+1
\end{align*}

\vspace{1in}

\columnbreak


{\bf Proof of $\star$ by strong induction} ($b=8$ and $j=2$)

{\bf Basis step}:  WTS property is true about  $8, 9, 10$
\vspace{50pt}

{\bf Recursive step}: Consider an  arbitrary  $n \geq 10$.
Assume (as the  IH) that the property is true about  each of $8, 9, 10, \ldots, n$.  
WTS
that there are nonnegative integers $x', y'$ such
that  $n+1 = 5x' +  3y'$.

\vspace{200pt}
\end{multicols}


\vfill

\newpage
{\bf  Representing positive integers}


{\bf Theorem}: Every positive integer is a sum of (one or more) distinct powers of $2$. {\it  
binary expansions exist!}



{\bf Proof by strong induction}, with $b=1$ and $j=0$.


{\bf Basis step}:  WTS property is true about  $1$.


{\bf Recursive step}: Consider an arbitrary integer $n \geq 1$.
Assume (as the IH) that the property is true about  each of $1, \ldots, n$.  
WTS that the property is true about  $n+1$.


\vfill

{\bf Definition} (Rosen p257):  An integer $p$ greater than $1$ is called {\bf prime} if the only positive factors of 
$p$ are $1$ and $p$. A positive integer that is greater than $1$ and is not prime is called composite. 



{\bf Theorem} (Rosen p336): Every positive integer {\it greater than 1} is a product of (one or more) primes.

{\bf Proof by strong induction}, with $b=2$ and $j=0$.

{\bf Basis step}:  WTS property is true about  $2$.
\vspace{20pt}

{\bf Recursive step}: Consider an arbitrary integer $n \geq 2$.
Assume (as the IH) that the property is true about  each of $2, \ldots, n$.  
WTS that the property is true about  $n+1$.


{\bf Case 1}: 


{\bf Case 2}: 

\vfill

\end{document}
