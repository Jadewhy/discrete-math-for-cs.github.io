\documentclass[12pt, oneside]{article}

\usepackage{amssymb,amsmath,pifont,amsfonts,comment,enumerate,enumitem}
\usepackage{currfile,xstring,hyperref,tabularx,graphicx,wasysym}
\usepackage[labelformat=empty]{caption}
\usepackage[dvipsnames,table]{xcolor}
\usepackage{multicol,multirow,array,listings,tabularx,lastpage,textcomp,booktabs}

% NOTE(joe): This environment is credit @pnpo (https://tex.stackexchange.com/a/218450)
\lstnewenvironment{algorithm}[1][] %defines the algorithm listing environment
{   
    \lstset{ %this is the stype
        mathescape=true,
        frame=tB,
        numbers=left, 
        numberstyle=\tiny,
        basicstyle=\rmfamily\scriptsize, 
        keywordstyle=\color{black}\bfseries,
        keywords={,procedure, div, for, to, input, output, return, datatype, function, in, if, else, foreach, while, begin, end, }
        numbers=left,
        xleftmargin=.04\textwidth,
        #1
    }
}
{}
\lstnewenvironment{java}[1][]
{   
    \lstset{
        language=java,
        mathescape=true,
        frame=tB,
        numbers=left, 
        numberstyle=\tiny,
        basicstyle=\ttfamily\scriptsize, 
        keywordstyle=\color{black}\bfseries,
        keywords={, int, double, for, return, if, else, while, }
        numbers=left,
        xleftmargin=.04\textwidth,
        #1
    }
}
{}

\newcommand\abs[1]{\lvert~#1~\rvert}
\newcommand{\st}{\mid}

\newcommand{\A}[0]{\texttt{A}}
\newcommand{\C}[0]{\texttt{C}}
\newcommand{\G}[0]{\texttt{G}}
\newcommand{\U}[0]{\texttt{U}}

\newcommand{\cmark}{\ding{51}}
\newcommand{\xmark}{\ding{55}}


\usepackage{enumitem}

\begin{document}
% \begin{flushright}
% \StrBefore{\currfilename}{.}
% \end{flushright}


\section*{Structural Induction}
{\bf Definition} The set of RNA strands $S$ is defined (recursively) by:
\[
\begin{array}{ll}
\textrm{Basis Step: } & \A \in S, \C \in S, \U \in S, \G \in S \\
\textrm{Recursive Step: } & \textrm{If } s \in S\textrm{ and }b \in B \textrm{, then }sb \in S
\end{array}
\]
where $sb$ is string concatenation.

The function \textit{rnalen} that computes the length of RNA strands in $S$ is defined recursively by
$rnalen: S  \to \mathbb{Z}^+$

\vspace{-20pt}

\begin{quote}
Basis step: If $b \in B$ then $rnalen(b)  = 1$

Recursive step: If $s \in S$ and $b \in B$, then $rnalen(sb)  = 1 + rnalen(s)$
\end{quote}

\vspace{-20pt}

The function \textit{basecount} that computes the number of a given base $b$ appearing in a RNA strand $s$ is defined recursively by $basecount : S \times B \to\mathbb{N}$

\vspace{-20pt}

\begin{quote}
Basis step: If  $b_1 \in B$, $b_2 \in B$, $basecount(b_1, b_2) =
        \begin{cases}
            1 & \textrm{when } b_1 = b_2 \\
            0 & \textrm{when } b_1 \neq b_2 \\
        \end{cases}$

Recursive Step: If $s \in S$, $b_1 \in B$, $b_2 \in B$, $basecount(s b_1, b_2) =
        \begin{cases}
            1 + \textit{basecount}(s, b_2) & \textrm{when } b_1 = b_2 \\
            \textit{basecount}(s, b_2) & \textrm{when } b_1 \neq b_2 \\
        \end{cases}$
\end{quote}

\vspace{-20pt}

Prove or disprove $\exists s \in S \, (\textit{rnalen(s)} = \textit{basecount}(s, \A))$:

\vfill

Prove or disprove $\forall s \in S \, (\textit{rnalen(s)} \geq \textit{basecount}(s, \A))$:

{\bf Proof}: 

{\bf Basis case}: Assume $s = \A \lor s = \C \lor s = \U \lor s = \G$. 
Need to show $\textit{rnalen(s)} \geq \textit{basecount}(s, \A)$.

{\it Case 1}: Want to show $(s = \A) \to (~\textit{rnalen(s)} \geq \textit{basecount}(s, \A)~)$.

\vfill

{\it Case 2}: Want to show $(s = \C \lor S = \U \lor S = \G) \to (~\textit{rnalen(s)} \geq \textit{basecount}(s, \A)~)$.

\vfill

\vfill

{\it Continued next page} 

\fbox{\parbox{\textwidth}{%
{\bf Proof by universal generalization}: To prove that $\forall x \, P(x)$
is true, we can take an arbitrary element $e$ from the domain and show that $P(e)$ is true, without making any assumptions about $e$ other than that it comes from the domain.

{\bf New! Proof by Structural Induction} (Rosen 5.3 p354) To prove a universal quantification over a recursively defined set:

\vspace{-10pt}


\begin{itemize}
\item[] {\bf Basis Step}:  Show the statement holds for elements specified in the basis step of the definition.
\item[]  {\bf Recursive Step}:  Show that if the statement is true for each of the elements used to construct
new elements in the recursive step of the definition, the result holds for these new elements.
\end{itemize}

\vspace{-10pt}

}}


\newpage


{\bf Recursive case}: Want to show
\[
\forall e \in S ~\left(~ \textit{rnalen(e)} \geq \textit{basecount}(e, \A) \to 
 \forall b \in B ~(~\textit{rnalen(eb)} \geq \textit{basecount}(eb, \A)~) ~\right)
 \]

Consider arbitrary $e$. Assume, as the {\bf induction hypothesis} that 
\[
\textit{rnalen(e)} \geq \textit{basecount}(e, \A)
\]

Need to show 
\[
 \forall b \in B ~(~\textit{rnalen(eb)} \geq \textit{basecount}(eb, \A)~) 
\]
Consider arbitrary $b \in B$. 


{\it Case 1}: Want to show $(b = \A) \to (~\textit{rnalen(eb)} \geq \textit{basecount}(eb, \A)~) $.

\vfill

{\it Case 2}: Want to show $(b = \C \lor b = \U \lor b = \G) \to (~\textit{rnalen(eb)} \geq \textit{basecount}(eb, \A)~) $.

\vfill
\vfill


\newpage


\fbox{\parbox{\textwidth}{%
{\bf Proof by Structural Induction} (Rosen 5.3 p354) To prove a universal quantification over a recursively defined set:

\vspace{-10pt}

\begin{itemize}
\item[] {\bf Basis Step}:  Show the statement holds for elements specified in the basis step of the definition.
\item[]  {\bf Recursive Step}:  Show that if the statement is true for each of the elements used to construct
new elements in the recursive step of the definition, the result holds for these new elements.
\end{itemize}

\vspace{-10pt}

}}

{\bf Definition} The set of natural numbers (aka nonnegative integers), $\mathbb{N}$, is defined (recursively) by:
\[
\begin{array}{ll}
\textrm{Basis Step: } & 0 \in \mathbb{N} \\
\textrm{Recursive Step: } & \textrm{If } n \in \mathbb{N} \textrm{ then } n+1 \in \mathbb{N} \textrm{ (where $n+1$ is integer addition)}
\end{array}
\]

The function $sumPow$ with domain $\mathbb{N}$, codomain
$\mathbb{N}$, and which computes, for input $i$, the sum of the first $i$ powers of $2$ is defined
recursively by $sumPow: \mathbb{N} \to \mathbb{N}$ with

\vspace{-20pt}

\begin{quote}
Basis step: $sumPow(0) = 1$.

Recursive step: If $x \in \mathbb{N}$ then $sumPow(x+1) = sumPow(x) + 2^{x+1}$.
\end{quote}

\vfill



Fill in the blanks in the following proof of $\forall n \in \mathbb{N} \, (sumPow(n) = 2^{n+1} - 1)$:

\vspace{-20pt}

\begin{quote}
Since $\mathbb{N}$ is recursively defined, we proceed by \underline{\phantom{structural induction \hspace{0.3in}}}.

{\bf Basis case} We need to show that \underline{\phantom{$sumPow(0) = 2^{0+1} - 1$ \hspace{0.2in}}}.
Evaluating each side: $LHS = sumPow(0) = 1$ by the basis case in the recursive definition
of $sumPow$; $RHS = 2^{0+1} - 1 = 2^1 - 1 = 2-1 = 1$. Since $1=1$, the equality holds.

{\bf Recursive step} Consider arbitrary natural number $n$ and assume, as the 
\underline{\phantom{Induction Hypothesis (IH)}} that $sumPow(n) = 2^{n+1} - 1$. We need to show that
\underline{\phantom{$sumPow(n+1) = 2^{(n+1) + 1} - 1$}}.  Evaluating each side: 
\[
LHS = sumPow(n+1) \overset{\text{rec def}}{=} sumPow(n)  + 2^{n+1}\overset{\text{IH}}{=} (2^{n+1} - 1) + 2^{n+1}.
\]
\[
RHS = 2^{(n+1)+1}- 1 \overset{\text{exponent rules}}{=} 2 \cdot 2^{n+1} -1  = \left(2^{n+1} + 2^{n+1} \right) - 1
\overset{\text{regrouping}}{=}  (2^{n+1} - 1) + 2^{n+1} 
\]
Thus, $LHS = RHS$. The structural induction is complete and we have proved the universal generalization.
\end{quote}



\vfill
{\it Extra example} Connect the function $sumPow$ to binary expansions of positive integers.

\newpage
{\bf Definition} The set of linked lists of natural numbers $L$ is defined:
\[
\begin{array}{ll}
    \textrm{Basis Step: } & [] \in L \\
    \textrm{Recursive Step: } & \textrm{If } l \in L\textrm{ and }n \in \mathbb{N} \textrm{, then } (n, l) \in L
\end{array}
\]

Examples: 

\vfill

{\bf Definition} The length of a linked list of natural numbers $L$, $\textit{length}: L \to \mathbb{N}$ is defined by:
\[
\begin{array}{llll}
\textrm{Basis Step:} &  & length([]) &= 0 \\
\textrm{Recursive Step:} & \textrm{If } l \in L\textrm{ and }n \in \mathbb{N}\textrm{, then  } & length((n, l))  &\phantom{= 1+ length(l)}
\end{array}
\]

Examples: 

\vfill


{\it Extra example:} The function $prepend : L \times \mathbb{N} \to L$ that adds an element at the front of a linked list is defined:
\[
\phantom{prepend(l, n) = (n, l)}
\]


\vfill

{\bf Definition} The function $\textit{append} : L \times \mathbb{N} \to L$ that adds an element at the end of a linked list is defined:
\[
\begin{array}{llll}
\textrm{Basis Step:} & \textrm{If } m \in \mathbb{N}\textrm{ then } & \phantom{\textit{append}([], m)} & \phantom{= (m, []) }\\
\textrm{Recursive Step:} & \textrm{If } l \in L\textrm{ and }n \in \mathbb{N}\textrm{ and }m \in \mathbb{N}\textrm{, then  } & \phantom{\textit{append}((n, l), m) } &\phantom{= (n, \textit{append}(l, m))}
\end{array}
\]

Examples: 

\vfill 

\newpage
{\bf Claim}: $\forall l \in L \, (~length(~append(l, 100)~) > length(l)~)$

{\bf Proof:} By structural induction on $L$, we have two cases:

\vspace{-10pt}

{\bf Basis Step}

\vspace{-10pt}

    \begin{tabular}{l p{3.5in}}
    1. \begin{tabular}{l}
        \textbf{To Show} $length(~append([], 100)~) > length([])$ \\
    \end{tabular}
    &  Because $[]$ is the only element defined in the basis step of $L$, we only need to 
prove that the property holds for $[]$.\\
    \\
    2. \begin{tabular}{l}
        \textbf{To Show} $length(~(100,[])~) > length([])$ \\
    \end{tabular}
    &  By basis step in definition of $append$.\\
    \\
    3. \begin{tabular}{l}
        \textbf{To Show} $(1 +length([])) > length([])$ \\
    \end{tabular}
    &  By recursive step in definition of $length$.\\
    \\    
    4. \begin{tabular}{l}
        \textbf{To Show} $1+0 > 0$ \\
    \end{tabular}
    &  By basis step in definition of $length$.\\
    \\       
    5. \begin{tabular}{l}
        \textbf{To Show } $T$ \\
    \end{tabular}
     & By properties of integers \\
    QED & Because we got to $T$ only by rewriting \textbf{To Show} to equivalent statements, using well-defined proof techniques, and applying definitions. \\
    \end{tabular}

\vspace{-10pt}

{\bf Recursive Step}

\vspace{-10pt}

Consider an arbitrary: $l = (n, l')$, $l' \in L$, $n \in \mathbb{N}$, and we  assume
as the {\bf induction hypothesis} that:
\[
length(~append(l', 100)~) > length(l')
\]
Our goal is to show that $length(~append( (n,l'), 100)~) > length((n,l'))$ is also true. We evaluate each side of the 
candidate inequality:
\begin{align*}
LHS &= length(~append( (n,l'), 100)~) = length(~(n, append(l', 100) )~) \qquad \text{by the recursive definition of $append$}\\
&= 1 + length(~ append(l', 100) ~) \qquad \text{by the recursive definition of $length$}\\
&> 1+ length(l')  \qquad \text{by the induction hypothesis}\\
&= length( (n,l') )  \qquad \text{by the recursive definition of $length$}\\
&= RHS 
\end{align*}

\end{document}
