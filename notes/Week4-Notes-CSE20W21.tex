\documentclass[12pt, oneside]{article}

\usepackage{amssymb,amsmath,pifont,amsfonts,comment,enumerate}
\usepackage{currfile,xstring,hyperref,tabularx,graphicx,wasysym}
\usepackage[labelformat=empty]{caption}
\usepackage[dvipsnames,table]{xcolor}
\usepackage{multicol,multirow,array,listings,tabularx,lastpage,textcomp,booktabs}

% NOTE(joe): This environment is credit @pnpo (https://tex.stackexchange.com/a/218450)
\lstnewenvironment{algorithm}[1][] %defines the algorithm listing environment
{   
    \lstset{ %this is the stype
        mathescape=true,
        frame=tB,
        numbers=left, 
        numberstyle=\tiny,
        basicstyle=\rmfamily\scriptsize, 
        keywordstyle=\color{black}\bfseries,
        keywords={,procedure, div, for, to, input, output, return, datatype, function, in, if, else, foreach, while, begin, end, }
        numbers=left,
        xleftmargin=.04\textwidth,
        #1
    }
}
{}
\lstnewenvironment{java}[1][]
{   
    \lstset{
        language=java,
        mathescape=true,
        frame=tB,
        numbers=left, 
        numberstyle=\tiny,
        basicstyle=\ttfamily\scriptsize, 
        keywordstyle=\color{black}\bfseries,
        keywords={, int, double, for, return, if, else, while, }
        numbers=left,
        xleftmargin=.04\textwidth,
        #1
    }
}
{}

\newcommand\abs[1]{\lvert~#1~\rvert}
\newcommand{\st}{\mid}

\newcommand{\A}[0]{\texttt{A}}
\newcommand{\C}[0]{\texttt{C}}
\newcommand{\G}[0]{\texttt{G}}
\newcommand{\U}[0]{\texttt{U}}

\newcommand{\cmark}{\ding{51}}
\newcommand{\xmark}{\ding{55}}



\begin{document}
\begin{flushright}
\StrBefore{\currfilename}{.}
\end{flushright}

\section*{This week's highlights}
\begin{itemize}
\item Evaluate the truth value of a compound proposition given truth values of its constituent variables.
\item Prove propositional equivalences using truth tables
\item Prove propositional equivalences using other known equivalences, e.g.
\begin{itemize}
\item DeMorgan's laws
\item Double negation laws
\item Distributive laws, etc.
\end{itemize}
\item Compute the CNF and DNF of a given compound proposition.
\item Define a predicate over a finite domain using a table of values and as properties
\item Determine the truth value of the proposition resulting from evaluating a predicate
\item Describe the set of domain elements that make a predicate with one input evaluate to true.
\item Evaluate universal and existential statements about finite domains (with no quantifier alternations).
\item Counterexample and witness-based arguments for predicates with infinite domains
\item Practice combinations of $\land$, $\rightarrow$ in conjunction with universal and existential quantifiers
\item State and apply DeMorgan's law for quantified statements.
\item Translate sentences from English to propositional logic using appropriate propositional variables and boolean operators.
\item Decide and justify whether or not a collection of propositions is consistent.

\item Use predicates with set of tuples as their domain to relate values to one another
\item Evaluate nested quantifiers: both alternating and not.
\end{itemize}

\section*{Lecture videos}
Week 4 Day 1
\href{https://youtube.com/playlist?list=PLML4QilACLk47QBNBgn3XU9VKJteswzfV}{YouTube playlist}

Week 4 Day 2
\href{https://youtube.com/playlist?list=PLML4QilACLk7ZmUbf4VXu8oUqcVmPMl4l}{YouTube playlist}

Week 4 Day 3
\href{https://youtube.com/playlist?list=PLML4QilACLk496T5XZ30JKU-t1iVaHIXm}{YouTube playlist}

\newpage
\section*{Monday January 25}
\begin{tabular}{lp{4in}p{2in}}
{\bf Logical equivalence } &Two compound  propositions are {\bf logically  equivalent} means that  they 
have the  same  truth  values for all settings of truth  values to their propositional  variables.\\
{\bf Tautology} & A compound proposition that evaluates to true
for all settings of truth  values to its propositional  variables; it is  abbreviated $T$.\\
{\bf Contradiction} & A compound proposition that  evaluates  to  false 
for  all settings of truth  values to its propositional  variables; it  is abbreviated $F$.\\
{\bf Contingency} & A compound proposition that is neither a tautology nor a contradiction.\\
\end{tabular}


{\it Can replace $p$ and $q$ with any compound proposition}


 {\bf Definition}: A collection of  compound  propositions
is called {\bf consistent} if  there
is  an assignment  of  truth values
to  the  propositional variables that makes
each of the compound propositions  true.

{\bf Consistency}: 
\begin{quote}
Whenever the system software is being upgraded, users cannot access the file system. 
If users can access the file system, then they can save new files. 
If users cannot save new files, then the system software is not being upgraded.
\end{quote}

\begin{enumerate}
\item Translate to symbolic compound propositions
\vfill
\item Look for some truth assignment to the propositional variables for which all the compound propositions output $T$
\vfill
\end{enumerate}


\newpage
Consider the following algorithm to introduce redundancy in a string of $0$s and $1$s.
\begin{algorithm}[caption={Create redundancy by repeating each bit three times}]
procedure $\textit{redun3}$($a_{k-1} \cdots a_0$: a binary string)
for $i$ := $0$ to $k-1$
  $c_{3i}$ := $a_i$
  $c_{3i+1}$ := $a_i$
  $c_{3i+2}$ := $a_i$
return $c_{3k-1} \cdots c_0$
\end{algorithm}

\begin{algorithm}[caption={Decode sequence of bits using majority rule on consecutive three bit sequences}]
procedure $\textit{decode3}$($c_{3k-1} \cdots c_0$: a binary string whose length is an integer multiple of $3$)
for $i$ := $0$ to $k-1$
  if exactly two or three of $c_{3i}, c_{3i+1}, c_{3i+2}$ are set to $1$
    $a_i$ := 1
  else 
    $a_i$ := 0
return $a_{k-1} \cdots a_0$
\end{algorithm}

Give a recursive definition of the set of outputs of the $redun3$ procedure, $Out$,

\vspace{-20pt}

\begin{itemize}
\item[] {\bf Basis step}: \underline{ \phantom{$\lambda \in Out$ \qquad}}
\item[] {\bf Recursive step}: \underline{\phantom{If $x \in Out$ then $x000 \in Out$ and $x111 \in Out$} }
%(where $x000$ and $x111$
%are the results of string concatenation).
\end{itemize}


Consider the message $m = 0001$ so that the sender calculates $redun3(m) = redun3(0001) = 000000000111$.

Introduce $\underline{\phantom{~~4~~}} $
errors into the message so that the signal received by the 
receiver is $\underline{\phantom{010100010101}}$
but the receiver is still able to decode the original message.

\vspace{-10pt}

{\it Challenge: what is the biggest number of errors you can introduce?} 

Building a circuit for line 3 in $decode$ procedure: given three input bits, we need to determine whether the
majority is a $0$ or a $1$.

\begin{center}
\begin{multicols}{2}\begin{tabular}{ccc|c}
$c_{3i}$ & $c_{3i+1}$ & $c_{3i+2}$ & $a_i$ \\
\hline
$1$ & $1$ & $1$ & $\phantom{1}$ \\
$1$ & $1$ & $0$ & $\phantom{1}$ \\
$1$ & $0$ & $1$ & $\phantom{1}$ \\
$1$ & $0$ & $0$ & $\phantom{0}$ \\
$0$ & $1$ & $1$ & $\phantom{1}$ \\
$0$ & $1$ & $0$ & $\phantom{0}$ \\
$0$ & $0$ & $1$ & $\phantom{0}$ \\
$0$ & $0$ & $0$ & $\phantom{0}$ \\\\
\end{tabular}
\columnbreak

Circuit 
\end{multicols}
\end{center}

\newpage
\section*{Wednesday January 27}

{\bf  Definition}: A  {\bf predicate}  is  a function from a given set (domain) to $\{T,F\}$.

A predicate can be applied, or {\bf evaluated} at, an element of the domain.

Two predicates over the same domain are {\bf equivalent} means they evaluate to
the same truth values for all possible assignments of domain elements to the
input.


\begin{center}
\begin{tabular}{c||c|c|c}
Input & \multicolumn{3}{c}{Output} \\
&$P(x)$ & $N(x)$ & $Mystery(x)$\\
$x$ & $[x]_{2c,3} > 0$ & $[x]_{2c,3} < 0$& \\
\hline
$000$  & $F$ & & $T$\\
$001$  & $T$ & & $T$\\
$010$  & $T$ & & $T$\\
$011$  & $T$ & & $F$\\
$100$  & $F$ & & $F$\\
$101$  & $F$ & & $T$\\
$110$  & $F$ & & $F$\\
$111$  & $F$ & & $T$\\
\end{tabular}
\end{center}

The domain for each of the predicates $P(x)$, $N(x)$, $Mystery(x)$ is
\underline{\phantom{$\{ b_1b_2b_3 ~\mid~ b_i \in \{0,1\} \textrm{ for each } i, 1 \leq i \leq 3 \}$}}.


{\bf Definition}: The {\bf truth  set} of a  predicate is the collection of all elements in its
domain where the predicate evaluates to $T$.

The truth set for the predicate $P(x)$ is $\underline{\phantom{\{ x ~\mid~ P(x) = T\} = \{ 001, 010, 011 \}}}$.

The truth set for the predicate $N(x)$ is $\underline{\phantom{\{ x ~\mid~ N(x) = T\} = \{ 101, 111 \}}}$.

The truth set for the predicate $Mystery(x)$ is $\underline{\phantom{\{ x ~\mid~ Mystery(x) = T\} = \{ 000, 001, 010, 101, 111 \}}}$.


\vfill


\newpage

{\bf Definitions} (Rosen 40-45): 

\vspace{-15pt}

The {\bf universal quantification} of $P(x)$ is the statement ``$P(x)$ for all values of $x$ in the domain''
and is written $\forall x P(x)$. An element for which $P(x) = F$ is called a {\bf counterexample} of $\forall x P(x)$.

The {\bf existential quantification} of $P(x)$ is the statement ``There exists an element $x$ in the domain such that $P(x)$'' and is written $\exists x P(x)$. An element for which $P(x) = T$ is called a {\bf witness} of $\exists x P(x)$.

{\bf Example}: 
\underline{\phantom{$\exists x ~(P(x) \to N(x))$}} is a true existential quantification.

Statements involving predicates and quantifiers are {\bf logically equivalent} 
means they have the same truth value no matter which predicates (domains and functions) are substituted in. 

{\bf Quantifier version of De Morgan's laws}: 
$\boxed{\neg \forall x P(x) ~\equiv~ \exists x \left( \neg P(x) \right)}$
\qquad
\qquad
$\boxed{\neg \exists x Q(x) ~\equiv~ \forall x \left( \neg Q(x) \right)}$


{\bf Example}: 
\underline{\phantom{$\forall x ~(P(x) \lor N(x))$}} is a false universal quantification. It is logically equivalent to \underline{\phantom{$\exists x ~\not (P(x) \lor N(x))$}}





\vfill

Recall: Each RNA strand is a string whose symbols are elements of the set $B  = \{\A, \C, \G, \U \}$.
The {\bf set of all RNA strands} is called $S$.
The function \textit{rnalen} that computes the length of RNA strands in $S$ is:
\[
\begin{array}{llll}
& & \textit{rnalen} : S & \to \mathbb{Z}^+ \\
\textrm{Basis Step:} & \textrm{If } b \in B\textrm{ then } & \textit{rnalen}(b) & = 1 \\
\textrm{Recursive Step:} & \textrm{If } s \in S\textrm{ and }b \in B\textrm{, then  } & \textit{rnalen}(sb) & = 1 + \textit{rnalen}(s)
\end{array}
\]
{\bf Example predicates on $S$}

\vspace{-20pt}


\begin{center}
\begin{tabular}{|p{4in}p{3.5in}|}
\hline
& \\
$H(s) = T$ & Truth set of $H$ is \underline{\phantom{$S$\hspace{1in}}}\\
\hline
$L_3(s) = \begin{cases}
T &\qquad\text{if $rnalen(s) = 3$} \\
F & \qquad\text{otherwise}
\end{cases}$ & 
Strand where $L_3$ evaluates to $T$ is e.g.\underline{\phantom{$\A\A\A$~\hspace{0.3in}}}

\vspace{10pt}

Strand where $L_3$ evaluates to $F$ is e.g. \underline{\phantom{$\A\A\U\A$\hspace{0.3in}}}\\
\hline
& \\
$F_{\A}$ is defined recursively by: 

~~Basis step: $F_{\A}(\A) = T$, $F_{\A}(\C) = F_{\A}(\G) = F_{\A}(\U) = F$

~~Recursive step: If $s \in S$ and $b \in B$, then $F_{\A}(sb) = F_{\A}(s)$& 
Strand where $F_{\A}$ evaluates to $T$ is e.g.\underline{\phantom{$\A\C\G$~\hspace{0.3in}}}

\vspace{10pt}

Strand where $F_{\A}$ evaluates to $F$ is e.g. \underline{\phantom{$\U\A\C\U$\hspace{0.3in}}}\\
\hline
& \\
$P_{\A\U\C}$ is defined as the predicate whose truth set
is the collection of RNA strands where the string $\A\U\C$
is a substring (appears inside $s$, in order and consecutively)& 
Strand where $P_{\A\U\C}$ evaluates to $T$ is e.g.\underline{\phantom{$\A\A\A$~\hspace{0.3in}}}

\vspace{10pt}

Strand where $P_{\A\U\C}$ evaluates to $F$ is e.g. \underline{\phantom{$\A\A\U\A$\hspace{0.3in}}}\\
\hline
\end{tabular}
\end{center}

\vfill

\section*{Friday January 29}


{\bf Definition} (Rosen p123): The {\bf Cartesian product} of the sets $A$ and $B$, $A \times B$, is the set of all ordered pairs $(a, b)$, where $a \in A$ and $b \in B$. 
That is: $A \times B = \{(a, b) \mid (a \in A) \land (b \in B)\}.$ The Cartesian product of the sets $A_1, A_2, \ldots ,A_n$, denoted by $A_1 \times A_2 \times \cdots \times A_n$, is the
set of ordered n-tuples $(a_1, a_2,...,a_n)$, where $a_i$ belongs to $A_i$ for $i = 1, 2,\ldots,n$. That is,
$A_1 \times A_2 \times \cdots \times A_n = \{(a_1, a_2,\ldots,a_n) \mid a_i \in A_i \textrm{ for } i = 1, 2,\ldots,n\}$



Recall: Each RNA strand is a string whose symbols are elements of the set $B  = \{\A, \C, \G, \U \}$.
The {\bf set of all RNA strands} is called $S$.
The function \textit{rnalen} that computes the length of RNA strands in $S$ is:
\[
\begin{array}{llll}
& & \textit{rnalen} : S & \to \mathbb{Z}^+ \\
\textrm{Basis Step:} & \textrm{If } b \in B\textrm{ then } & \textit{rnalen}(b) & = 1 \\
\textrm{Recursive Step:} & \textrm{If } s \in S\textrm{ and }b \in B\textrm{, then  } & \textit{rnalen}(sb) & = 1 + \textit{rnalen}(s)
\end{array}
\]
A function \textit{basecount} that computes the number of a given base $b$ appearing in a RNA strand $s$ is:
\[
\begin{array}{llll}
& & \textit{basecount} : S \times B & \to \mathbb{N} \\
\textrm{Basis Step:} &  \textrm{If } b_1 \in B, b_2 \in B & \textit{basecount}(b_1, b_2) & =
        \begin{cases}
            1 & \textrm{when } b_1 = b_2 \\
            0 & \textrm{when } b_1 \neq b_2 \\
        \end{cases} \\
\textrm{Recursive Step:} & \textrm{If } s \in S, b_1 \in B, b_2 \in B &\textit{basecount}(s b_1, b_2) & =
        \begin{cases}
            1 + \textit{basecount}(s, b_2) & \textrm{when } b_1 = b_2 \\
            \textit{basecount}(s, b_2) & \textrm{when } b_1 \neq b_2 \\
        \end{cases}
\end{array}
\]

\vfill

\begin{tabular}{p{3.5in}p{0.1in}p{3.5in}}
$L$ with domain $S \times \mathbb{Z}^+$ is defined by, for $s \in S$ and $n \in \mathbb{Z}^+$,
\[
L( s, n) = \begin{cases}
T &\qquad\text{if $rnalen(s) = n$}\\
F &\qquad\text{otherwise}\\
\end{cases}
\]
&&
$BC$ with domain $\underline{\phantom{S \times B \times \mathbb{N}}}$ is defined by, for $s \in S$ and $b \in B$ and $n \in \mathbb{N}$,
\[
BC( s, b, n) = \begin{cases}
T &\qquad\text{if $basecount(s,b) = n$}\\
F &\qquad\text{otherwise}\\
\end{cases}
\]\\
Element where $L$ evaluates to $T$: $\underline{\phantom{(\A, 1)\hspace{0.2in}}}$ & & Element where $BC$ evaluates to $T$: 
$\underline{\phantom{(\A, \A1)\hspace{0.2in}}}$ \\
& & \\
& & \\
Element where $L$ evaluates to $F$: $\underline{\phantom{(\A, 2)\hspace{0.2in}}}$  & & Element where $BC$ evaluates to $F$: 
$\underline{\phantom{(\A, \C, 1)\hspace{0.2in}}}$ \\
& & \\
& & \\
\end{tabular}

\vfill

{\bf Notation}: for a predicate $P$ with domain $X_1 \times \cdots \times X_n$ and a $n$-tuple $(x_1, \ldots, x_n)$ 
with each $x_i \in X$, we write $P(x_1, \ldots, x_n)$ to mean $P( ~(x_1, \ldots, x_n)~)$.


\newpage

$\exists t ~BC(t)$ \qquad In English: \underline{\phantom{There exists an ordered $3$-tuple 
%of strand, base, and nonnegative integer
at which the predicate $BC$ evaluates to $T$.}}

Witness that proves this existential quantification is true: \underline{\phantom{$(\G\G, \G, 2)$ or $(\G\A\U\G, \G, 2)$)}}

$\forall (s,b,n) ~(~BC(s,b,n)~)$  \qquad In English: \underline{\phantom{For all ordered $3$-tuples
%of strand, base, and nonnegative integer
the predicate $BC$ evaluates to $T$.}}

Counterexample that proves this universal quantification is false: \underline{\phantom{$(\G\G, \A, 2)$ or $(\G\A\U\G, \G, 3)$)}}


{\bf New predicates from old} \qquad $BC(s,b,n)$ means $basecount(s,b) = n$.
\begin{center}
\begin{tabular}{|p{2.5in}p{1.5in}p{2in}|}
\hline
{\bf Predicate} & Domain &  Example domain element \\
& & where predicate is $T$\\
\hline
&& \\
$basecount(s,b) = 3$ & \phantom{$S \times B$}& 
\phantom{$(\A\U\A\A, \A)$}\\
\hline
&& \\
$basecount(s,\A) = n $&\phantom{$S \times \mathbb{N}$} & 
\phantom{$(\A\U\A,2)$}\\
\hline
&& \\
$\exists n \in \mathbb{N} ~(basecount(s,b) = n) $&\phantom{$S \times \mathbb{N}$} & 
\phantom{$(\A\U\A,2)$}\\
\hline
&& \\
$\forall b \in B ~(basecount(s,b) = 1) $&\phantom{$S$} & 
\phantom{$\A\C\G\U$}\\
\hline
\end{tabular}
\end{center}


{\bf Alternating quantifiers}
$$\forall s \exists n ~BC(s,\A,n)$$

In English: \underline{\phantom{\hspace{3in}}}

\vfill

$$\exists n \forall s ~BC(s,\U,n)$$

In English: \underline{\phantom{\hspace{3in}}}


\vfill

Evaluate each quantified statement as $T$ or $F$.

\vspace{-20pt}

\begin{center}
\begin{tabular}{|c|c|c|}
\hline
&&\\
$\forall s ~\forall b ~\exists n~BC(s,b,n)$ & $\forall s ~\forall n ~ \exists b~BC(s,b,n)$ & $\forall b ~\forall n ~\exists s~BC(s,b,n)$ \\
&&\\
&&\\
\hline
&&\\
$\exists s ~\forall b ~\exists n~BC(s,b,n)$ & $\forall s~ \exists n ~\forall b~BC(s,b,n)$ & $\exists b ~\exists n ~\forall s~BC(s,b,n)$ \\
&&\\
&&\\
\hline
\end{tabular}
\end{center}

{\it Extra example}: Write the negation of each of the statements above, and use De Morgan's law to find a 
logically equivalent version where the negation is applied only to the $BC$ predicate (not next to a quantifier).




\section*{Review quiz questions}
\begin{enumerate}
\item {\bf Monday}
Real-life representations are often prone to corruptions.  Biological codes, like RNA, 
may mutate naturally\footnote{Mutations of specific RNA codons have been linked to many disorders and cancers.}
and during measurement; cosmic radiation and other ambient noise 
can flip bits in computer storage\footnote{This RadioLab podcast episode
goes into more detail on bit flips: \url{https://www.wnycstudios.org/story/bit-flip}}. 
One way to recover from corrupted data is to exploit redundancy.  
Consider the following algorithm to introduce redundancy in a string of $0$s and $1$s.
\begin{algorithm}[caption={Create redundancy by repeating each bit three times}]
procedure $\textit{redun3}$($a_{k-1} \cdots a_0$: a binary string)
for $i$ := $0$ to $k-1$
  $c_{3i}$ := $a_i$
  $c_{3i+1}$ := $a_i$
  $c_{3i+2}$ := $a_i$
return $c_{3k-1} \cdots c_0$
\end{algorithm}

\begin{algorithm}[caption={Decode sequence of bits using majority rule on consecutive three bit sequences}]
procedure $\textit{decode3}$($c_{3k-1} \cdots c_0$: a binary string whose length is an integer multiple of $3$)
for $i$ := $0$ to $k-1$
  if exactly two or three of $c_{3i}, c_{3i+1}, c_{3i+2}$ are set to $1$
    $a_i$ := 1
  else 
    $a_i$ := 0
return $a_{k-1} \cdots a_0$
\end{algorithm}

For each of the following, type in your answers precisely including all notational punctuation.

\begin{enumerate}
\item  Give the output of $redun3(100)$.
\item  If the output of running $redun3$ is $000000111000111$, what was its input?
\item  Give the output of $decode3(100)$.
\item  How many distinct possible inputs to $decode3$ give the output $01$?
\end{enumerate}

\item {\bf Wednesday} Consider the following predicates, each of which has 
as its domain the set of all bitstrings whose leftmost bit is $1$

$E(x)$ is $T$ exactly when $(x)_{2}$ is even, and is $F$ otherwise

$L(x)$ is $T$ exactly when $(x)_2 < 3$, and is $F$ otherwise

$M(x)$ is $T$ exactly when $(x)_2 > 256$ and is $F$ otherwise.

\begin{enumerate}
\item What is $E(110)$?
\item Why is $L( 00)$ undefined?
\begin{enumerate}
\item Because the domain of $L$ is infinite
\item Because $00$ does not have $1$ in the leftmost position
\item Because $00$ has length 2, not length 3
\item Because $(00)_{2,3} = 0$ which is less than $3$
\end{enumerate}
\item Is there a bitstring of width (where width is the number of bits) $6$ at which $M(x)$ evaluates 
to $T$?
\end{enumerate}

\newpage
\item {\bf Friday} Recall that $S$ is defined as the set of all RNA strands, strings made of the bases in 
 $B = \{\A,\U,\G,\C\}$. Define the functions \textit{mutation}, \textit{insertion}, and \textit{deletion} as described by the pseudocode below:

\begin{algorithm}
procedure $\textit{mutation}$($b_1\cdots b_n$: $\textrm{a RNA strand}$, $k$: $\textrm{a  positive integer}$, $b$: $\textrm{an  element of } B$)
for $i$ := $1$ to $n$
  if $i$ = $k$
    $c_i$ := $b$
  else
    $c_i$ := $b_i$
return $c_1\cdots c_n$ $\{ \textrm{The return value is a RNA strand made of the } c_i \textrm{ values}\}$
\end{algorithm}

\begin{algorithm}
procedure $\textit{insertion}$($b_1\cdots b_n$: $\textrm{a RNA strand}$, $k$: $\textrm{a  positive integer}$, $b$: $\textrm{an  element of } B$)
if $k > n$
  for $i$ := $1$ to $n$
    $c_i$ := $b_i$
  $c_{n+1}$ := $b$
else 
  for $i$ := $1$ to $k-1$
    $c_i$ := $b_i$
  $c_k$ := $b$
  for $i$ := $k+1$ to $n+1$
    $c_i$ := $b_{i-1}$
return $c_1\cdots c_{n+1}$ $\{ \textrm{The return value is a RNA strand made of the } c_i \textrm{ values}\}$
\end{algorithm}

\begin{algorithm}
procedure $\textit{deletion}$($b_1\cdots b_n$: $\textrm{a RNA strand}$, $k$: $\textrm{a  positive integer}$)
if $k > n$
  $m$ := $n$
  for $i$ := $1$ to $n$
    $c_i$ := $b_i$
else
  $m$ := $n-1$
  for $i$ := $1$ to $k-1$ 
    $c_i$ := $b_i$
  for $i$ := $k$ to $n-1$
    $c_i$ := $b_{i+1}$
return $c_1\cdots c_m$ $\{ \textrm{The return value is a RNA strand made of the } c_i \textrm{ values}\}$
\end{algorithm}

For this question, we will use the following predicates.

$F_{\A}$ with domain $S$ is defined recursively by: 
\begin{itemize}
\item[]Basis step: $F_{\A}(\A) = T$, $F_{\A}(\C) = F_{\A}(\G) = F_{\A}(\U) = F$
\item[]Recursive step: If $s \in S$ and $b \in B$, then $F_{\A}(sb) = F_{\A}(s)$
\end{itemize}

$P_{\A\U\C}$ with domain $S$ is defined as the predicate whose truth set
is the collection of RNA strands where the string $\A\U\C$
is a substring (appears inside $s$, in order and consecutively)

$L$ with domain $S \times \mathbb{Z}^+$ is defined by, for $s \in S$ and $n \in \mathbb{Z}^+$,
\[
L( s, n) = \begin{cases}
T &\qquad\text{if $rnalen(s) = n$}\\
F &\qquad\text{otherwise}\\
\end{cases}
\]

$Mut$ with domain $S \times S$ is defined by, for $s_1 \in S$ and $s_2 \in S$,
\[
Mut(s_1,s_2) = \exists k\in \mathbb{Z^+} \exists b \in B (~ mutation(s_1, k, b) = s_2~)
\]
$Ins$ with domain $S \times S$ is defined by, for $s_1 \in S$ and $s_2 \in S$,
\[
Ins(s_1,s_2) = \exists k\in \mathbb{Z^+} \exists b \in B (~ insertion(s_1, k, b) = s_2~)
\]
$Del$ with domain $S \times S$ is defined by, for $s_1 \in S$ and $s_2 \in S$,
\[
Del(s_1,s_2) = \exists k\in \mathbb{Z^+} (~ deletion(s_1, k) = s_2~)
\]

\begin{enumerate}
\item Which of the following is true? (Select  all and only that apply.)
 \begin{enumerate}
    \item $F_\A ( \A\A)$
    \item $F_\A ( \A\C)$
    \item $F_\A ( \A\G)$
    \item $F_\A ( \A\U)$
    \item $F_\A ( \C\A)$
    \item $F_\A ( \C\C)$
    \item $F_\A ( \C\G)$
    \item $F_\A ( \C\U)$
 \end{enumerate}    
\item Which of the following is true? (Select  all and only that apply.)
 \begin{enumerate}
    \item $\exists s \in S ~\exists n \in \mathbb{Z}^+ ~(L(s,n))$
    \item $\exists s \in S ~\forall n \in \mathbb{Z}^+ ~(L(s,n))$
    \item $\forall n \in \mathbb{Z}^+~\exists s \in S ~(L(s,n))$
    \item $\forall s \in S ~\exists n \in \mathbb{Z}^+ ~(L(s,n))$
    \item $\exists n \in \mathbb{Z}^+ ~\forall s \in S ~(L(s,n))$
    \item $\forall s \in S ~\forall n \in \mathbb{Z}^+ ~(L(s,n))$
 \end{enumerate} 
\item Which of the following is true? (Select  all and only that apply.)
 \begin{enumerate}
    \item $\exists s \in S ~Mut(s,s)$
    \item $\forall s \in S ~Mut(s,s)$
    \item $\exists s \in S ~Ins(s,\A)$
    \item $\exists s \in S ~Ins(\A,s)$
    \item $\exists s \in S ~Del(s,\A)$
    \item $\forall s \in S ~Del(s,\A)$
 \end{enumerate} 
\end{enumerate}

\end{enumerate}



\end{document}
