\documentclass[12pt, oneside]{article}

\usepackage{amssymb,amsmath,pifont,amsfonts,comment,enumerate,enumitem}
\usepackage{currfile,xstring,hyperref,tabularx,graphicx,wasysym}
\usepackage[labelformat=empty]{caption}
\usepackage[dvipsnames,table]{xcolor}
\usepackage{multicol,multirow,array,listings,tabularx,lastpage,textcomp,booktabs}

% NOTE(joe): This environment is credit @pnpo (https://tex.stackexchange.com/a/218450)
\lstnewenvironment{algorithm}[1][] %defines the algorithm listing environment
{   
    \lstset{ %this is the stype
        mathescape=true,
        frame=tB,
        numbers=left, 
        numberstyle=\tiny,
        basicstyle=\rmfamily\scriptsize, 
        keywordstyle=\color{black}\bfseries,
        keywords={,procedure, div, for, to, input, output, return, datatype, function, in, if, else, foreach, while, begin, end, }
        numbers=left,
        xleftmargin=.04\textwidth,
        #1
    }
}
{}
\lstnewenvironment{java}[1][]
{   
    \lstset{
        language=java,
        mathescape=true,
        frame=tB,
        numbers=left, 
        numberstyle=\tiny,
        basicstyle=\ttfamily\scriptsize, 
        keywordstyle=\color{black}\bfseries,
        keywords={, int, double, for, return, if, else, while, }
        numbers=left,
        xleftmargin=.04\textwidth,
        #1
    }
}
{}

\newcommand\abs[1]{\lvert~#1~\rvert}
\newcommand{\st}{\mid}

\newcommand{\A}[0]{\texttt{A}}
\newcommand{\C}[0]{\texttt{C}}
\newcommand{\G}[0]{\texttt{G}}
\newcommand{\U}[0]{\texttt{U}}

\newcommand{\cmark}{\ding{51}}
\newcommand{\xmark}{\ding{55}}


\usepackage{enumitem}

\begin{document}
\begin{flushright}
\StrBefore{\currfilename}{.}
\end{flushright}

\section*{This week's highlights}
\begin{itemize}
\item Determine what evidence is required to establish that a quantified statement is true or false.
\item Use logical equivalence to rewrite quantified statements (including negated quantified statements)
\item Use universal generalization to prove that universal statements are true
\item Define predicates associated with integer factoring and primes
\item Define ``arbitrary"
\item Add to repertoire of proof strategies
\item Identify the main connective of a proposition and associated proof strategies
\item Determine whether a proposition is true or false using valid reasoning (proofs) in multiple contexts
\end{itemize}

\section*{Lecture videos}
Week 5 Day 1
\href{https://youtube.com/playlist?list=PLML4QilACLk6mhPkI6360G-jsgYetoTN1}{YouTube playlist}

Week 5 Day 2
\href{https://youtube.com/playlist?list=PLML4QilACLk7ooxYgPtpZWI-5UULS5sfm}{YouTube playlist}

Week 5 Day 3
\href{https://youtube.com/playlist?list=PLML4QilACLk5N_Byc5vp3qZrK1L-7CTCW}{YouTube playlist}

\newpage
\section*{Monday February 1}


When a predicate $P(x)$ is over a {\bf finite} domain:
\begin{itemize}
\item To prove that $\forall x  P(x)$ is true: \underline{\phantom{\hspace{4.6in}}}
\item To prove that $\forall x  P(x)$ is false: \underline{\phantom{\hspace{4.6in}}}
\item To prove that $\exists x  P(x)$ is true: \underline{\phantom{\hspace{4.6in}}}
\item To prove that $\exists x  P(x)$ is false: \underline{\phantom{\hspace{4.6in}}}
\end{itemize}

\fbox{\parbox{\linewidth}{%
{\bf Proof of universal by exhaustion}: To prove that $\forall x \, P(x)$
is true when $P$ has a finite domain, evaluate the predicate at {\bf each} domain element to confirm that it is always T.
}}


{\bf Some sets of numbers}

\vspace{-20pt}

\begin{center}
\begin{tabular}{cllp{2.5in}}
$\mathbb{N}$  &  The set of  natural numbers & $\{ 0, 1, 2, 3, \ldots \}$ & {\it Recursively defined by}

Basis step: \phantom{$0 \in \mathbb{N}$} 

Recursive step:%: \phantom{If $x \in \mathbb{N}$ then $x+1 \in \mathbb{N}$}
\\
$\mathbb{Z}$  &  The set of  integers & $\{ \ldots ,-2, -1, 0, 1, 2, \ldots \}$ & {\it Recursively defined by}

Basis step: \phantom{$0 \in \mathbb{Z}$} 

Recursive step:%: \phantom{If $x \in \mathbb{Z}$ then $x+1 \in \mathbb{Z}$ and $x-1 \in \mathbb{Z}$}
\\
$\mathbb{Z}^+$ & The set of positive integers  & $\{ 1, 2, 3,  \ldots  \}$ & {\it Set builder notation definition is} $\{  x \in \mathbb{N} \mid x > 0 \} = \{  x \in \mathbb{Z} \mid x > 0 \}$ \\
$\mathbb{Z}^{\neq 0}$ & The set of nonzero integers & & {\it Set builder notation definition is} $\{  x \in \mathbb{Z} \mid ( x < 0 \lor x > 0) \}$

\end{tabular}
\end{center}

{\bf Factoring}


{\bf Definition} (Rosen  p. 238): When $a$ and $b$ are integers and $a$ is nonzero, 
{\bf $a$ divides $b$} means there is an integer $c$ such that $b = ac$ . 


Symbolically, $F(a,b) = $ \underline{\phantom{\hspace{2in}}} and is  a predicate over the domain \underline{\phantom{\hspace{1in}}}


\vfill 
Other (synonymous) ways to say that $F(a,b)$ is true: 
\begin{quote}
$a$ is a {\bf factor} of $b$
\qquad 
$a$ is a {\bf divisor} of $b$
\qquad  $b$ is a {\bf multiple} of $a$
\qquad $a | b$
\qquad $b \textbf{ mod } a = 0$
\end{quote}

\newpage
{\it Translate these quantified statements by matching to English statement on right.}

\begin{multicols}{2}
$\exists a\in \mathbb{Z}^{\neq 0} ~(~F(a,a)~)$

$\exists a\in \mathbb{Z}^{\neq 0} ~(~\lnot F(a,a)~)$

$\forall a\in \mathbb{Z}^{\neq 0} ~(~F(a,a)~)$

$\forall a\in \mathbb{Z}^{\neq 0} ~(~\lnot F(a,a)~)$


Every nonzero integer is a factor of itself.

No nonzero integer is a factor of itself.

At least one nonzero integer is a factor of itself.

Some nonzero integer is not a factor of itself.
\end{multicols}


{\bf Claim}: Every nonzero integer is a factor of itself.

{\bf Proof}: 


\vfill


{\bf Claim}: The statement ``There is a nonzero integer that does not divide its square" is True / False 

\vspace{-15pt}

\hfill{\it Circle one}

\vspace{-15pt}

{\bf Proof}: 


\vfill


\fbox{\parbox{\linewidth}{%

{\bf New! Proof by universal generalization}: To prove that $\forall x \, P(x)$
is true, we can take an arbitrary element $e$ from the domain and show that $P(e)$ is true, without making any assumptions about $e$ other than that it comes from the domain.


An {\bf arbitrary} element of a set or domain is a fixed but unknown element from that set. 
}}

\newpage

{\bf Definition} (Rosen p. 257):  An integer $p$ greater than $1$ is called {\bf prime} means 
the only positive factors of 
$p$ are $1$ and $p$.  A positive integer that is greater than $1$ and is not prime is called composite. 

A formal definition of the predicate $Pr$ over the domain $\mathbb{Z}$ which evaluates to T exactly when the input is prime is:
\phantom{$(x > 1) \land \forall a( ~ (~ a > 0 \land F(a,x) ~) \to (a = 1 \lor a = x) ~)$}

\vfill

{\bf Claim}: $1$ is not prime.

{\bf Proof}: 

\vfill

{\bf Claim}: $4$ is not prime.

{\bf Proof}: 

\vfill


\vfill
\fbox{\parbox{\linewidth}{%
$(p \to  q)  \equiv \neg (p \wedge \neg q)\qquad \qquad \neg(p \land q) \equiv \neg p \lor \neg q  \qquad  \qquad q \lor \neg p \equiv p \to q \qquad \qquad \neg \exists x \, P(x) \equiv \forall x \, \neg (P(x))$

\vspace{1em}

To prove that $\exists x P(x)$ is {\bf false}, write the universal statement that is logically equivalent to its negation and then prove it true using universal generalization.

\vspace{1em}

To prove that $p \land q$ is true, have two subgoals: subgoal (1) prove $p$ 
is  true; and, subgoal (2) prove $q$ is true.

\vspace{1em}

 To prove that $p \land q$ is false, it's enough to prove that $p$ is false.
 
 To prove that $p \land q$ is false, it's enough to prove that $q$ is false.
}}

\newpage
\section*{Wednesday February 3}


Recall the predicate $F(a,b) = \exists c \in \mathbb{Z} ~(b = ac)$  is  a predicate over the domain $\mathbb{Z}^{\neq 0} 
\times \mathbb{Z}$.  In English, $F(a,b)$ evaluates to $T$ means $a$ is a nonzero integer, $b$ is an integer, and 
$a$ is a factor of $b$.  An equivalent definition is that $F(a,b) = T$ exactly when $b \textbf{ mod } a = 0$.

{\bf Definition} (Rosen p. 257):  An integer $p$ greater than $1$ is called {\bf prime} means 
the only positive factors of $p$ are $1$ and $p$. We write $Pr(x)$ to indicate that an positive
integer $x$ is prime. A positive integer that is greater than $1$ and is not prime is called composite. 


{\bf Trial Division}: If $n$ is a composite integer, then $n$ has a prime divisor less than or equal to $\sqrt n$.



{\bf Claim}: The statement ``There are three consecutive positive integers that are prime." is True / False 

{\it Hint}: These numbers would be of the form $p, p+1, p+2$ (where $p$ is a positive integer).

{\bf Proof}: We need to show \underline{\phantom{$\exists p \in \mathbb{Z}^+ ~(~Pr(p) \land Pr(p+1) \land Pr(p+2)~)$}}

\vfill

{\bf Claim}: The statement ``There are three consecutive odd positive integers that are prime." is True / False 

{\it Hint}: These numbers would be of the form $p, p+2, p+4$ (where $p$ is an odd positive integer).

{\bf Proof}: We need to show \underline{\phantom{$\exists p \in \mathbb{Z}^+ ~(~(p \textbf{ mod } 2 = 1 \land Pr(p) \land Pr(p+2) \land Pr(p+4)~)$}}

\vfill


\fbox{\parbox{\linewidth}{%

{\bf Proof of universal by exhaustion}: To prove that $\forall x \, P(x)$
is true when $P$ has a finite domain, evaluate the predicate at {\bf each} domain element to confirm that it is always T.

\vspace{1em}

{\bf Proof by universal generalization}: To prove that $\forall x \, P(x)$
is true, we can take an arbitrary element $e$ from the domain and show that $P(e)$ is true, without making any assumptions about $e$ other than that it comes from the domain.

\vspace{1em}

To prove that $\exists x P(x)$ is {\bf false}, write the universal statement that is logically equivalent to its negation and then prove it true using universal generalization.

\vspace{1em}

To prove that $p \land q$ is true, have two subgoals: subgoal (1) prove $p$ 
is  true; and, subgoal (2) prove $q$ is true.

\vspace{1em}

 To prove that $p \land q$ is false, it's enough to prove that $p$ is false.
 
 To prove that $p \land q$ is false, it's enough to prove that $q$ is false.
}}

\newpage
Each Netflix user's viewing history can be represented as a $n$-tuple indicating their preferences about
movies in the database, where $n$ is the number of movies in the database.  Each element in the $n$-tuple indicates
the user's rating of the corresponding movie: $1$ indicates the person liked the movie, $-1$ that they didn't, and $0$ that 
they didn't rate it one way or another. Consider a four movie database. Recall the Netflix example from class: Consider a four movie database. We denote the set of possible ratings 
$\{-1,0,1\} \times \{-1,0,1\} \times \{-1,0,1\} \times \{-1,0,1\}$ as $R_4$. We have the functions
\[
d_{1,4}(~ (x_1, x_2, x_3, x_4) , (y_1, y_2, y_3, y_4) ~) =  \sum_{i=1}^4\left( (\abs{x_i-y_i} + 1) \textbf{ div } 2 \right)
\]
\[
d_{2,4}(~ (x_1, x_2, x_3, x_4) , (y_1, y_2, y_3, y_4) ~) =  \sqrt{ \sum_{i=1}^4 (x_i - y_i)^2}
\]


{\bf Claim}: The statement ``$\forall r_1 \in R_4 ~\forall r_2 \in R_4 ~(~r_1 = r_2 \to \neg \left( d_{1,4}(r_1,r_2) < d_{2,4} (r_1,r_2) \right)~)$" is True / False 


The statement in English: \underline{\hspace{5.5in}}

\phantom{When two ratings are equal then it's not the case that their $d_1$ distance
is less than their $d_2$ distance.}

\vfill



{\bf Claim}: The statement ``$\forall r_1 \in R_4 ~\forall r_2 \in R_4 ~(~d_{1,4}(r_1,r_2) < d_{2,4} (r_1,r_2)  ~)$" 
is True / False 

The statement in English: \underline{\hspace{5.5in}}

\phantom{When two ratings are equal then it's not the case that their $d_1$ distance
is less than their $d_2$ distance.}

\vfill

\section*{Friday February 5}


\begin{center}
\begin{tabular}{lp{2.5in}p{3.5in}}
{\bf  Term} & {\bf Definition}  & {\bf Examples} \\
\hline 
{\bf set} & an  unordered collection of  elements &  \\
&  &  \\
\hline
{\bf set  equality} & When $A$ and  $B$ are sets,  $A = B$ means  $\forall x  ( x\in A \leftrightarrow x \in B)$ & $\{ 43, 7, 9 \} = \{ 7, 43, 9, 7\}$
\newline $\left \{ \frac{p}{q} \mid p \in \mathbb{Z}, q \in \mathbb{Z}, q \neq  0 \right\} 
= \left \{ \frac{p}{q} \mid p \in \mathbb{Z}, q \in \mathbb{Z^+}\right\}$\\
&&\\
\hline
{\bf subset} & When $A$ and  $B$ are sets, $A \subseteq B$ means $\forall x  (x \in A  \to x  \in B)$ \\
&&\\
\hline
{\bf proper subset} &When $A$ and  $B$ are sets,  $A \subsetneq B$ means $(A\subseteq B) \wedge  (A \neq B)$ &
\\
&&\\
\hline
\end{tabular}
\end{center}


Claim: $\{ \A,  \C,  \U,  \G\} \subseteq \{ \A\A, \A\C, \A\U, \A\G \}$ 
\qquad {\bf Prove} or {\bf  disprove} \hfill{\it Circle one}

\vfill
\vfill


Claim: $\{ 4, 6 \} \subseteq \{ n ~\textbf{mod}~10 \mid  \exists c \in \mathbb{Z} ( n = 4c) \} $
\qquad {\bf Prove} or {\bf  disprove} \hfill{\it Circle one}

\vfill
\vfill


Claim: The empty set is a proper subset of every set.
\qquad {\bf Prove} or {\bf  disprove} \hfill{\it Circle one}

\vfill
\vfill

Claim: For  some  set $B$, $\emptyset \in B$.
\qquad
{\bf Prove} or {\bf  disprove} \hfill{\it Circle one}

\vfill
\vfill

\fbox{\parbox{\linewidth}{%

{\bf Proof by universal generalization}: To prove that $\forall x \, P(x)$
is true, we can take an arbitrary element $e$ from the domain and show that $P(e)$ is true, without making any assumptions about $e$ other than that it comes from the domain.


\vspace{1em}
{\bf Evidence for conjunction} being true or false: 

To prove that $p \land q$ is true, have two subgoals: subgoal (1) prove $p$ 
is  true; and, subgoal (2) prove $q$ is true.

 To prove that $p \land q$ is false, it's enough to prove that $p$ is false.
 
 To prove that $p \land q$ is false, it's enough to prove that $q$ is false.

\vspace{1em}

{\bf New! Proof by Cases}: To prove $q$, if we know that $p_1 \lor p_2$ is true, and we can show that $(p_1 \to q)$ is true and we can show that $(p_2 \to q)$, then we can conclude $q$ is true. Sec 1.8 p92



}}


\newpage

\begin{center}
\begin{tabular}{lp{2.5in}p{3in}}
{\bf  Term} & {\bf Definition}  & {\bf Examples} \\
\hline 
{\bf Cartesian product} & When $A$ and  $B$ are sets, \newline$A \times  B = \{ (a,b) \mid a \in A  \wedge b  \in B \}$ &$\{43, 9\} \times  \{9, A\}  = $
\newline $\mathbb{Z} \times \emptyset  = $  \\
&  &  \\
\hline
{\bf union} & When $A$ and  $B$ are sets, \newline$A \cup  B = \{ x \mid x \in A  \vee x \in B \}$ &$\{43, 9\} \cup \{9, A\}  = $
\newline $\mathbb{Z} \cup \emptyset  = $  \\
&  &  \\
\hline
{\bf intersection} & When $A$ and  $B$ are sets, \newline$A \cap  B = \{ x \mid x \in A  \wedge x \in B \}$ &$\{43, 9\} \cap \{9, A\}  = $
\newline $\mathbb{Z} \cap \emptyset  = $  \\
&  &  \\
\hline
{\bf set  difference} & When $A$ and  $B$ are sets, \newline$A -  B = \{ x \mid x \in A  \wedge x \notin B \}$ &$\{43, 9\} - \{9, A\}  = $
\newline $\mathbb{Z} - \emptyset  = $  \\
&  &  \\
\hline{\bf disjoint sets} & sets $A$ and  $B$ are disjoint \newline means $A \cap  B  = \emptyset$ &$\{43, 9\}, \{9, A\} $ are not  disjoint 
\newline $\mathbb{Z},  \emptyset $  are disjoint\\
&  &  \\
\hline
{\bf power set} & When $S$ is a   set, \newline $\mathcal{P}(S) = \{ X \mid X \subseteq S\}$ &
$\mathcal{P}(\{43, 9\}) = $
\newline $\mathcal{P}(\emptyset) = $  \\
\hline
\end{tabular}
\end{center}

\vspace{-10pt}

Let $W =  \mathcal{P}(  \{ 1,2,3,4,5\} ) = 
\underline{\phantom{\hspace{5in} }}$

\vspace{20pt}

{\bf Prove} or {\bf  disprove}:  $\forall  A \in W\,  \forall B \in W\,  \left( A \subseteq B
~\to ~ \mathcal{P}(A) \subseteq \mathcal{P}(B) \right)$

\vfill
\vfill
\vfill

{\it Extra example}: {\bf Prove} or {\bf  disprove}:  $\forall  A \in W\,  \forall B \in W\,  \left( \mathcal{P}(A)  =\mathcal{P}(B)
~\to ~ A = B \right)$

\vspace{20pt}

{\it Extra example}: {\bf Prove} or {\bf  disprove}:  $\forall  A \in W\,  \forall B \in W\, \forall C  \in W\,  \left( A\cup B  = A \cup  C
~\to ~ B = C \right)$

\vspace{20pt}

\fbox{\parbox{\linewidth}{%

{\bf New! Proof of conditional by direct proof}: To prove that the conditional statement $p \to q$ is true, we can assume $p$ is true and use that assumption to show $q$ is true.

{\bf New! Proof of Conditional by Contrapositive Proof}: To prove that the implication $p \to q$ is true, we can assume $q$ is false and use that assumption to show $p$ is also false. Sec 1.7 p83
}}


\section*{Review quiz questions}
\begin{enumerate}

\item {\bf Monday} Consider the predicate  $F(a,b)  = ``a \text{ is a factor of } b"$ over  the domain $\mathbb{Z}^{\neq 0} \times \mathbb{Z}$. Consider the following quantified
statements
\begin{multicols}{2}
\begin{enumerate}[label=(\roman*)]
\item $\forall x \in \mathbb{Z} ~(F(1,x))$
\item $\forall x \in \mathbb{Z}^{\neq 0} ~(F(x,1))$
\item $\exists x \in \mathbb{Z} ~(F(1,x))$
\item $\exists x \in \mathbb{Z}^{\neq 0} ~(F(x,1))$
\item $\forall x \in \mathbb{Z}^{\neq 0} ~\exists y \in \mathbb{Z} ~(F(x,y))$
\item $\exists x \in \mathbb{Z}^{\neq 0} ~\forall y \in \mathbb{Z} ~(F(x,y))$
\item $\forall y \in \mathbb{Z} ~\exists x \in \mathbb{Z}^{\neq 0} ~(F(x,y))$
\item $\exists y \in \mathbb{Z} ~\forall x \in \mathbb{Z}^{\neq 0} ~(F(x,y))$
\end{enumerate}
\end{multicols}
\begin{enumerate}
\item Select the statement whose translation is
\begin{quote}
``The number $1$ is  a factor of every integer."
\end{quote}
or write NONE if none of (i)-(viii) work.

\item Select the statement whose translation is
\begin{quote}
``Every integer has at least one nonzero factor."
\end{quote}
or write NONE if none of (i)-(viii) work.

\item Select the statement whose translation is
\begin{quote}
``There is an integer of which
all nonzero integers are a factor."
\end{quote}
or write NONE if none of (i)-(viii) work.

\item For each  statement (i)-(viii), determine
if  it is true or  false.
\end{enumerate}

\item {\bf Wednesday} Suppose $P(x)$ is  a predicate over a
domain $D$.
\begin{enumerate}
    \item True or False: To translate the statement
    ``There are at least two  elements in $D$
    where the predicate $P$ evaluates to true", we
    could  write
    \[
    \exists  x_1 \in D \, \exists x_2 \in D  \, (P(x_1) \wedge P(x_2))
    \]
    \item True or False: To translate the statement
    ``There are at most two  elements in $D$
    where the predicate $P$ evaluates to true", we
    could write
    \[
    \forall  x_1 \in D \, \forall x_2 \in D \, \forall x_3 \in  D \, \left(~ (~P(x_1) \wedge P(x_2)  \wedge P(x_3) ~) \to (~ x_1 = x_2 \vee x_2 = x_3 \vee x_1 = x_3~)~\right)
    \]

\end{enumerate}


\item {\bf Friday} Let $W = \mathcal{P}(\{1,2,3,4,5\})$.
Which of the following are true  about  $W$? 
(Select all and  only that  apply.)
\begin{multicols}{2}
\begin{enumerate}
    \item $\emptyset  \in W$
    \item $\emptyset \subseteq W$
    \item $\emptyset \subsetneq W$
    \item $\{1,2,3,4,5\} \in W$
    \item $\{1,2,3,4,5\} \subseteq W$
    \item $\{1,2,3,4,5\} \subsetneq W$
\end{enumerate}
\end{multicols}

\item {\bf Friday} Recall the Netflix example from class: Consider a four movie database. We denote the set of possible ratings 
$\{-1,0,1\} \times \{-1,0,1\} \times \{-1,0,1\} \times \{-1,0,1\}$ as $R_4$. We have the functions
\[
d_{1,4}(~ (x_1, x_2, x_3, x_4) , (y_1, y_2, y_3, y_4) ~) =  \sum_{i=1}^4\left( (\abs{x_i-y_i} + 1) \textbf{ div } 2 \right)
\]
\[
d_{2,4}(~ (x_1, x_2, x_3, x_4) , (y_1, y_2, y_3, y_4) ~) =  \sqrt{ \sum_{i=1}^4 (x_i - y_i)^2}
\]

{\bf Claim}: $\forall r_1 \in R_4 ~\forall r_2 \in R_4 ~(~r_1 = r_2 \to \neg \left( d_{1,4}(r_1,r_2) < d_{2,4} (r_1,r_2) \right)~)$

Select the correct choice to fill in the blanks in the following proof of the claim.

{\bf Proof}: Towards universal generalization, let $e_1$ be \underline{{\bf ~~~Blank (a)~~~}} element of $R_4$ and 
let $e_2$ be \underline{{\bf~~~Blank (a)~~~}} element of $R_4$.  We need to show that 
$e_1 = e_2 \to \neg ( \left( d_{1,4}(e_1,e_2) < d_{2,4} (e_1,e_2) \right)~)$.  Towards a \underline{\bf ~~~Blank (b)~~~}, we 
assume $e_1 = e_2$ and we need to prove that $\neg ( \left( d_{1,4}(e_1,e_2) < d_{2,4} (e_1,e_2) \right)~)$.
Calculating, $d_{1,4}(e_1,e_2) = d_{1,4}(e_1,e_1)$ by assumption that $e_1$ and $e_2$ are equal, and therefore (since 
each element in the summation in the definition of $d_{1,4}$ will be $0$ because it
is the quotient of $1$ upon division by $2$, since the absolute value of the difference of a number
with itself is $0$), 
$d_{1,4}(e_1,e_2) = 0$. Similarly, $d_{2,4} (e_1,e_2) = d_{2,4} (e_1,e_1)$ by assumption, and therefore
(since each term in the sum in the definition of $d_{2,4}$ will be $0$ because it is the square of the difference of a number
with itself), $d_{2,4} ( e_1,e_2) = 0$. Thus, $d_{1,4}(e_1,e_2) = d_{2,4}(e_1,e_2)$ so $d_{1,4}(e_1,e_2) < d_{2,4}(e_1,e_2)$ is False.  By definition of negation, $\lnot (d_{1,4}(e_1,e_2) < d_{2,4}(e_1,e_2) )$ is true, as required.
Thus the direct proof is complete and we have proved that the predicate being claimed to be universally 
true is true for an arbitrary element.  This means the universal generalization is also complete, and the proof is done. 
$\Box$

\begin{multicols}{2}
For Blank (a): 
\begin{enumerate}[label=\roman*.]
\item a counterexample
\item a witness
\item an arbitrary
\end{enumerate}

\columnbreak

For Blank (b)
\begin{enumerate}[label=\roman*.]
\item universal generalization
\item exhaustion
\item direct proof
\item proof by cases
\end{enumerate}
\end{multicols}

\item {\bf Friday} Let $W =  \mathcal{P}(\{1,2,3,4,5\})$.
The statement $$\forall A \in W~ \forall B\in W~ \forall  C  \in W~  ( A \cup B =  A \cup C ~\to~  B = C) $$ is false.
Which of the following  choices for  $A, B, C$ could  
be used to  give a counterexample to this claim?
(Select all and only that  apply.)
\begin{enumerate}
    \item $A = \{ 1, 2, 3 \}, B = \{ 1, 2\}, C= \{1, 3\}$
    \item $A = \{ \emptyset, 1, 2, 3 \}, B = \{ 1, 2\}, C= \{1, 3\}$
    \item $A = \{ 1, 2, 3 \}, B = \{ 1, 2\}, C= \{1, 4\}$
    \item $A = \{ 1, 2 \}, B = \{ 2, 3\}, C= \{1, 3\}$
    \item $A = \{ 1,2 \}, B =  \{ 1,3\}, C =  \{ 1,3\}$
\end{enumerate}

\item {\bf Friday} Let $W =  \mathcal{P}(\{1,2,3,4,5\})$.
Consider the  statement
$$\forall A \in W~ \forall B\in W~  \big( ( \mathcal{P}(A) = \mathcal{P}(B) )~\to~ (A = B) \big) $$

This statement is true. A proof of this statement starts with universal generalization, considering
arbitrary $A$ and $B$ in $W$. At this point, it remains to prove that $( \mathcal{P}(A) = \mathcal{P}(B) )~\to~ (A = B)$
is true about these arbitrary elements.  There are two ways to proceed: 

\begin{itemize}
\item[] First approach: By direct proof, in which we assume the hypothesis of the 
conditional and work to show that the conclusion follows.
\item[] Second approach: By proving the contrapositive version of the conditional instead, in which we
assume the negation of the conclusion and work to show that the negation of hypothesis follows.
\end{itemize} 
Pick an option from below for the assumption and ``need to show" in each approach.

\begin{enumerate}
    \item First approach, assumption.
    \item First approach, ``need to show".
    \item Second approach, assumption.
    \item Second approach, ``need to show".
\end{enumerate}


\begin{multicols}{2}
\begin{enumerate}[label=(\roman*)]
\item $\forall X ( X \subseteq A \leftrightarrow X \subseteq B)$
\item $\exists X ( X \subseteq A \leftrightarrow X \subseteq B)$
\item $\forall X ( X \subseteq A \oplus X \subseteq B)$
\item $\exists X ( X \subseteq A \oplus X \subseteq B)$
\item $\forall x ( x \in A \leftrightarrow x \in B)$
\item $\exists x ( x \in A \leftrightarrow x \in B)$
\item $\forall x ( x \in A \oplus x \in B)$
\item $\exists x ( x \in A \oplus x \in B)$
\end{enumerate}
\end{multicols}


\end{enumerate}
\end{document}
