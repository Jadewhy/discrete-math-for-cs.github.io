\documentclass[12pt, oneside]{article}

\usepackage{amssymb,amsmath,pifont,amsfonts,comment,enumerate,enumitem}
\usepackage{currfile,xstring,hyperref,tabularx,graphicx,wasysym}
\usepackage[labelformat=empty]{caption}
\usepackage[dvipsnames,table]{xcolor}
\usepackage{multicol,multirow,array,listings,tabularx,lastpage,textcomp,booktabs}

% NOTE(joe): This environment is credit @pnpo (https://tex.stackexchange.com/a/218450)
\lstnewenvironment{algorithm}[1][] %defines the algorithm listing environment
{   
    \lstset{ %this is the stype
        mathescape=true,
        frame=tB,
        numbers=left, 
        numberstyle=\tiny,
        basicstyle=\rmfamily\scriptsize, 
        keywordstyle=\color{black}\bfseries,
        keywords={,procedure, div, for, to, input, output, return, datatype, function, in, if, else, foreach, while, begin, end, }
        numbers=left,
        xleftmargin=.04\textwidth,
        #1
    }
}
{}
\lstnewenvironment{java}[1][]
{   
    \lstset{
        language=java,
        mathescape=true,
        frame=tB,
        numbers=left, 
        numberstyle=\tiny,
        basicstyle=\ttfamily\scriptsize, 
        keywordstyle=\color{black}\bfseries,
        keywords={, int, double, for, return, if, else, while, }
        numbers=left,
        xleftmargin=.04\textwidth,
        #1
    }
}
{}

\newcommand\abs[1]{\lvert~#1~\rvert}
\newcommand{\st}{\mid}

\newcommand{\A}[0]{\texttt{A}}
\newcommand{\C}[0]{\texttt{C}}
\newcommand{\G}[0]{\texttt{G}}
\newcommand{\U}[0]{\texttt{U}}

\newcommand{\cmark}{\ding{51}}
\newcommand{\xmark}{\ding{55}}


\usepackage{enumitem}

\begin{document}
\begin{flushright}
\StrBefore{\currfilename}{.}
\end{flushright}

\section*{This week's highlights}
\begin{itemize}
\item Practice with properties of recursively defined sets and functions
\item Define linked lists: a recursively defined data structure
\item Prove and disprove properties of recursively defined sets and functions with structural induction and/
or mathematical induction
\end{itemize}

\section*{Lecture videos}
Week 6 Day 1
\href{https://youtube.com/playlist?list=PLML4QilACLk5tqLcES_OqUwwxp3yqDlBk}{YouTube playlist}

Week 6 Day 2
\href{https://youtube.com/playlist?list=PLML4QilACLk4TgcwJFIyRw10sZ8JbVcF2}{YouTube playlist}

Week 6 Day 3
\href{https://youtube.com/playlist?list=PLML4QilACLk7GwJnGi4Sn0qqccvz1568Q}{YouTube playlist}

\newpage
\section*{Monday February 8}
{\bf Definition} The set of RNA strands $S$ is defined (recursively) by:
\[
\begin{array}{ll}
\textrm{Basis Step: } & \A \in S, \C \in S, \U \in S, \G \in S \\
\textrm{Recursive Step: } & \textrm{If } s \in S\textrm{ and }b \in B \textrm{, then }sb \in S
\end{array}
\]
where $sb$ is string concatenation.

The function \textit{rnalen} that computes the length of RNA strands in $S$ is defined recursively by
$rnalen: S  \to \mathbb{Z}^+$

\vspace{-20pt}

\begin{quote}
Basis step: If $b \in B$ then $rnalen(b)  = 1$

Recursive step: If $s \in S$ and $b \in B$, then $rnalen(sb)  = 1 + rnalen(s)$
\end{quote}

\vspace{-20pt}

The function \textit{basecount} that computes the number of a given base $b$ appearing in a RNA strand $s$ is defined recursively by $basecount : S \times B \to\mathbb{N}$

\vspace{-20pt}

\begin{quote}
Basis step: If  $b_1 \in B$, $b_2 \in B$, $basecount(b_1, b_2) =
        \begin{cases}
            1 & \textrm{when } b_1 = b_2 \\
            0 & \textrm{when } b_1 \neq b_2 \\
        \end{cases}$

Recursive Step: If $s \in S$, $b_1 \in B$, $b_2 \in B$, $basecount(s b_1, b_2) =
        \begin{cases}
            1 + \textit{basecount}(s, b_2) & \textrm{when } b_1 = b_2 \\
            \textit{basecount}(s, b_2) & \textrm{when } b_1 \neq b_2 \\
        \end{cases}$
\end{quote}

\vspace{-20pt}

Prove or disprove $\exists s \in S \, (\textit{rnalen(s)} = \textit{basecount}(s, \A))$:

\vfill

Prove or disprove $\forall s \in S \, (\textit{rnalen(s)} \geq \textit{basecount}(s, \A))$:

{\bf Proof}: 

{\bf Basis case}: Assume $s = \A \lor s = \C \lor s = \U \lor s = \G$. 
Need to show $\textit{rnalen(s)} \geq \textit{basecount}(s, \A)$.

{\it Case 1}: Want to show $(s = \A) \to (~\textit{rnalen(s)} \geq \textit{basecount}(s, \A)~)$.

\vfill

{\it Case 2}: Want to show $(s = \C \lor S = \U \lor S = \G) \to (~\textit{rnalen(s)} \geq \textit{basecount}(s, \A)~)$.

\vfill

\vfill

{\it Continued next page} 

\fbox{\parbox{\textwidth}{%
{\bf Proof by universal generalization}: To prove that $\forall x \, P(x)$
is true, we can take an arbitrary element $e$ from the domain and show that $P(e)$ is true, without making any assumptions about $e$ other than that it comes from the domain.

{\bf New! Proof by Structural Induction} (Rosen 5.3 p354) To prove a universal quantification over a recursively defined set:

\vspace{-10pt}


\begin{itemize}
\item[] {\bf Basis Step}:  Show the statement holds for elements specified in the basis step of the definition.
\item[]  {\bf Recursive Step}:  Show that if the statement is true for each of the elements used to construct
new elements in the recursive step of the definition, the result holds for these new elements.
\end{itemize}

\vspace{-10pt}

}}


\newpage


{\bf Recursive case}: Want to show
\[
\forall e \in S ~\left(~ \textit{rnalen(e)} \geq \textit{basecount}(e, \A) \to 
 \forall b \in B ~(~\textit{rnalen(eb)} \geq \textit{basecount}(eb, \A)~) ~\right)
 \]

Consider arbitrary $e$. Assume, as the {\bf induction hypothesis} that 
\[
\textit{rnalen(e)} \geq \textit{basecount}(e, \A)
\]

Need to show 
\[
 \forall b \in B ~(~\textit{rnalen(eb)} \geq \textit{basecount}(eb, \A)~) 
\]
Consider arbitrary $b \in B$. 


{\it Case 1}: Want to show $(b = \A) \to (~\textit{rnalen(eb)} \geq \textit{basecount}(eb, \A)~) $.

\vfill

{\it Case 2}: Want to show $(b = \C \lor b = \U \lor b = \G) \to (~\textit{rnalen(eb)} \geq \textit{basecount}(eb, \A)~) $.

\vfill
\vfill


\newpage
\section*{Wednesday February 10}


\fbox{\parbox{\textwidth}{%
{\bf Proof by Structural Induction} (Rosen 5.3 p354) To prove a universal quantification over a recursively defined set:

\vspace{-10pt}

\begin{itemize}
\item[] {\bf Basis Step}:  Show the statement holds for elements specified in the basis step of the definition.
\item[]  {\bf Recursive Step}:  Show that if the statement is true for each of the elements used to construct
new elements in the recursive step of the definition, the result holds for these new elements.
\end{itemize}

\vspace{-10pt}

}}

{\bf Definition} The set of natural numbers (aka nonnegative integers), $\mathbb{N}$, is defined (recursively) by:
\[
\begin{array}{ll}
\textrm{Basis Step: } & 0 \in \mathbb{N} \\
\textrm{Recursive Step: } & \textrm{If } n \in \mathbb{N} \textrm{ then } n+1 \in \mathbb{N} \textrm{ (where $n+1$ is integer addition)}
\end{array}
\]

The function $sumPow$ with domain $\mathbb{N}$, codomain
$\mathbb{N}$, and which computes, for input $i$, the sum of the first $i$ powers of $2$ is defined
recursively by $sumPow: \mathbb{N} \to \mathbb{N}$ with

\vspace{-20pt}

\begin{quote}
Basis step: $sumPow(0) = 1$.

Recursive step: If $x \in \mathbb{N}$ then $sumPow(x+1) = sumPow(x) + 2^{x+1}$.
\end{quote}

\vfill



Fill in the blanks in the following proof of $\forall n \in \mathbb{N} \, (sumPow(n) = 2^{n+1} - 1)$:

\vspace{-20pt}

\begin{quote}
Since $\mathbb{N}$ is recursively defined, we proceed by \underline{\phantom{structural induction \hspace{0.3in}}}.

{\bf Basis case} We need to show that \underline{\phantom{$sumPow(0) = 2^{0+1} - 1$ \hspace{0.2in}}}.
Evaluating each side: $LHS = sumPow(0) = 1$ by the basis case in the recursive definition
of $sumPow$; $RHS = 2^{0+1} - 1 = 2^1 - 1 = 2-1 = 1$. Since $1=1$, the equality holds.

{\bf Recursive step} Consider arbitrary natural number $n$ and assume, as the 
\underline{\phantom{Induction Hypothesis (IH)}} that $sumPow(n) = 2^{n+1} - 1$. We need to show that
\underline{\phantom{$sumPow(n+1) = 2^{(n+1) + 1} - 1$}}.  Evaluating each side: 
\[
LHS = sumPow(n+1) \overset{\text{rec def}}{=} sumPow(n)  + 2^{n+1}\overset{\text{IH}}{=} (2^{n+1} - 1) + 2^{n+1}.
\]
\[
RHS = 2^{(n+1)+1}- 1 \overset{\text{exponent rules}}{=} 2 \cdot 2^{n+1} -1  = \left(2^{n+1} + 2^{n+1} \right) - 1
\overset{\text{regrouping}}{=}  (2^{n+1} - 1) + 2^{n+1} 
\]
Thus, $LHS = RHS$. The structural induction is complete and we have proved the universal generalization.
\end{quote}



\vfill
{\it Extra example} Connect the function $sumPow$ to binary expansions of positive integers.

\newpage
{\bf Definition} The set of linked lists of natural numbers $L$ is defined:
\[
\begin{array}{ll}
    \textrm{Basis Step: } & [] \in L \\
    \textrm{Recursive Step: } & \textrm{If } l \in L\textrm{ and }n \in \mathbb{N} \textrm{, then } (n, l) \in L
\end{array}
\]

Examples: 

\vfill

{\bf Definition} The length of a linked list of natural numbers $L$, $\textit{length}: L \to \mathbb{N}$ is defined by:
\[
\begin{array}{llll}
\textrm{Basis Step:} &  & length([]) &= 0 \\
\textrm{Recursive Step:} & \textrm{If } l \in L\textrm{ and }n \in \mathbb{N}\textrm{, then  } & length((n, l))  &\phantom{= 1+ length(l)}
\end{array}
\]

Examples: 

\vfill


{\it Extra example:} The function $prepend : L \times \mathbb{N} \to L$ that adds an element at the front of a linked list is defined:
\[
\phantom{prepend(l, n) = (n, l)}
\]


\vfill

{\bf Definition} The function $\textit{append} : L \times \mathbb{N} \to L$ that adds an element at the end of a linked list is defined:
\[
\begin{array}{llll}
\textrm{Basis Step:} & \textrm{If } m \in \mathbb{N}\textrm{ then } & \phantom{\textit{append}([], m)} & \phantom{= (m, []) }\\
\textrm{Recursive Step:} & \textrm{If } l \in L\textrm{ and }n \in \mathbb{N}\textrm{ and }m \in \mathbb{N}\textrm{, then  } & \phantom{\textit{append}((n, l), m) } &\phantom{= (n, \textit{append}(l, m))}
\end{array}
\]

Examples: 

\vfill 

\newpage
{\bf Claim}: $\forall l \in L \, (~length(~append(l, 100)~) > length(l)~)$

{\bf Proof:} By structural induction on $L$, we have two cases:

\vspace{-10pt}

{\bf Basis Step}

\vspace{-10pt}

    \begin{tabular}{l p{3.5in}}
    1. \begin{tabular}{l}
        \textbf{To Show} $length(~append([], 100)~) > length([])$ \\
    \end{tabular}
    &  Because $[]$ is the only element defined in the basis step of $L$, we only need to 
prove that the property holds for $[]$.\\
    \\
    2. \begin{tabular}{l}
        \textbf{To Show} $length(~(100,[])~) > length([])$ \\
    \end{tabular}
    &  By basis step in definition of $append$.\\
    \\
    3. \begin{tabular}{l}
        \textbf{To Show} $(1 +length([])) > length([])$ \\
    \end{tabular}
    &  By recursive step in definition of $length$.\\
    \\    
    4. \begin{tabular}{l}
        \textbf{To Show} $1+0 > 0$ \\
    \end{tabular}
    &  By basis step in definition of $length$.\\
    \\       
    5. \begin{tabular}{l}
        \textbf{To Show } $T$ \\
    \end{tabular}
     & By properties of integers \\
    QED & Because we got to $T$ only by rewriting \textbf{To Show} to equivalent statements, using well-defined proof techniques, and applying definitions. \\
    \end{tabular}

\vspace{-10pt}

{\bf Recursive Step}

\vspace{-10pt}

Consider an arbitrary: $l = (n, l')$, $l' \in L$, $n \in \mathbb{N}$, and we  assume
as the {\bf induction hypothesis} that:
\[
length(~append(l', 100)~) > length(l')
\]
Our goal is to show that $length(~append( (n,l'), 100)~) > length((n,l'))$ is also true. We evaluate each side of the 
candidate inequality:
\begin{align*}
LHS &= length(~append( (n,l'), 100)~) = length(~(n, append(l', 100) )~) \qquad \text{by the recursive definition of $append$}\\
&= 1 + length(~ append(l', 100) ~) \qquad \text{by the recursive definition of $length$}\\
&> 1+ length(l')  \qquad \text{by the induction hypothesis}\\
&= length( (n,l') )  \qquad \text{by the recursive definition of $length$}\\
&= RHS 
\end{align*}

\newpage
\section*{Friday February 12}


{\bf Invariant}: A property that is true about our algorithm no matter what. \hfill Rosen p375

{\bf Theorem}: Statement that can be shown to be true, usually an important one. \hfill Rosen p81

\begin{quote}
 Less important theorems can be called {\bf proposition}, {\bf fact}, {\bf result}.

 A less important theorem that is useful in proving a theorem is called a {\bf lemma}.
 
 A theorem that can be proved directly after another one has been proved is called a {\bf corollary}
\end{quote}







{\bf Theorem}: A robot on an infinite 2-dimensional integer grid starts at $(0,0)$ and at each step moves
to diagonally adjacent grid point. This robot can / cannot {\footnotesize({\it circle one})} reach $(1,0)$.


{\bf Definition} The set of positions the robot can visit  $P$ is defined by:
\[
\begin{array}{ll}
    \textrm{Basis Step: } & (0,0) \in P \\
    \textrm{Recursive Step: } & \textrm{If } (x,y) \in P  \textrm{, then } 
    \phantom{(x+1, y+1), (x+1, y-1), (x-1, y-1), (x-1, y+1)} \textrm{ are also in } P
\end{array}
\]

{\bf Lemma}: $\forall (x,y) \in P( (x+y \textrm{ is an even integer})~)$


Proof of theorem using lemma: To show is $(1,0) \notin P$. Rewriting the lemma to explicitly 
restrict the domain of the universal, 
we have $\forall (x,y) ~(~ (x,y) \in P  \to (x+y \textrm{ is an even integer})~)$.  Since
the universal is true, 
$ (~ (1,0) \in P \to (1+0 \textrm{ is an even integer})~)$ is a true statement.
Evaluating the conclusion of this conditional statement: 
By definition of long division, since $1 = 0 \cdot 2 + 1$ (where $0 \in \mathbb{Z}$ and 
$1 \in \mathbb{Z}$ and $0 \leq 1 < 2$ mean that $0$ is the quotient and $1$ is the remainder), $1 ~\textrm{\bf mod}~ 2 = 1$ which is not $0$ 
so the conclusion is false.  A true conditional with a false conclusion must have a false hypothesis.
Thus, $(1,0) \notin P$, QED. $\square$



Proof of lemma by structural induction:

{\bf Basis Step}


{\bf Recursive Step}.  Consider arbitrary $(x,y) \in P$.  To show is:
\[
(x+y \text{ is an even integer}) \to (\text{sum of coordinates of next position is even integer})
\]
Assume {\bf as the induction hypothesis, IH} that: 
\vfill


\newpage

\fbox{\parbox{\textwidth}{%

{\bf ``New"! Proof by Mathematical Induction} (Rosen 5.1 p329)

To prove a universal quantification over the set of  all integers greater than  or  equals some  base integer $b$:

\vspace{-10pt}

\begin{itemize}
\item[] {\bf Basis Step}:  Show the statement holds for $b$. 
\item[]  {\bf Recursive Step}:  Consider an arbitrary integer $n$ greater than or  equal to  $b$, assume
    (as the {\bf induction hypothesis})  that the property holds  for $n$, and use  this and
    other facts to  prove that  the property holds for $n+1$.
\end{itemize}

\vspace{-10pt}

}}


\vspace{-20pt}


\begin{center}
\begin{tabular}{|p{3.5in}|p{3.5in}|}
\hline
Recall that the set of linked lists of natural numbers $L$

\vspace{-10pt}

\begin{itemize}
\item[] Basis Step: $[] \in L$

\vspace{-10pt}

\item[] Recursive Step: If $l \in L$ and $n \in \mathbb{N}$ then $(n, l) \in L$
\end{itemize}

\vspace{-20pt}

&
Recall that length of a linked list of natural numbers $L$, $\textit{length}: L \to \mathbb{N}$ is defined by:

\vspace{-10pt}

\begin{itemize}
\item[] Basis step: $length([]) = 0 $

\vspace{-10pt}

\item[] Recursive step: If $l \in L$ and $n \in \mathbb{N}$ then $length((n, l)) = 1+ length(l)$
\end{itemize}

\vspace{-20pt}

\\
\hline
\end{tabular}
\end{center}

\vspace{-10pt}

Prove or disprove: $\forall n \in \mathbb{N} ~\exists l \in L ~(~length(l) = n~)$


\vfill
\vfill
\vfill


\newpage
\section*{Review quiz questions}
\begin{enumerate}
\item {\bf Monday} The function \textit{rnalen} that computes the length of RNA strands in $S$ is defined recursively by
$rnalen: S  \to \mathbb{Z}^+$


\begin{quote}
Basis step: If $b \in B$ then $rnalen(b)  = 1$

Recursive step: If $s \in S$ and $b \in B$, then $rnalen(sb)  = 1 + rnalen(s)$
\end{quote}


The function \textit{basecount} that computes the number of a given base $b$ appearing in a RNA strand $s$ is defined recursively by $basecount : S \times B \to\mathbb{N}$

\begin{quote}
Basis step: If  $b_1 \in B$, $b_2 \in B$, $basecount(b_1, b_2) =
        \begin{cases}
            1 & \textrm{when } b_1 = b_2 \\
            0 & \textrm{when } b_1 \neq b_2 \\
        \end{cases}$

Recursive Step: If $s \in S$, $b_1 \in B$, $b_2 \in B$, $basecount(s b_1, b_2) =
        \begin{cases}
            1 + \textit{basecount}(s, b_2) & \textrm{when } b_1 = b_2 \\
            \textit{basecount}(s, b_2) & \textrm{when } b_1 \neq b_2 \\
        \end{cases}$
\end{quote}

\begin{enumerate}
\item Select all and only options that give a witness for the existential quantification
$$\exists s \in S ~(~rnalen(s) = basecount(s,\U)~)$$
\begin{enumerate}
\item $\A$
\item $\U\U$
\item $\C\U$
\item $(\U, 1)$
\end{enumerate}

\item Select all and only options that give a counterexample for the universal quantification
$$\forall s \in S ~(~rnalen(s) > basecount(s,\G)~)$$
\begin{enumerate}
\item $\U$
\item $\G\G$
\item $\A\G$
\item $\C\U\G$
\end{enumerate}

\item Select all and only the true statements
\begin{enumerate}
\item $\forall s \in S ~\exists b \in B ~\left(~rnalen(s) = basecount(s,b)~ \right)$
\item $\exists s \in S ~\forall b \in B ~\left(~rnalen(s) = basecount(s,b)~ \right)$
\item \begin{align*} \forall s_1 \in S~\forall s_2 \in S ~&\forall b \in B ~\big( ~\big( rnalen(s_1) = basecount(s_1,b) \\
&\land rnalen(s_2) = basecount(s_2,b) \land rnalen(s_1) = rnalen(s_2) \big) \to s_1 = s_2  \big)\end{align*}
\end{enumerate}

\end{enumerate}

\item {\bf Wednesday}  The function $sumPow$ with domain $\mathbb{N}$, codomain
$\mathbb{N}$, and which computes, for input $i$, the sum of the first $i$ powers of $2$ is defined
recursively by $sumPow: \mathbb{N} \to \mathbb{N}$ with

\begin{quote}
Basis step: $sumPow(0) = 1$.

Recursive step: If $x \in \mathbb{N}$ then $sumPow(x+1) = sumPow(x) + 2^{x+1}$.
\end{quote}

\begin{enumerate}
\item Calculate $sumPow(0)$
\item Calculate $sumPow(1)$
\item Calculate $sumPow(2)$
\end{enumerate}

\item  {\bf Wednesday} {\bf Definition} The set of linked lists of natural numbers $L$ is defined by:
\[
\begin{array}{ll}
    \textrm{Basis Step: } & [] \in L \\
    \textrm{Recursive Step: } & \textrm{If } l \in L\textrm{ and }n \in \mathbb{N} \textrm{, then } (n, l) \in L
\end{array}
\]

{\bf Definition} The function $\textit{length} : L \to \mathbb{N}$ that computes the length of a list is:
\[
\begin{array}{llll}
& & \textit{length} : L & \to \mathbb{N} \\
\textrm{Basis Step:} &  & \textit{length}([]) & = 0 \\
\textrm{Recursive Step:} & \textrm{If } l \in L\textrm{ and }n \in \mathbb{N}\textrm{, then  } & \textit{length}((n, l)) & = 1 + \textit{length}(l)
\end{array}
\]


Consider this (incomplete) definition:

{\bf Definition} The function $\textit{increment} : \underline{\hspace{6em}}$ that adds 1 to each element of a linked list is defined by:
\[
\begin{array}{llll}
& & \textit{increment} : \underline{\hspace{3em}} & \to \underline{\hspace{3em}} \\
\textrm{Basis Step:} & & \textit{increment}([]) & = [] \\
\textrm{Recursive Step:} & \textrm{If } l \in L, n \in \mathbb{N} & \textit{increment}((n, l)) & = (1 + n, \textit{increment}(l))
\end{array}
\]

Consider this (incomplete) definition:

{\bf Definition} The function $\textit{sum} : L \to \mathbb{N}$ that adds together all the elements of the list is defined by:
\[
\begin{array}{llll}
& & \textit{sum} : L & \to \mathbb{N} \\
\textrm{Basis Step:} & & \textit{sum}([]) & = 0 \\
\textrm{Recursive Step:} & \textrm{If } l \in L, n \in \mathbb{N} & \textit{sum}((n, l)) & = \underline{\hspace{8em}}
\end{array}
\]


\begin{enumerate}
    \item (You will compute a sample function application and then fill in the blanks for the domain and codomain) Based on the definition, what is the result of $\textit{increment}((4, (2, (7, []))))$? Write your answer directly with no spaces.
    
    \item Which of the following describes the domain and codomain of \textit{increment}?
    
    \begin{multicols}{2}
    \begin{enumerate}
        \item $L \to \mathbb{N}$
        \item $L \to \mathbb{N} \times L$
        \item $L \times \mathbb{N} \to L$
        \item $L \times \mathbb{N} \to \mathbb{N}$
        \item $L \to L$
        \item None of the above
    \end{enumerate}
    \end{multicols}
    
    \item Assuming we would like $sum((5, (6, [])))$ to evaluate to $11$ and $sum((3, (1, (8, []))))$ to evaluate to $12$, which of the following could be used to fill in the definition of the recursive case of \textit{sum}?
    
     \begin{multicols}{2}
    \begin{enumerate}
        \item $\begin{cases}
            1 + \textit{sum}(l) & \textrm{when } n \neq 0 \\
            \textit{sum}(l) & \textrm{when } n = 0 \\
        \end{cases}$
        \item $1 + \textit{sum}(l)$
        \item $n + \textit{increment}(l)$
        \item $n + \textit{sum}(l)$
        \item None of the above
    \end{enumerate}
    \end{multicols}
    
    \item Choose only and all of the following statements that are \textbf{well-defined}; that is, they correctly reflect the domains and codomains of the functions and quantifiers, and respect the notational conventions we use in this class. Note that a well-defined statement may be true or false.

    \begin{multicols}{2}    
    \begin{enumerate}
        \item $\forall l \in L \, (\textit{sum}(l))$
        \item $\exists l \in L \, (\textit{sum}(l) \land \textit{length}(l))$
        \item $\forall l \in L \, (\textit{sum}(\textit{increment}(l)) = 10)$
        \item $\exists l \in L \, (\textit{sum}(\textit{increment}(l)) = 10)$
        \item $\forall l \in L \, \forall n \in \mathbb{N} \, ((n \times l) \subseteq L)$
        \item $\forall l_1 \in L \, \exists l_2 \in L \, (\textit{increment}(\textit{sum}(l_1)) = l_2)$
        \item $\forall l \in L \, (\textit{length}(\textit{increment}(l)) = \textit{length}(l))$
    \end{enumerate}
    \end{multicols}
    
    \item Choose only and all of the statements in the previous part that are both well-defined and true.
\end{enumerate}

\item {\bf Friday} Recall the set $P$ defined by the recursive definition
\[
\begin{array}{ll}
    \textrm{Basis Step: } & (0,0) \in P\\
     \textrm{Recursive Step: } & \textrm{If } (x,y) \in P \textrm{ then } 
     (x+1, y+1) \in P \textrm{ and } (x+1, y-1) \in P \textrm{ and }\\ 
     & (x-1,y-1) \in P 
     \textrm{ and } (x-1, y+1) \in P
\end{array}
\]
\begin{enumerate}
\item Select all and only the ordered pairs below that are elements of $P$
\begin{enumerate}
\item $(0,0)$
\item $(4,0)$
\item $(1,1)$
\item $(1.5,2.5)$
\item $(0, -2)$
\end{enumerate}
\item What is another description of the set $P$ ? (Select all and only the true descriptions.)
\begin{enumerate}
\item $\mathbb{Z} \times \mathbb{Z}$
\item $\{ (n,n) ~|~ n \in \mathbb{Z} \}$
\item $\{ (a,b) \in \mathbb{Z} \times \mathbb{Z} ~|~ (a+b) \textbf{ mod } 2 =0 \}$
\end{enumerate}
\end{enumerate}


\end{enumerate}
\end{document}
