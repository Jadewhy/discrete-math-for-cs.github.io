\documentclass[12pt, oneside]{article}

\usepackage{amssymb,amsmath,pifont,amsfonts,comment,enumerate,enumitem}
\usepackage{currfile,xstring,hyperref,tabularx,graphicx,wasysym}
\usepackage[labelformat=empty]{caption}
\usepackage[dvipsnames,table]{xcolor}
\usepackage{multicol,multirow,array,listings,tabularx,lastpage,textcomp,booktabs}

% NOTE(joe): This environment is credit @pnpo (https://tex.stackexchange.com/a/218450)
\lstnewenvironment{algorithm}[1][] %defines the algorithm listing environment
{   
    \lstset{ %this is the stype
        mathescape=true,
        frame=tB,
        numbers=left, 
        numberstyle=\tiny,
        basicstyle=\rmfamily\scriptsize, 
        keywordstyle=\color{black}\bfseries,
        keywords={,procedure, div, for, to, input, output, return, datatype, function, in, if, else, foreach, while, begin, end, }
        numbers=left,
        xleftmargin=.04\textwidth,
        #1
    }
}
{}
\lstnewenvironment{java}[1][]
{   
    \lstset{
        language=java,
        mathescape=true,
        frame=tB,
        numbers=left, 
        numberstyle=\tiny,
        basicstyle=\ttfamily\scriptsize, 
        keywordstyle=\color{black}\bfseries,
        keywords={, int, double, for, return, if, else, while, }
        numbers=left,
        xleftmargin=.04\textwidth,
        #1
    }
}
{}

\newcommand\abs[1]{\lvert~#1~\rvert}
\newcommand{\st}{\mid}

\newcommand{\A}[0]{\texttt{A}}
\newcommand{\C}[0]{\texttt{C}}
\newcommand{\G}[0]{\texttt{G}}
\newcommand{\U}[0]{\texttt{U}}

\newcommand{\cmark}{\ding{51}}
\newcommand{\xmark}{\ding{55}}


\usepackage{enumitem}

\begin{document}
\begin{flushright}
\StrBefore{\currfilename}{.}
\end{flushright}

\section*{This week's highlights}
\begin{itemize}
\item Define (binary) relations and give examples.
\item Define equivalence using relations and give examples.
\item Use the equivalence relation of congruence modulo integers and apply its properties
\item Trace the algorithms involved in Diffie-Helman key exchange
\item Trace the algorithms involved in modular exponentiation
\item Determine and prove whether a given binary relation is
\begin{itemize}
\item symmetric
\item antisymmetric
\item reflexive
\item transitive
\end{itemize}
\item Determine and prove whether a given binary relation is an equivalence relation
\item Determine and prove whether a given binary relation is a partial order
\item Draw the Hasse diagram of a partial order
\end{itemize}

\section*{Lecture videos}
Week 9 Day 1
\href{https://youtube.com/playlist?list=PLML4QilACLk4qF_srbNXwhV_CbLjv_cg8}{YouTube playlist}

Week 9 Day 2
\href{https://youtube.com/playlist?list=PLML4QilACLk4DKz8D5EjwZ5QWXpBG8R17}{YouTube playlist}

Week 9 Day 3
\href{https://youtube.com/playlist?list=PLML4QilACLk4ef0s6DQNGUiZpgLSNZe7c}{YouTube playlist}

\newpage
\section*{Monday March 1}


{\bf Definition}: When $A$ and $B$ are sets, we say any subset of $A \times B$ is a {\bf binary relation}. There 
are other ways to represent a relation $R$

\vspace{-20pt}

\begin{itemize}
\item A function $f_{TF} : A \times B \to \{T, F\}$
with $f_{TF}( ~~~~~~ ) = \underline{\phantom{\hspace{1in}}}$
\item A function $f_{\mathcal{P}} : A   \to \mathcal{P}(B)$ with $f_{\mathcal{P}}( ~~~~~~ ) = \underline{\phantom{\hspace{1in}}}$
\end{itemize}


{\bf Definition}: When $A$ is a set, we say any subset of $A \times A$ is a (binary) {\bf relation} on $A$.




{\it Example}: For $A = \mathcal{P}(\mathbb{R})$, we can define the relation $EQ_{\mathbb{R}}$ on $A$ as 
\[
\{ (X_1, X_2 ) \in\mathcal{P}(\mathbb{R})  \times \mathcal{P}(\mathbb{R}) ~|~ |X_1| = |X_2| \}
\]


\vfill

{\it Example}: Let $R_{(\textbf{mod } n)}$ be the set of all pairs of integers $(a, b)$ such that $(a \textbf{ mod } n = b \textbf{ mod } n)$.
Then $a$ is {\bf congruent to} $b$ \textbf{mod} $n$ means $(a, b) \in R_{(\textbf{mod } n)}$. A common notation is to write this as $a \equiv b (\textbf{mod } n)$.


$R_{(\textbf{mod } n)}$ is a relation on the set $\underline{\hspace{25em}}$


Some example elements of $R_{(\textbf{mod } 4)}$ are: \underline{\hspace{25em}}

\vfill
\newpage

{\it Example}: Recall that $S$ is defined as the set of all RNA strands, strings made of the bases in 
 $B = \{\A,\U,\G,\C\}$. Define the functions \textit{mutation}, \textit{insertion}, and \textit{deletion} as described by the pseudocode below:

\begin{algorithm}
procedure $\textit{mutation}$($b_1\cdots b_n$: $\textrm{a RNA strand}$, $k$: $\textrm{a  positive integer}$, $b$: $\textrm{an  element of } B$)
for $i$ := $1$ to $n$
  if $i$ = $k$
    $c_i$ := $b$
  else
    $c_i$ := $b_i$
return $c_1\cdots c_n$ $\{ \textrm{The return value is a RNA strand made of the } c_i \textrm{ values}\}$
\end{algorithm}

\begin{algorithm}
procedure $\textit{insertion}$($b_1\cdots b_n$: $\textrm{a RNA strand}$, $k$: $\textrm{a  positive integer}$, $b$: $\textrm{an  element of } B$)
if $k > n$
  for $i$ := $1$ to $n$
    $c_i$ := $b_i$
  $c_{n+1}$ := $b$
else 
  for $i$ := $1$ to $k-1$
    $c_i$ := $b_i$
  $c_k$ := $b$
  for $i$ := $k+1$ to $n+1$
    $c_i$ := $b_{i-1}$
return $c_1\cdots c_{n+1}$ $\{ \textrm{The return value is a RNA strand made of the } c_i \textrm{ values}\}$
\end{algorithm}

\begin{algorithm}
procedure $\textit{deletion}$($b_1\cdots b_n$: $\textrm{a RNA strand}$, $k$: $\textrm{a  positive integer}$)
if $k > n$
  $m$ := $n$
  for $i$ := $1$ to $n$
    $c_i$ := $b_i$
else
  $m$ := $n-1$
  for $i$ := $1$ to $k-1$ 
    $c_i$ := $b_i$
  for $i$ := $k$ to $n-1$
    $c_i$ := $b_{i+1}$
return $c_1\cdots c_m$ $\{ \textrm{The return value is a RNA strand made of the } c_i \textrm{ values}\}$
\end{algorithm}


$Mut$ with domain $S \times S$ is defined by, for $s_1 \in S$ and $s_2 \in S$,
\[
Mut(s_1,s_2) = \exists k\in \mathbb{Z^+} \exists b \in B (~ mutation(s_1, k, b) = s_2~)
\]
$Ins$ with domain $S \times S$ is defined by, for $s_1 \in S$ and $s_2 \in S$,
\[
Ins(s_1,s_2) = \exists k\in \mathbb{Z^+} \exists b \in B (~ insertion(s_1, k, b) = s_2~)
\]
$Del$ with domain $S \times S$ is defined by, for $s_1 \in S$ and $s_2 \in S$,
\[
Del(s_1,s_2) = \exists k\in \mathbb{Z^+} (~ deletion(s_1, k) = s_2~)
\]

{\bf Definition}: We say that a RNA strand $s_1$ is ``within one edit'' of a RNA strand $s_2$ to mean
\[
Mut(s_1,s_2) \lor Mut(s_2,s_1) \lor Ins(s_1,s_2) \lor Ins(s_2, s_1) \lor Del(s_1, s_2) \lor Del(s_2,s_1)
\]

\[
\begin{array}{ll}
    \begin{array}{lll}
    & \textit{within1}_{TF} : \underline{\phantom{S \times S}} & \to \underline{\phantom{\{T, F\}}} \\
    \\
    & \textit{within1}_{TF}(s_1, s_2) & = \underline{\phantom{\textrm{``CHECK if within 1 edit''}}} \\
    \end{array}
&


    \begin{array}{lll}
    & \textit{within1}_{\mathcal{P}} : \underline{\phantom{S \hspace{1em}}} & \to \underline{\phantom{\mathcal{P}(S)}} \\
    \\
    & \textit{within1}_{\mathcal{P}}(s_1) & = \underline{\phantom{\textrm{``COMPUTE all within 1 edit''}}} \\
    \end{array}
\end{array}
\]
\[
W_1 = \{ \underline{\phantom{(s_1, s_2) \in S \times S ~\mid~ s_1, s_2 \textrm{ are within 1 edit}}}\}
\]

\vfill
\newpage

A relation $R$ on a set $A$ is called {\bf reflexive} means $(a, a) \in R$ for every element $a \in A$. 


\vfill

A relation $R$ on a set $A$ is called {\bf symmetric} means $(b, a) \in R$ whenever $(a, b) \in R$, for all $a, b \in A$. 

{\it Example}: when the domain is $\{ a,b,c,d,e,f,g,h\}$ consider the relation $\{ (a,b), (b,a), (b,c), (c,b),(f,g), (g,f) \}$.

\vfill




A relation $R$ on a set $A$ is called {\bf transitive} means whenever $(a, b) \in R$ and $(b, c) \in R$, then $(a, c) \in R$, for all $a, b, c \in A$.

{\it Example}: when the domain is $\{ a,b,c,d,e,f,g,h\}$ consider the relation 
$$\{ (a,b), (b,a), (b,c), (c,b), (a,a), (b,b), (c,c), (e,g), (f,g), (e,f) \}$$

\vfill


\begin{tabular}{|l|l|l|l|}
    \hline
    Relation & Reflexive? (why/why not) & Symmetric? (why/why not) & Transitive? (why/why not) \\
    \hline
    &&&\\
    $W_1$ & & &  \\
    &&&\\ &&&\\
    &&&\\ &&&\\
    &&&\\ &&&\\
    &&&\\ &&&\\
    \hline
    &&&\\ 
    $R_{(\textbf{mod } 4)}$ & & & \\
    &&&\\ &&&\\
    &&&\\ &&&\\
    &&&\\ &&&\\
    &&&\\ &&&\\
    \hline
\end{tabular}

\newpage
\section*{Wednesday March 3}

{\bf Definition}: ({\it Rosen 9.1}) A relation $R$ on a set $A$ is called {\bf reflexive} means $(a, a) \in R$ for every element $a \in A$. A relation $R$ on a set $A$ is called {\bf symmetric} means $(b, a) \in R$ whenever $(a, b) \in R$, for all $a, b \in A$. A relation $R$ on a set $A$ is called {\bf transitive} means whenever $(a, b) \in R$ and $(b, c) \in R$, then $(a, c) \in R$, for all $a, b, c \in A$.


{\bf Definition}: ({\it Rosen 9.5}) A relation is an {\bf equivalence relation} means it is reflexive, symmetric, and transitive.

{\bf Definition}: ({\it Rosen 9.5}) An {\bf equivalence class} of an element $a \in A$ for an equivalence relation $R$ on the set $A$ is the set $\{s \in A | (a, s) \in R \}$. We write this as $[a]_R$.


$$[5]_{R_{(\textbf{mod } 4)}} = \{ s \in \mathbb{Z} ~|~ (5,s) \in R_{{(\textbf{mod } 4)}} \}$$

Some examples of elements of $[5]_{R_{(\textbf{mod } 4)}}$ are: \underline{\hspace{20em}}

Some examples of elements of $[9]_{R_{(\textbf{mod } 4)}}$ are: \underline{\hspace{20em}}

Some examples of elements of $[6]_{R_{(\textbf{mod } 4)}}$ are: \underline{\hspace{20em}}

\vfill

{\bf Definition}: A {\bf partition} of a set $A$ is a set of non-empty, disjoint subsets $A_1, A_2, \cdots, A_n$ such that $A_1 \cup A_2 \cup \cdots \cup A_n = A$.

We can partition the set of integers using equivalence classes of  $R_{(\textbf{mod } 4)}$
\[
\mathbb{Z} =  [0]_{R_{(\textbf{mod } 4)}}~ \cup ~[1]_{R_{(\textbf{mod } 4)}} ~\cup~[2]_{R_{(\textbf{mod } 4)}}~\cup~
[3]_{R_{(\textbf{mod } 4)}}
\]




{\it Recall}: We say $a$ is {\bf congruent to} $b$ \textbf{mod} $n$ means $(a, b) \in R_{(\textbf{mod } n)}$. A common notation is to write this as $a \equiv b (\textbf{mod } n)$.


{\bf Modular arithmetic}: 

$(102 + 48) \textbf{ mod } 10 = \underline{\phantom{\hspace{3in}}} $ 

$(7 \cdot 10) \textbf{ mod } 5 = \underline{\phantom{\hspace{3.3in}}} $ 

$(2^5) \textbf{ mod } 3 =  \underline{\phantom{\hspace{3.45in}}} $ 

\vfill

{\bf Lemma} (Section 4.1 Theorem 5): For $a, b \in \mathbb{Z}$ 
and positive integer $n$, if $a \equiv b (\textbf{mod } n)$ and $c \equiv d (\textbf{mod } n)$ 
then $a+c \equiv b+d (\textbf{mod } n)$ and $ac \equiv bd (\textbf{mod } n)$.
{\bf Informally}: can bring mod ``inside" and do it first, for addition and for multiplication.

\newpage



{\bf Lemma} (Section 4.1, page 241): For $a, b \in \mathbb{Z}$ 
and positive integer $n$, $(a,b) \in R_{(\textbf{mod } n)}$ if and only if  $n | a-b$.

\vfill
\vfill
\vfill
\vfill

{\bf Application: Cycling}

How many minutes past the hour are we at?  \hfill {\it Model with} $+15 \textbf{ mod } 60$

\begin{tabular}{lccccccccccc}
{\bf Time:} &12:00pm  &12:15pm&12:30pm  &12:45pm&1:00pm  &1:15pm&1:30pm  &1:45pm&2:00pm \\
{\bf ``Minutes past":} &$0$ & $15$ & $30$ & $45$ &$0$ & $15$ & $30$ & $45$ &$0$\\
\end{tabular}

Replace each English letter by a letter that's fifteen ahead of it in the alphabet
  \hfill {\it Model with} $+15 \textbf{ mod } 26$

{\tiny
\begin{tabular}{lcccccccccccccccccccccccccc}
{\bf Original index:} & $0$ & $1$
 & $2$ & $3$ &  $4$ & $5$ &  $6$ & $7$ &  $8$ & $9$ & $10$ & $11$ & $12$ & $13$ & $14$ & $15$ & 
  $16$ & $17$ &  $18$ & $19$ &  $20$ & $21$ &  $22$ & $23$ & $24$ & $25$\\
{\bf Original letter:} & A & B& C & D & E & F& G& H & I & J & K & L &M & N& O &P &Q & R & S & T & U & V & W & X & Y & Z \\
{\bf Shifted letter}: &P &Q & R & S & T & U & V & W & X & Y & Z & A & B& C & D & E & F& G& H & I & J & K & L &M & N& O \\
{\bf Shifted index:} &$15$ & 
  $16$ & $17$ &  $18$ & $19$ &  $20$ & $21$ &  $22$ & $23$ & $24$ & $25$ & $0$ & $1$
 & $2$ & $3$ &  $4$ & $5$ &  $6$ & $7$ &  $8$ & $9$ & $10$ & $11$ & $12$ & $13$ & $14$ 
\end{tabular}
}

\newpage
{\bf Application: Cryptography}

{\bf Definition}: Let $a$ be a positive integer and $p$ be a large\footnote{We leave the definition of ``large'' vague here, but think hundreds of digits for practical applications. In practice, we also need a particular relationship between $a$ and $p$ to hold, which we leave out here. See more in Rosen, 4.6, p302.} prime number, both known to everyone. Let $k_1$ be a secret large number known only to person $P_1$ (Alice) and $k_2$ be a secret large number known only to person $P_2$ (Bob). Let the {\bf Diffie-Helman shared key} for $a, p, k_1, k_2$ be $(a^{k_1\cdot k_2} \textbf{ mod } p)$.

{\bf Idea}: $P_1$ can quickly compute the Diffie-Helman shared key knowing only $a, p, k_1$ and the result of $a^{k_2} \textbf{ mod } p$ (that is, $P_1$ can compute the shared key without knowing $k_2$, only $a^{k_2} \textbf{ mod } p$). Further, any person $P_3$ who knows neither $k_1$ nor $k_2$ (but may know any and all of the other values) cannot compute the shared secret efficiently.

{\bf Key Property}: $\forall a \in \mathbb{Z} \, \forall b \in \mathbb{Z} \, \forall g \in \mathbb{Z}^+ \, \forall n \in \mathbb{Z}^+ ((g^a \textbf{ mod } n)^b, (g^b \textbf{ mod } n)^a) \in R_{(\textbf{mod } n)}$

\newpage
\section*{Friday March 5}




\begin{minipage}{4.5in}
\begin{algorithm}[caption={Modular Exponentation; Algorithm 5 in Section 4.2 (page 254)}]
procedure $modular~exponentiation$($b$: integer; 
             $n = (a_{k-1}a_{k-2} \ldots a_1 a_0)_2$, $m$: positive integers)
$x$ := $1$
$power$ := $b$ mod $m$
for $i$:= $0$ to $k-1$
  if $a_i = 1$ then $x$:= $(x \cdot power)$ mod $m$
  $power$ := $(power \cdot power)$ mod $m$
return $x$ $\{x~\textrm{equals}~b^n \textbf{ mod } m\} $
\end{algorithm}
\end{minipage}

Calculate $3^8 \textbf{ mod } 7$

\begin{tabular}{|p{3.5in}|p{3.5in}|}
\hline
{\it Approach 1: Directly} & {\it Approach 2: Using Algorithm 5} \\
\begin{itemize}
\item[] $3^1 \textbf{ mod } 7 = $ 
\item[] $3^2 \textbf{ mod } 7 = $ 
\item[] $3^3 \textbf{ mod } 7 = $ 
\item[] $3^4 \textbf{ mod } 7 = $ 
\item[] $3^5 \textbf{ mod } 7 = $ 
\item[] $3^6 \textbf{ mod } 7 = $ 
\item[] $3^7 \textbf{ mod } 7 = $ 
\item[] $3^8 \textbf{ mod } 7 = $ 
\end{itemize}&
$b = \underline{\phantom{~~3~~}}$, $n = \underline{\phantom{8 = (1000)_2}}$, $k = \underline{\phantom{~~4~~}}$, $m = \underline{\phantom{~~7~~}}$
\begin{center}
\begin{tabular}{c|c|c|c}
$i$ & $a_i$ & $x$ & $power$ \\
\hline
      &           & ~~$1$~~ & $b \textbf{ mod } m = \phantom{3 mod 7 = 3}$\\
  0  &           &\phantom{$1$}& \phantom{$(3 \cdot 3 \mod 7= 9 \mod 7 = 2$}\\
  1  &           &\phantom{$1$}& \phantom{$(2 \cdot 2 \mod 7= 4 \mod 7 = 4$}\\
  2  &           &\phantom{$1$}& \phantom{$(4 \cdot 4 \mod 7= 16 \mod 7 = 2$}\\
  3  &           &\phantom{$2$}& \phantom{$(2 \cdot 2 \mod 7= 4 \mod 7 = 4$}\\
\end{tabular}
\end{center}
\\
How many multiplication operations did we use? & How many multiplication operations did we use?\\
& \\
& \\
\hline
\end{tabular}


\newpage
For a binary relation $R$ on a set $A$: 

$R$ is {\bf reflexive} means $\forall a \in A ~(~(a,a) \in R~)$

$R$ is {\bf symmetric} means $\forall a \in A ~\forall b \in A~\left( ~(a,b) \in R \to (b,a)\in R~\right)$


$R$ is {\bf transitive} means whenever 
$\forall a \in A~ \forall b \in A~ \forall c \in A~\left( ~\left(~ (a,b) \in R \land (b,c) \in R~\right) \to (a,c) \in R~\right)$

$R$ is {\bf antisymmetric} means $\forall a \in A ~\forall b \in A~\left(~\left( ~(a,b) \in R \land (b,a) \in R ~\right) \to a=b~\right)$
\hfill {\bf *New*}


{\it Example}: 
\[
\{ (X_1, X_2 ) \in\mathcal{P}(\mathbb{R})  \times \mathcal{P}(\mathbb{R}) ~|~ |X_1| = |X_2| \}
\]
 is a reflexive, symmetric, transitive binary relation on $\mathcal{P}(\mathbb{R})$.
Is it antisymmetric?

\vfill

{\it Example}: $R_{(\textbf{mod } n)}$ is the set of all pairs of integers $(a, b)$ such that $(a \textbf{ mod } n = b \textbf{ mod } n)$ 
 is a reflexive, symmetric, transitive binary relation on $\mathbb{Z}$.
Is it antisymmetric?

\vfill

{\it Example}: On the set $\mathcal{P}(\{1,2\})$, define the binary relation 
$\{ (X,Y) ~|~X \subseteq Y \}$. Is it reflexive? Is it symmetric? Is it antisymmetric? Is it transitive?

\vfill

{\it Example}: On the set $\mathbb{Z}$, define the binary relation 
$\{ (x,y) ~|~x < y \}$. Is it reflexive? Is it symmetric? Is it antisymmetric? Is it transitive?

\vfill

What's an example of a set and a relation on that set that is reflexive, not symmetric, and transitive?

\vfill

\newpage
{\bf Definition}: ({\it Rosen 9.6}) A relation is a {\bf partial ordering} (or partial order) means 
it is reflexive, anitsymmetric, and transitive.

For a partial ordering, its {\bf Hasse diagram} is a graph whose nodes (vertices) are the elements of the 
domain of the binary relation and which are located such that nodes connected to nodes
above them by (undirected) edges indicate that the relation holds between the lower node and the higher node. 
Moreover, the diagram omits self-loops and omits edges that are guaranteed by transitivity.


{\it Example}: On the set $\mathcal{P}(\{1,2\})$, the binary relation 
$\{ (X,Y) ~|~X \subseteq Y \}$ is a partial ordering. 

\vfill

{\it Example}: On the set $\mathbb{Z}$, define the binary relation 
$\{ (x,y) ~|~x \leq y\}$. 

\vfill

{\it Example}: On the set $\mathbb{Z}$, define the binary relation 
$\{ (x,y) ~|~x \geq y\}$. 

\vfill

{\it Example}: On the set $\mathbb{Z}^+$, define the binary relation 
$\{ (x,y) ~|~ F(x,y) \}$ where $F(x,y)$ means $\exists c \in \mathbb{Z} ~(~y = cx~)$

\vfill


\newpage
\section*{Review quiz questions}
\begin{enumerate}

\item {\bf Monday} Recall that the relation $EQ_{\mathbb{R}}$ on $\mathcal{P}(\mathbb{R})$ is
\[
\{ (X_1, X_2 ) \in\mathcal{P}(\mathbb{R})  \times \mathcal{P}(\mathbb{R}) ~|~ |X_1| = |X_2| \}
\]
and $R_{(\textbf{mod } n)}$ is the set of all pairs of integers $(a, b)$ such that $(a \textbf{ mod } n = b \textbf{ mod } n)$
and $W_1 = \{ (s_1, s_2) \in S \times S ~\mid~ s_1, s_2 \textrm{ are within 1 edit}\}$.

Select all and only the correct items.
\begin{enumerate}
\item $(\mathbb{Z}, \mathbb{R}) \in EQ_{\mathbb{R}}$
\item $(0,1) \in EQ_{\mathbb{R}}$
\item $(\emptyset, \emptyset) \in EQ_{\mathbb{R}}$
\item $(-1,1) \in R_{(\textbf{mod } 2)}$
\item $(1,-1) \in R_{(\textbf{mod } 3)}$ 
\item $(4, 16, 0) \in R_{\textbf{(mod } 4)}$ 
\item $(\A\A\A, \A\A) \in W_1$
\item $(\A\A\A, \C\C\C) \in W_1$
\end{enumerate}



\item {\bf Monday} Recall that 
in a movie recommendation system, each 
user's ratings of movies is represented as a $n$-tuple (with the positive integer $n$  being the number of movies in the database), and each  component of 
the $n$-tuple is an element of  the collection $\{-1,0,1\}$.

Assume there are five movies in the database, so that   each user's ratings
can be represented as a $5$-tuple. Consider the following two binary relations on the  set of all $5$-tuples where each  component 
of the $5$-tuple is an element of  the collection $\{-1,0,1\}$.
\[
G_1 =  \{  (u,v)  \mid 
\text{the ratings of users $u$  and  $v$  agree about the first 
movie in the database} \}
\]
\[
G_2 =  \{  (u,v)  \mid 
\text{the ratings of users $u$  and  $v$  agree about at least two movies} \}
\]
Binary  relations that satisfy certain 
properties (namely,  are  reflexive, symmetric,
and transitive)  can help us group 
elements in a set into categories. 


\begin{enumerate}
    \item {\bf True} or {\bf False}: 
    The  relation $G_1$ holds of  $u=(1,1,1,1,1)$ and
    $v=(-1,-1,-1,-1,-1)$
    \item {\bf True} or {\bf False}: 
    The  relation $G_2$ holds of  $u=(1,0,1,0,-1)$ and
    $v=(-1,0,1,-1,-1)$
    \item {\bf True} or {\bf False}: $G_1$ is reflexive; namely, 
    $\forall u  ~(~(u,u) \in G_1~)$
    \item {\bf True} or {\bf False}:  $G_1$ is symmetric; namely, 
    $\forall u ~\forall  v ~(~(u,v) \in G_1 \to  (v,u) \in G_1~)$
    \item {\bf True} or {\bf False}:  $G_1$ is transitive; namely, 
    $\forall u ~\forall  v  ~\forall w (~\left( (u,v) \in G_1 \wedge (v,w)\in G_1\right) \to  (u,w) \in G_1~)$
    \item {\bf True} or {\bf False}:  $G_2$ is reflexive; namely, 
    $\forall u   ~(~(u,u) \in G_2~)$
    \item {\bf True} or {\bf False}:  $G_2$ is symmetric; namely, 
    $\forall u ~\forall  v  ~(~(u,v)\in G_2 \to  (v,u) \in G_2~)$
    \item {\bf True} or {\bf False}:  $G_2$ is transitive; namely, 
    $\forall u~\forall  v  ~\forall w (~\left( (u,v) \in G_2 \wedge (v,w)\in G_2\right) \to  (u,w) \in G_2~)$
\end{enumerate}

\newpage
\item {\bf Wednesday} Fill in the blanks in the following proof that, for any equivalence relation $R$ on a set $A$,
\[
\forall a \in A ~\forall b \in A~\left( (a,b) \in R \leftrightarrow [a]_R\cap [b]_R \neq \emptyset \right)
\]

{\bf Proof}: Towards a  \textbf{(a)}$\underline{\phantom{\hspace{1.3in}}}$, consider arbitrary elements $a$, $b$ in $A$. We will 
prove the biconditional statement by proving each direction of the conditional in turn.

{\bf Goal 1}: we need to show $(a,b) \in R \to [a]_R\cap [b]_R \neq \emptyset$
{\it Proof of Goal 1}: Assume towards a \textbf{(b)}$\underline{\phantom{\hspace{1.3in}}}$ 
that $(a,b) \in R$. We will work to show
that $[a]_R\cap [b]_R \neq \emptyset$. Namely, we need an element that is in both equivalence classes, that is, we
 need to prove the existential claim $\exists x \in A ~(x \in [a]_{R} \land x \in [b]_{R})$. 
 Towards a \textbf{(c)}$\underline{\phantom{\hspace{1.3in}}}$, consider $x = b$, 
 an element of $A$ by definition. By \textbf{(d)}$\underline{\phantom{\hspace{1.3in}}}$  of $R$, we know that $(b,b) \in R$ 
 and thus, $b \in [b]_{R}$.
 By assumption in this proof, we have that $(a,b) \in R$, and so by  definition of equivalence classes, $b \in [a]_R$.
 Thus, we have proved both conjuncts and this part of the proof is complete.
 
{\bf Goal 2}: we need to show $[a]_R\cap [b]_R \neq \emptyset \to (a,b) \in R $
{\it Proof of Goal 2}: Assume towards a \textbf{(e)}$\underline{\phantom{\hspace{1.3in}}}$ 
that $[a]_R\cap [b]_R \neq \emptyset $. We will work to show
that $(a,b) \in R$. By our assumption, the existential claim $\exists x \in A ~(x \in [a]_{R} \land x \in [b]_{R})$
is true. Call $w$ a witness; thus, $w \in [a]_R$ and $w \in [b]_R$. 
By  definition of equivalence classes, $w \in [a]_R$ means $(a,w) \in R$ and $w \in [b]_R$ means $(b,w) \in R$.
By \textbf{(f)}$\underline{\phantom{\hspace{1.3in}}}$  of $R$, $(w,b) \in R$. By 
\textbf{(g)}$\underline{\phantom{\hspace{1.3in}}}$ of $R$, since $(a,w) \in R$ and $(w,b) \in R$, we have that
$(a,b) \in R$, as required for  this part of the proof.
 
Consider the following expressions as options to fill in the two proofs above. Give your answer as one of the numbers below for each blank a-c. You may use some numbers for more than one blank, but each letter only uses one of the expressions below.

\begin{multicols}{2}
\begin{enumerate}[label=\roman*]
\item exhaustive proof
\item proof by universal generalization
\item proof of existential using a witness
\item proof by cases
\item direct proof
\item proof by contrapositive
\item proof by contradiction
\item reflexivity
\item symmetry
\item transitivity
\end{enumerate}
\end{multicols}

\item {\bf Friday} Consider the binary relation on $\mathbb{Z}^+$ defined by $\{(a,b) ~|~ \exists c \in \mathbb{Z} ( b = ac)\}$.
Select all and only the properties that this binary relation has.
\begin{enumerate}
\item It is reflexive.
\item It is symmetric.
\item It is transitive.
\item It is antisymmetric.
\end{enumerate}


\item {\bf Friday} Consider the partial order on the set $\mathcal{P}(\{1,2,3\})$ given by the binary relation 
$\{ (X,Y) ~|~X \subseteq Y \}$
\begin{enumerate}
\item How many nodes are in the Hasse diagram of this partial order?
\item How many edges are in the Hasse diagram of this partial order?
\end{enumerate}

\end{enumerate}
\end{document}
