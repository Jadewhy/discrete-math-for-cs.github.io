%! app: 
%! outcome: Model systems with tools from discrete mathematics and reason about implications of modelling choices, Data Types, Sets and Functions

\begin{center}
    \begin{tabular}{p{4.4in}p{2.8in}}
    {\bf  Term} & {\bf Examples}:\\
    &  (add additional examples from class)\\
    \hline 
    {\bf set} \newline
    unordered collection of elements & $7 \in \{43, 7, 9 \}$ \qquad $2 \notin \{43, 7, 9 \}$ \\
    {\it Equal means agree on membership of all elements}& \\
    \hline
    {\bf $n$-tuple} \newline
    ordered sequence of elements with $n$ ``slots" & \\
    {\it Equal means corresponding components equal}& \\
    \hline
    {\bf string} \newline
    ordered finite sequence of elements each from specified
    set & \\
    {\it Equal means same length and corresponding characters equal}
    \end{tabular}
    \end{center}
    \[
    \{ -1, 1\} \qquad 
    \{0, 0 \} \qquad
    \{-1, 0, 1 \} \qquad
    \mathbb{Z} \qquad
    \mathbb{N} = \{ x \in \mathbb{Z} \mid x \geq 0 \} \qquad
    \emptyset \qquad
    \mathbb{Z}^+ = \{ x \in \mathbb{Z}  \mid x > 0 \}
    \]
    
    \vfill
    
    {\it Which of the sets above are defined using the roster method? Which are defined using set builder notation?}
    
    {\it Which of the sets above have $0$ as an element?}
    
    {\it Can you write any of the sets above more simply?}
    
    \vfill