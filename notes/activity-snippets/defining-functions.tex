%! app: TODOapp
%! outcome: TODOoutcome

\fbox{\parbox{\textwidth}{%
{\bf New! Defining functions} A function is defined by its (1) domain, (2) codomain, and (3) rule assigning each 
element in the domain exactly one element in the codomain.\\

The domain and codomain are nonempty sets.

The rule can be depicted as a table, formula, or English description.
}}


Examples: 



\vfill 


{\bf Definition} (Of a function, recursively) A function \textit{rnalen} that computes the length of RNA strands in $S$ is defined by:
\[
\begin{array}{llll}
& & \textit{rnalen} : S & \to \mathbb{Z}^+ \\
\textrm{Basis Step:} & \textrm{If } b \in B\textrm{ then } & \textit{rnalen}(b) & = 1 \\
\textrm{Recursive Step:} & \textrm{If } s \in S\textrm{ and }b \in B\textrm{, then  } & \textit{rnalen}(sb) & = 1 + \textit{rnalen}(s)
\end{array}
\]

The domain of \textit{rnalen} is \underline{\phantom{$S$\hspace{1.5in}}}.
The codomain of \textit{rnalen} is \underline{\phantom{$\mathbb{Z}^+$\hspace{1.5in}}}.
\[
rnalen(\A\C\U) = \underline{\phantom{\hspace{5in}}}
\]

\vfill

{\it Extra example}: A function \textit{basecount} that computes the number of a given base $b$ appearing in a RNA strand $s$ is defined recursively:  {\it fill in codomain and sample function
applications}
\[
\begin{array}{llll}
& & \textit{basecount} : S \times B & \to \phantom{\mathbb{N}} \\
\textrm{Basis Step:} &  \textrm{If } b_1 \in B, b_2 \in B & \textit{basecount}(b_1, b_2) & =
        \begin{cases}
            1 & \textrm{when } b_1 = b_2 \\
            0 & \textrm{when } b_1 \neq b_2 \\
        \end{cases} \\
\textrm{Recursive Step:} & \textrm{If } s \in S, b_1 \in B, b_2 \in B &\textit{basecount}(s b_1, b_2) & =
        \begin{cases}
            1 + \textit{basecount}(s, b_2) & \textrm{when } b_1 = b_2 \\
            \textit{basecount}(s, b_2) & \textrm{when } b_1 \neq b_2 \\
        \end{cases}
\end{array}
\]
\[
basecount(\A\C\U,\A) = \underline{\phantom{\hspace{5in}}}
\]
\[
basecount(\A\C\U,\G) = \underline{\phantom{\hspace{5in}}}
\]
