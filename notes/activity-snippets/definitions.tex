%! app: 
%! outcome: data types, translating, important sets, write set definition, function and relation definitions

\begin{center}
\begin{tabular}{|llp{9.8cm}|}
\hline
{\bf Term} & {\bf Notation Example(s)} & {\bf We say in English \ldots } \\
\hline
%$n$-tuple & $(x_1, x_2, x_3)$ & The 3-tuple of $x_1$, $x_2$, and $x_3$ \\
%          & $(3, 4)$ & The 2-tuple or {\bf ordered pair} of $3$ and $4$ \\
sequence & $x_1, \ldots, x_n$ & A sequence $x_1$ to $x_n$ \\
%         & $x_1, \ldots, x_n$ where $n = 0$ & An empty sequence \\
%         & $x_1, \ldots, x_n$ where $n = 1$ & A sequence containing just $x_1$ \\
%         & $x_1, \ldots, x_n$ where $n = 2$ & A sequence containing just $x_1$ and $x_2$ in order \\
%         & $x_1, x_2$ & A sequence containing just $x_1$ and $x_2$ in order \\
summation & $\sum_{i=1}^n x_i$ or $\displaystyle{\sum_{i=1}^n x_i}$ & The sum of the terms of the sequence $x_1$ to $x_n$ \\
&&\\
%maximum & $\displaystyle \max(x, y)$ & The max of $x$ and $y$, when they are numbers \\ % Note that this is different than summation!
%        & $\displaystyle \max_{1 \leq i \leq n} x_i$ & The max of $x_1$ to $x_n$, when they are numbers \\ % Also different from display
%&&\\
%set & & Unordered collection of objects. The set of \ldots \\
all reals & $\mathbb{R}$ & The (set of all) real numbers (numbers on the number line)\\
all integers & $\mathbb{Z}$ & The (set of all) integers (whole numbers including negatives, zero, and positives) \\
all positive integers & $\mathbb{Z}^+$ & The (set of all) strictly positive integers \\
all natural numbers & $\mathbb{N}$ & The (set of all) natural numbers. {\bf Note}: we use the convention that $0$ is a natural number. \\
%roster method & $\{43, 7, 9\}$ & The set whose elements are $43$, $7$, and $9$\\
%              & $\{9, \mathbb{N}\}$ & The set whose elements are $9$ and $\mathbb{N}$\\
%&&\\
%set builder notation & $\{ x \in \mathbb{Z} \mid x > 0\}$ & The set of all $x$ from the integers such that $x$ is greater than $0$ \\
%                     & $\{ 3x  \mid x \in \mathbb{Z} \}$ & The set of all integer multiples of $3$. {\bf Note}: we use the convention that writing two numbers next to each other means multiplication. \\
&&\\
%function rule definition & $f(x) = x + 4$ & Define $f$ of $x$ to be $x + 4$ \\
piecewise rule definition & $f(x) = \begin{cases} x & \text{if~}x \geq 0 \\ -x & \text{if~}x<0\end{cases}$ &
Define $f$ of $x$ to be $x$ when $x$ is nonnegative and to be $-x$ when $x$ is negative\\
function application & $f(7)$ & $f$ of $7$ {\bf or} $f$ applied to $7$ {\bf or} the image of $7$ under $f$\\
                     & $f(z)$ & $f$ of $z$ {\bf or} $f$ applied to $z$ {\bf or} the image of $z$ under $f$\\
                     & $f(g(z))$ & $f$ of $g$ of $z$ {\bf or} $f$ applied to the result of $g$ applied to $z$ \\
&&\\
absolute value & $\lvert -3 \rvert$ & The absolute value of $-3$ \\
square root & $\sqrt{9}$ & The non-negative square root of $9$ \\
&&\\
%summation notation & $\displaystyle \sum_{i=1}^n i$ & The sum of the integers from $1$ to $n$, inclusive \\
%                    & $\displaystyle \sum_{i=1}^n i^2 - 1$ & The sum of $i^2 - 1$ ($i$ squared minus $1$) for each $i$ from $1$ to $n$, inclusive \\
%&&\\
%quotient, integer division & $n~\textbf{div}~m$ & The (integer) quotient upon dividing $n$ by $m$; informally: divide and then 
%drop the fractional part\\
%modulo, remainder & $n~\textbf{mod}~m$ & The remainder upon dividing $n$ by $m$ \\

\hline
\end{tabular}
\end{center}