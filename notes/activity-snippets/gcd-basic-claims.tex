%! app: TODOapp
%! outcome: divisibility and primes, important sets

{\bf Claim}: For any integers $a,b$ (not both zero), $gcd(~(a,b)~) \geq 1$.
{\it In other words, $1$ is the smallest possible value of the $gcd$}.

{\bf Proof}: WTS that 
\[
    \forall a \in \mathbb{Z} \forall b \in \mathbb{Z} \left( ~\lnot ( a = 0 \land b = 0)~ \to 
    ~(F(~(1,a)~) \land F(~(1,b)~) \right)
\]
because this says that $1$ is a common factor of any integers that are not both zero. The gcd 
is the greatest common factor so would then need to be greater than or equal to any common factor.

\vspace{250pt}

{\bf Claim}: For any positive $a,b$, $gcd(~(a,b)~) \leq a$ and $gcd( ~(a,b)~) \leq b$.

{\bf Proof} Using the definition of gcd and the fact that factors of a positive integers
are less then or equal to that integer.

\vspace{50pt}