\begin{tabular}{clc}
    $\mathbb{Z}$ &  The  set of integers  & $\{ \ldots, -2, -1, 0,  1, 2, \ldots\}$ \\
    $\mathbb{Z}^+$ &  The  set of positive integers  & $\{1, 2, \ldots\}$ \\
    $\mathbb{N}$ &  The  set of nonnegative integers  & $\{0, 1, 2, \ldots\}$ \\
    $\mathbb{Q}$ &  The  set of rational numbers  & $\left\{ \frac{p}{q} \mid p \in \mathbb{Z}  \text{ and  } q  \in \mathbb{Z} \text{ and } q \neq  0 \right\}$\\
    $\mathbb{R}$ & The  set  of  real numbers &  \\
    \end{tabular}
    \[
    \underline{\phantom{\mathbb{Z}^+}} ~~\subsetneq~~ \underline{\phantom{\mathbb{N}~}} ~~\subsetneq ~~\underline{\phantom{\mathbb{Z}~}}~~ \subsetneq~~ \underline{\phantom{\mathbb{Q}~}} 
    ~~\subsetneq~~ \underline{\phantom{\mathbb{R}~}} 
    \]
    
    
    The above sets are all {\bf infinite}.
    
    
    A {\bf finite} set is one whose distinct elements can be counted by a natural number.
    
    {\it Examples of finite sets}: $\emptyset$ , $\{ \sqrt{2} \}$
    
    
    {\bf Motivating question}: Are some of the above sets {\it bigger than} others?