%! app: TODOapp
%! outcome: TODOoutcome

Fill in the blanks in the following proof that, for any equivalence relation $R$ on a set $A$,
\[
\forall a \in A ~\forall b \in A~\left( (a,b) \in R \leftrightarrow [a]_R\cap [b]_R \neq \emptyset \right)
\]

{\bf Proof}: Towards a  \textbf{(a)}$\underline{\phantom{\hspace{1.3in}}}$, consider arbitrary elements $a$, $b$ in $A$. We will 
prove the biconditional statement by proving each direction of the conditional in turn.

{\bf Goal 1}: we need to show $(a,b) \in R \to [a]_R\cap [b]_R \neq \emptyset$
{\it Proof of Goal 1}: Assume towards a \textbf{(b)}$\underline{\phantom{\hspace{1.3in}}}$ 
that $(a,b) \in R$. We will work to show
that $[a]_R\cap [b]_R \neq \emptyset$. Namely, we need an element that is in both equivalence classes, that is, we
 need to prove the existential claim $\exists x \in A ~(x \in [a]_{R} \land x \in [b]_{R})$. 
 Towards a \textbf{(c)}$\underline{\phantom{\hspace{1.3in}}}$, consider $x = b$, 
 an element of $A$ by definition. By \textbf{(d)}$\underline{\phantom{\hspace{1.3in}}}$  of $R$, we know that $(b,b) \in R$ 
 and thus, $b \in [b]_{R}$.
 By assumption in this proof, we have that $(a,b) \in R$, and so by  definition of equivalence classes, $b \in [a]_R$.
 Thus, we have proved both conjuncts and this part of the proof is complete.
 
{\bf Goal 2}: we need to show $[a]_R\cap [b]_R \neq \emptyset \to (a,b) \in R $
{\it Proof of Goal 2}: Assume towards a \textbf{(e)}$\underline{\phantom{\hspace{1.3in}}}$ 
that $[a]_R\cap [b]_R \neq \emptyset $. We will work to show
that $(a,b) \in R$. By our assumption, the existential claim $\exists x \in A ~(x \in [a]_{R} \land x \in [b]_{R})$
is true. Call $w$ a witness; thus, $w \in [a]_R$ and $w \in [b]_R$. 
By  definition of equivalence classes, $w \in [a]_R$ means $(a,w) \in R$ and $w \in [b]_R$ means $(b,w) \in R$.
By \textbf{(f)}$\underline{\phantom{\hspace{1.3in}}}$  of $R$, $(w,b) \in R$. By 
\textbf{(g)}$\underline{\phantom{\hspace{1.3in}}}$ of $R$, since $(a,w) \in R$ and $(w,b) \in R$, we have that
$(a,b) \in R$, as required for  this part of the proof.
 
Consider the following expressions as options to fill in the two proofs above. Give your answer as one of the numbers below for each blank a-c. You may use some numbers for more than one blank, but each letter only uses one of the expressions below.

\begin{multicols}{2}
\begin{enumerate}[label=\roman*]
\item exhaustive proof
\item proof by universal generalization
\item proof of existential using a witness
\item proof by cases
\item direct proof
\item proof by contrapositive
\item proof by contradiction
\item reflexivity
\item symmetry
\item transitivity
\end{enumerate}
\end{multicols}