%! app: numbers
%! outcome: Recursively Defined Sets and Functions, Proofs with Recursively Defined Sets and Functions, Induction Types, Mathematical Induction

Consider the following statement: For $n > 0$, the sum of the first $n$ positive integers, also written as $\sum_{i=1}^n 
i$, is equal to $$\left(\dfrac{n \cdot (n + 1)}{2}\right)$$.

For example, when $n = 3$, the statement would mean that $1+2+3 = \left(\dfrac{3 \cdot (3 + 1)}{2}\right)$.

Consider the following (start to) a proof of this statement:

Basis Step: Choose $n = 1$ as the basis step. Applying the formula from the original statement, we find that $\dfrac{1 \cdot (1 + 1)}{2}$ is equal to $1$. Since the total sum of the first positive integer is $1$, these are equal and the basis step is complete.

Recursive Step: Consider an arbitrary $k \geq 1$.  Towards a direct proof, we assume (as the induction hypothesis) that the sum of the first $k$ positive integers is $\left(\dfrac{k \cdot (k + 1)}{2}\right)$. We want to show that the sum of the first $k + 1$ positive integers is $$\left( \dfrac{(k + 1) \cdot ((k + 1) + 1)}{2}\right)$$.

[Proof would continue here...]

\begin{enumerate}
   \item In a recursive definition of the function that gives the sum of the first $n$ positive integers, the domain is 
    \begin{enumerate}
        \item $\mathbb{N}$
        \item $\mathbb{Z}^+$
        \item $\mathbb{Z}$
    \end{enumerate}

   \item In a recursive definition of the function that gives the sum of the first $n$ positive integers, the basis step is
    \begin{enumerate}
        \item $\sum_{i=1}^1 i = 1$
        \item $\sum_{i=1}^n 1 = n$
        \item None of the above.
    \end{enumerate}

   \item In a recursive definition of the function that gives the sum of the first $n$ positive integers, the recursive step is
    \begin{enumerate}
        \item If $n$ is a positive integer, $\sum_{i=1}^n i = n$
        \item If $n$ is a positive integer, $\sum_{i=1}^n i =\left(  \sum_{i=1}^n i \right)+ 1$
        \item If $n$ is a positive integer, $\sum_{i=1}^{n+1} i =\left(  \sum_{i=1}^n i \right) + 1$
        \item If $n$ is a positive integer, $\sum_{i=1}^{n+1} i =\left(  \sum_{i=1}^n i \right) + n$
        \item If $n$ is a positive integer, $\sum_{i=1}^{n+1} i =\left(  \sum_{i=1}^n i \right)+ (n+1)$
    \end{enumerate}

    \item The proof technique used here is:
    \begin{enumerate}
        \item Structural induction
        \item Mathematical induction
        \item Strong induction
    \end{enumerate}
    
    \item Which of these is both true, and a useful next step in the proof?
    
    \begin{enumerate}
        \item By the induction hypothesis, we know that $$\left( \dfrac{(k + 1) \cdot ((k + 1) + 1)}{2}\right) = \left(\dfrac{k \cdot (k + 1)}{2}\right)$$
        \item By the induction hypothesis, we know that $$\left( \dfrac{(k + 1) \cdot ((k + 1) + 1)}{2}\right) > \left(\dfrac{k \cdot (k + 1)}{2}\right)$$
        \item By the induction hypothesis and the definition of the statement we're proving, we know that $$\left(\dfrac{k \cdot (k + 1)}{2}\right) + k$$ is the sum of the first $k + 1$ positive integers.
        \item By the induction hypothesis and the definition of the statement we're proving, we know that $$\left(\dfrac{k \cdot (k + 1)}{2}\right) + k + 1$$ is the sum of the first $k + 1$ positive integers.
        \item None of the above
    \end{enumerate}
\end{enumerate}
