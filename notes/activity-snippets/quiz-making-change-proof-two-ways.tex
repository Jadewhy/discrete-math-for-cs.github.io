%! app: TODOapp
%! outcome: TODOoutcome

In this question, we'll consider two possible 
proofs of the statement
\[
    \forall n  \in  \mathbb{Z}^{\geq 8}  \exists x \in \mathbb{N}  \exists y \in \mathbb{N}  (  5x+3y =  n)
\]

{\it Note}: one way of thinking about this statement is that 
if we had postage stamps worth $5$ cents and $3$ cents, 
we could use some (nonnegative) number of them to 
pay any amount of postage above $8$ cents.

\begin{enumerate}
\item First approach: {\bf Proof of $\star$ by mathematical induction} ($b=8$)

{\bf Basis step}:  WTS property is true about  $8$
\vspace{50pt}

{\bf Recursive step}: Consider an  arbitrary  $n \geq 8$.
Assume (as the  IH) that  there are nonnegative integers
$x, y$ such that $n =  5x +  3y$.  WTS
that there are nonnegative integers $x', y'$ such
that  $n+1 = 5x' +  3y'$.  We consider two cases, 
depending on  whether  any  $5$ cent coins
are used for $n$.

\vspace{1in}

{\it Case 1}:  Assume $x \geq  1$.
Define $x' = x-1$ and $y'=y+2$ (both in  $\mathbb{N}$ by case assumption).




Calculating:
\begin{align*}
5x' +  3y' &\overset{\text{by def}}{=}  5(x-1) +  3(y+2)  = 5x -  5 +3y +   6  \\
&\overset{\text{rearranging}} = (5x+3y) -5  + 6\\
& \overset{\text{IH}}{=} n-5+6 =  n+1
\end{align*}

\vspace{1in}

{\it  Case 2}: Assume $x = 0$.  Therefore  $n  = 3y$,  so 
since  $n \geq 8$, $y \geq 3$. Define $x' = 2$ and $y'=y-3$
(both in $\mathbb{N}$ by case assumption).
Calculating:
\begin{align*}
5x' +  3y' &\overset{\text{by def}}{=}  5(2) +  3(y-3)  = 10  +3y -9  \\
&\overset{\text{rearranging}} = 3y +10 -9 \\
&\overset{\text{IH and case}}{=} n+10-9 =  n+1
\end{align*}

\item  Second approach: {\bf Proof of $\star$ by strong induction} ($b=8$ and $j=2$)

{\bf Basis step}:  WTS property is true about  $8, 9, 10$
\vspace{50pt}

{\bf Recursive step}: Consider an  arbitrary  $n \geq 10$.
Assume (as the  IH) that the property is true about  each of $8, 9, 10, \ldots, n$.  
WTS
that there are nonnegative integers $x', y'$ such
that  $n+1 = 5x' +  3y'$.

\end{enumerate}