%! app: TODOapp
%! outcome: TODOoutcome

In this question, we'll consider two possible 
proofs of the statement
\[
    \forall n  \in  \mathbb{Z}^{\geq 8}  \exists x \in \mathbb{N}  \exists y \in \mathbb{N}  (  5x+3y =  n)
\]

\begin{enumerate}
\item First approach, using mathematical induction ($b=8$)

{\bf Basis step}:  WTS property is true about  $8$
Consider the witnesses $x = 1$, $y=1$. These 
are nonnegative integers and $5 \cdot 1 + 3 \cdot 1 = 8$, as
required.

{\bf Recursive step}: Consider an  arbitrary  $n \geq 8$.
Assume (as the  IH) that  there are nonnegative integers
$x, y$ such that $n =  5x +  3y$.  WTS
that there are nonnegative integers $x', y'$ such
that  $n+1 = 5x' +  3y'$.  We consider two cases, 
depending on  whether  any  $5$ cent coins
are used for $n$.

{\it Case 1}:  Assume $x \geq  1$.
Define $x' = x-1$ and $y'=y+2$ (both in  $\mathbb{N}$ by case assumption).


Calculating:
\begin{align*}
5x' +  3y' &\overset{\text{by def}}{=}  5(x-1) +  3(y+2)  = 5x -  5 +3y +   6  \\
&\overset{\text{rearranging}} = (5x+3y) -5  + 6\\
& \overset{\text{IH}}{=} n-5+6 =  n+1
\end{align*}


{\it  Case 2}: Assume $x = 0$.  Therefore  $n  = 3y$,  so 
since  $n \geq 8$, $y \geq 3$. Define $x' = 2$ and $y'=y-3$
(both in $\mathbb{N}$ by case assumption).
Calculating:
\begin{align*}
5x' +  3y' &\overset{\text{by def}}{=}  5(2) +  3(y-3)  = 10  +3y -9  \\
&\overset{\text{rearranging}} = 3y +10 -9 \\
&\overset{\text{IH and case}}{=} n+10-9 =  n+1
\end{align*}


Why was the recursive step split into two cases?
\begin{enumerate}
    \item Because there are two variables $x$ and $y$ that need witnesses.
    \item Because the statement has alternating quantifiers $\forall$ and $\exists$
    \item Because the witness values need to be nonnegative and subtraction may lead to negative values.
    \item Because the domain is all integers greater than or equal to $8$.
    \item Because there are two steps in the recursive definition of $\mathbb{N}$
\end{enumerate}

\item  Second approach, by strong induction ($b=8$ and $j=2$)

{\bf Basis step}:  WTS property is true about  $8, 9, 10$
\begin{itemize}
\item Consider the witnesses $x = 1$, $y=1$. These 
are nonnegative integers and $5 \cdot 1 + 3 \cdot 1 = 8$, as
required.
\item Consider the witnesses $x = 0$, $y=3$. These 
are nonnegative integers and $5 \cdot 0 + 3 \cdot 3 = 9$, as
required.
\item Consider the witnesses $x = 2$, $y=0$. These 
are nonnegative integers and $5 \cdot 2 + 3 \cdot 0 = 10$, as
required.
\end{itemize}

{\bf Recursive step}: Consider an  arbitrary  $n \geq 10$.
Assume, as the strong induction hypothesis, 
that the property is true about each of $8, 9, 10, \ldots, n$.  
WTS
that there are nonnegative integers $x', y'$ such
that  $n+1 = 5x' +  3y'$. 

Since \underline{Blank 1}, 
by the strong induction hypothesis, there are nonnegative integers
$x, y$ such that $(n+1) - 3 = 5x + 3y$.
Choosing \underline{Blank 2} works because
\[
    5x' + 3y' = 5x + 3y + 3 = (n+1) - 3 + 3 = n+1.
\]


\begin{enumerate}
    \item Choose a true and useful statement to fill in Blank 1.
    \begin{enumerate}
        \item $n \geq 10$ and hence $(n+1) - 3 \geq 8$
        \item $n \geq 8$ and hence $(n+1) \geq 9$
        \item $n \geq 8$ and hence $(n+1) \geq 9$
    \end{enumerate}
    \item Choose the appropriate statement to fill in Blank 2.
    \begin{enumerate}
        \item $x' = x, y' = y$
        \item $x' = x+1, y' = y+1$
        \item $x' = x+1, y' = y$
        \item $x' = x, y' = y+1$
        \item $x' = x-1, y' = y-1$
        \item $x' = x-1, y' = y$
        \item $x' = x, y' = y-1$
    \end{enumerate}
\end{enumerate}
\end{enumerate}