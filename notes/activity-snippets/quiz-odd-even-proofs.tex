%! app: numbers
%! outcome: Proposition and its Proof Strategies, Universal Generalization, Using Proofs to Evaluate Propositions, Proof Technique, Proof by Contradiction, Div and Mod Operators, Converse, Contrapositive, and Inverse, Factoring and Primes, Arbitrary, Important Number Sets, Use DeMorgan’s Law in Translating Quantified Statements

{\it Goals for this question: recognize that we can prove the same statement
in different ways.  Trace proofs and justify why they are valid.}


 By definition, an integer $n$ is {\bf even} means that there is an integer $a$ such that $n = 2a$; 
an integer $n$ is {\bf odd} means that there is an integer $a$ such that $n = 2a+1$.  Equivalently, 
an integer $n$ is {\bf even} means $n ~\textbf{ mod }~2 = 0$; an integer $n$ is {\bf odd} means $n ~\textbf{ mod }~2 = 1$.  Also, an integer is even if and only if it is not odd.

{\it You can refer to any of the above definitions and claims in your proofs.}

Below are two proofs of the same statement: fill in the blanks with the 
expressions below.

{\bf Claimed statement}:  \textbf{(a)}$\underline{\phantom{\hspace{1.3in}}}$
\begin{quote}

{\bf Proof 1}: Using De Morgan's law for quantifiers, 
we can rewrite this statement as a universal of the negation of the body of the statement.
Towards a proof by universal generalization, let $x$ be an arbitrary element of $\mathbb{Z}$. Then we need to show that
$$\textbf{(b)}\underline{\phantom{\hspace{1.3in}}}$$

We proceed by contradiction to show that $$(x \textrm{ is odd} \land x^2 \textrm{ is even}) \to \textbf{(c)}\underline{\phantom{\hspace{1.3in}}}$$
We assume by direct proof that $(x \textrm{ is odd} \land x^2 \textrm{ is even})$. Then, $(x^2 \textrm{ is even})$ follows directly from this assumption, so by definition 
of conjunction, we must show that $(x^2 \textrm{ is not even})$ to complete the proof.
From the assumption, we have that $(x \textrm{ is odd})$.  Applying the definition of odd, $x = 2k + 1$ for some $k \in \mathbb{Z}$. Then $x^2 = 4k^2 + 4k + 1$. We can rewrite the right hand side to $2(2k^2 + 2k) + 1$. This shows that $x^2$ is odd by the definition of odd, since choosing $j = 2k^2 + 2k$ gives us $j \in \mathbb{Z}$ with $x^2 = 2j + 1$. Since a number is either even or odd and not both, and $x^2$ is odd, then it must not be even. 
This concludes the proof, as we have assumed the negation of the original statement and deduced a contradiction
from this assumption.
\end{quote}

\newpage
\begin{quote}{\bf Proof 2}: 

    \begin{tabular}{l p{2.5in}}
    1. \begin{tabular}{l}
        \textbf{To Show} $\forall x \in \mathbb{Z} \neg (x \textrm{ is odd} \land x^2 \textrm{ is even})$\\
    \end{tabular}
    & Rewriting statement using De Morgan's law for quantifiers
 \\ 
   2. \begin{tabular}{l}
        \textbf{Choose arbitrary} $x \in\mathbb{Z}$ \\
        \textbf{To Show} \textbf{(d)}$\underline{\phantom{\hspace{1.3in}}}$\\
    \end{tabular}
    & By \textbf{(e)}$\underline{\phantom{\hspace{1.3in}}}$\\
 \\
    3. \begin{tabular}{l}
        \textbf{To Show}
         $x \textrm{ is odd} \to \neg  (x^2 \textrm{ is even})$
    \end{tabular}
    & Rewrite previous {\bf To Show} using logical equivalence
    \\
    4. \begin{tabular}{l}
        \textbf{Assume } $x \textrm{ is odd}$ \\
        \textbf{To Show } $\neg  (x^2 \textrm{ is even})$ \\
    \end{tabular}
    & By \textbf{(f)}$\underline{\phantom{\hspace{1.3in}}}$\\
    \\    
    5. \begin{tabular}{l}
        \textbf{To Show} $x^2 \textrm{ is odd}$
    \end{tabular}
    & Rewrite previous {\bf To Show} using definition of even, odd
    \\
    6. \begin{tabular}{l}
        \textbf{Use the witness} $k$, an integer,\\
        where $x = 2k+1$\\
    \end{tabular}
     & By existential definition of $x$ being odd \\
    \\
    7. \begin{tabular}{l}
        \textbf{Choose the witness} \\
        $j = 2k^2 + 2k$, an integer\\
        \textbf{To Show} $x^2 = 2j+1$
    \end{tabular}
     & Show this new {\bf To Show} is true to prove the existential definition of $x^2$ being odd\\
    \\
    8.\begin{tabular}{l}
        \textbf{To Show} $(2k+1)^2  = 2j+1$
    \end{tabular}
    & Rewrite previous {\bf To Show} using definition of $k$
    \\
    9.  \begin{tabular}{l}
        \textbf{To Show} $(2k+1)^2  = 2(2k^2 + 2k) + 1$
    \end{tabular}
    & Rewrite previous {\bf To Show} using definition of $j$
    \\
    10. \begin{tabular}{l}
        \textbf{To Show } $T$ \\
    \end{tabular}
     & By algebra: multiplying out the LHS; factoring the RHS\\
    QED & Because we got to $T$ only by rewriting \textbf{To Show} to equivalent statements, using valid proof techniques and definitions. \\
    \end{tabular}
\end{quote}


Consider the following expressions as options to fill in the two proofs above. Give your answer as one of the numbers below for each blank a-c. You may use some numbers for more than one blank, but each letter only uses one of the expressions below.

\begin{multicols}{2}
\begin{enumerate}[label=\roman*.]
    \item $\exists x \in \mathbb{Z} \, (x \textrm{ is odd} \land x^2 \textrm{ is even})$
    \item $\neg \exists x \in \mathbb{Z} \, (x \textrm{ is odd} \land x^2 \textrm{ is even})$
    \item $\exists x \in \mathbb{Z} \, (x \textrm{ is odd} \land x \textrm{ is even})$
    \item $\neg \exists x \in \mathbb{Z} \, (x \textrm{ is odd} \land x \textrm{ is even})$
    \item $\exists x \in \mathbb{Z} \, (x^2 \textrm{ is odd} \land x^2 \textrm{ is even})$
    \item $\neg \exists x \in \mathbb{Z} \, (x^2 \textrm{ is odd} \land x^2 \textrm{ is even})$
    \item $(x^2 \textrm{ is even} \land x^2 \textrm{ is not even})$
    \item $\neg (x \textrm{ is odd} \land  x^2 \textrm{ is even})$
    \item $(x \textrm{ is odd} \land  x^2 \textrm{ is even})$
    \item $(x \textrm{ is odd} \land  x \textrm{ is not odd})$
    \item $\neg (x \textrm{ is odd} \land  x \textrm{ is not odd})$
    \item $x^2 \textrm{ is even}$
    \item $x^2 \textrm{ is odd}$
    \item universal generalization
    \item proof by cases
    \item direct proof
    \item proof by contraposition
    \item proof by contradiction
\end{enumerate}
\end{multicols}
