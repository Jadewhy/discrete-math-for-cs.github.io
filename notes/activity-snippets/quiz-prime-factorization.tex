%! app: TODOapp
%! outcome:  induction flavors, mathematical induction proofs, strong induction proofs

Recall that an integer $p$ greater than $1$ is called {\bf prime} if the only positive factors of 
$p$ are $1$ and $p$. Fill in the following blanks in the proof of the {\bf Theorem} (Rosen p336): 
Every positive integer {\it greater than 1} is a product of (one or more) primes.

{\bf Proof}: We proceed by \underline{~~~BLANK for part (a)~~~}.
\begin{itemize}
\item[] Basis step: We want to show that the property is about $2$. Since $2$ is itself prime,
it is already written as a product of (one) prime.
\item[] Recursive step: Consider an arbitrary integer $n \geq 2$.  Assume, as the induction hypothesis,
that \underline{~~~BLANK for part (c)~~~}. We want to show that $n+1$ can be written 
as a product of primes.  There are two cases to consider: $n+1$ is itself prime or it is composite.
In the first case, we assume $n+1$ is prime and then immediately it is written as a product
of (one) prime so we are done.  In the second case, we assume that $n+1$ is composite
so there are integers $x$ and $y$ where $n+1 = xy$ and each of them is between $2$ and $n$
(inclusive).  Therefore, the induction hypothesis applies to each of $x$ and $y$ so each 
of these factors of $n+1$ can be written as a product of primes.  Multiplying these products together, 
we get a product of primes that gives $n+1$, as required.  Since both cases give the necessary
conclusion, the proof by cases for the recursive step is complete.
\end{itemize}
\begin{enumerate}
\item  The proof technique used here is:
    \begin{enumerate}
        \item Structural induction
        \item Mathematical induction
        \item Strong induction
    \end{enumerate}
\item How many basis steps are used in this proof?
\newpage
\item What is the induction hypothesis?
\begin{enumerate}
\item That $n$ can be written as a product of (one or more) primes.
\item That each integer between $0$ and $n$ (inclusive) can be written as a product of (one or more) primes.
\item That each integer between $1$ and $n+1$ (inclusive) can be written as a product of (one or more) primes.
\item That each integer between $2$ and $n$ (inclusive) can be written as a product of (one or more) primes.
\item That each integer between $3$ and $n+1$ (inclusive) can be written as a product of (one or more) primes.
\end{enumerate}
\end{enumerate}
