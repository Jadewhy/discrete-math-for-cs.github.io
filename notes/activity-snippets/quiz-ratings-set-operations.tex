%! app: TODOapp
%! outcome: TODOoutcome

RNA is made up of strands of four different bases that encode genomic information
in specific ways. The bases are elements of the set 
$B  = \{\A, \C, \G, \U \}$. The set of RNA strands $S$ is defined (recursively) by:

\[
\begin{array}{ll}
\textrm{Basis Step: } & \A \in S, \C \in S, \U \in S, \G \in S \\
\textrm{Recursive Step: } & \textrm{If } s \in S\textrm{ and }b \in B \textrm{, then }sb \in S
\end{array}
\]

A function \textit{rnalen} that computes the length of RNA strands in $S$ is defined by:
\[
\begin{array}{llll}
& & \textit{rnalen} : S & \to \mathbb{Z}^+ \\
\textrm{Basis Step:} & \textrm{If } b \in B\textrm{ then } & \textit{rnalen}(b) & = 1 \\
\textrm{Recursive Step:} & \textrm{If } s \in S\textrm{ and }b \in B\textrm{, then  } & \textit{rnalen}(sb) & = 1 + \textit{rnalen}(s)
\end{array}
\]

\begin{enumerate}
\item How many distinct elements are in the set described using set builder notation as 
\[
\{ x \in S \mid rnalen(x) = 1\} \qquad ?
\]

\item How many distinct elements are in the set described using set builder notation as 
\[
\{ x \in S \mid rnalen(x) = 2\} \qquad ?
\]

\item How many distinct elements are in the set described using set builder notation as 
\[
\{ rnalen(x) \mid x \in S \text{ and } rnalen(x) = 2\} \qquad ?
\]


\item How many distinct elements are in the set obtained as the result
of the set-wise concatenation $\{ \A\A, \A\C \} \circ \{\U, \A\A \}$?

\item How many distinct elements are in the set obtained as the result
of the Cartesian product $\{ \A\A, \A\C \} \times \{\U, \A\A \}$?

\item {\bf True} or {\bf False}: There is an example of an RNA strand that is both in the set obtained as the result
of the set-wise concatenation $\{ \A\A, \A\C \} \circ \{\U, \A\A \}$ and in the set obtained as the result of the 
Cartesian product $\{ \A\A, \A\C \} \times \{\U\A, \A\A \}$

\end{enumerate}
{\it Bonus - not for credit: Describe each of the sets above using roster method.}
