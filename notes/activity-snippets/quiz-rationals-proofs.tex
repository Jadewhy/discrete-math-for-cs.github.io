%! app: numbers
%! outcome: Universal and Existential Statements, Infinite Domains, Nested Quantifiers, Quantified Statement, Proposition and its Proof Strategies, Universal Generalization, Using Proofs to Evaluate Propositions, Proof Technique, Important Number Sets, Translate using logical connectors and universal and existential quantifiers, Use DeMorgan’s Law in Translating Quantified Statements

{\it  Goals for this question: Reason through
multiple nested quantifiers. Fluently use the definition and properties of the set of rationals. 
}

Recall the definition of the set of rational numbers, $\mathbb{Q} = \left\{ \frac{p}{q} \mid p \in \mathbb{Z}  \text{ and  } q  \in \mathbb{Z} \text{ and } q \neq  0 \right\}$.
We define the set of {\bf irrational} numbers $\overline{\mathbb{Q}} = \mathbb{R} - \mathbb{Q}
= \{ x \in \mathbb{R} \mid x \notin \mathbb{Q} \}$.
\begin{multicols}{2}
\begin{enumerate}[label=(\roman*)]
\item $\forall x \in \mathbb{Q} ~\forall y \in \mathbb{Q}~ \exists z \in \mathbb{Q} ~( x + y = z)$
\item $\forall x \in \mathbb{Q} ~\forall y \in \mathbb{Q}~ \exists z \in \mathbb{Q} ~( x + z = y)$
\item $\forall x \in \mathbb{Q} ~\forall y \in \mathbb{Q}~ \exists z \in \mathbb{Q} ~( x \cdot y = z)$
\item $\forall x \in \mathbb{Q} ~\forall y \in \mathbb{Q} ~\exists z \in \mathbb{Q} ~( x \cdot z = y)$
\item $\forall x \in \overline{\mathbb{Q}}~ \forall y \in \overline{\mathbb{Q}}~ \exists z \in \overline{\mathbb{Q}} ~( x + y = z)$
\item $\forall x \in \overline{\mathbb{Q}}~ \forall y \in \overline{\mathbb{Q}}~ \exists z \in \overline{\mathbb{Q}}~( x + z = y)$
\item $\forall x \in \overline{\mathbb{Q}} ~\forall y \in \overline{\mathbb{Q}}~ \exists z \in \overline{\mathbb{Q}} ~( x \cdot y = z)$
\item $\forall x \in \overline{\mathbb{Q}} ~\forall y \in \overline{\mathbb{Q}}~ \exists z \in \overline{\mathbb{Q}}~( x \cdot z = y)$
\end{enumerate}
\end{multicols}

\begin{enumerate}
\item Which of the statements above (if any) could be {\bf disproved} using the counterexample 
$x = \frac{1}{2}$, $y= \frac{3}{4}$?
\item Which of the statements above (if any) could be {\bf disproved} using the counterexample 
$x = \sqrt{4}$, $y= \sqrt{3}$?
\item Which of the statements above (if any) could be {\bf disproved} using the counterexample 
$x = 0$, $y= 3$?
\item Which of the statements above (if any) could be {\bf disproved} using the counterexample 
$x = \sqrt{2}$, $y= 0$?
\item Which of the statements above (if any) could be {\bf disproved} using the counterexample 
$x = \sqrt{2}$, $y= -  \sqrt{2}$?
\end{enumerate}

{\it Hint: we proved in class that $\sqrt{2} \notin \mathbb{Q}$. You may also use the facts
that $\sqrt{3} \notin \mathbb{Q}$ and $-\sqrt{2} \notin \mathbb{Q}$.

Bonus - not to hand in: prove these facts; that is, prove that $\sqrt{3} \notin \mathbb{Q}$ and $-\sqrt{2} \notin \mathbb{Q}$. }
