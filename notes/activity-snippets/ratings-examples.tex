%! app: netflix
%! outcome: modeling
%! small-outcomes: function application, tuple notation

In the table  below,  each row represents a user's ratings of movies: 
\cmark~(check) indicates the person liked the movie, \xmark~(x)
that they didn't, and $\bullet$ (dot) that they didn't rate it one way or another (neutral rating or didn't watch).

\begin{center}
\begin{tabular}{c|ccc||c}
Person & Fyre & Frozen II & Picard & Ratings written as a  $3$-tuple\\
\hline
$P_1$     & \xmark & $\bullet$ & \cmark & $(-1, 0, 1)$ \\
$P_2$     & \cmark & \cmark & \xmark & $(1, 1, -1)$ \\
$P_3$     & \cmark & \cmark & \cmark & $(1, 1, 1)$ \\
$P_4$     & $\bullet$ & \xmark & \cmark &  \\
\end{tabular}
\end{center}

Which of $P_1$, $P_2$, $P_3$ has movie preferences most similar to $P_4$?

One approach to answer this question: use {\bf functions} to define distance between user preferences.

\begin{center}
\begin{tabular}{|c|c|}
\hline
\multicolumn{2}{|l|}{
Define the following functions whose inputs are ordered pairs of $3$-tuples each of whose components}\\
\multicolumn{2}{|l|}{
 comes from the set $\{-1,0,1\}$
}
\\
\hline
&\\
$\displaystyle d_{1}(~ (x_1, x_2, x_3) , (y_1, y_2, y_3) ~) =  \sum_{i=1}^3\left( (\abs{x_i-y_i} + 1) \textbf{ div } 2 \right)$
&
$\displaystyle d_{2}(~ (x_1, x_2, x_3) , (y_1, y_2, y_3) ~) =  \sqrt{ \sum_{i=1}^3 (x_i - y_i)^2}$ \\
&\\
\hline
\end{tabular}
\end{center}

\begin{tabularx}{\textwidth}{|X|X|X|}
\hline &&\\
$d_1(P_4, P_1)$ & $d_1(P_4, P_2)$ & $d_1(P_4, P_3)$ \\
&&\\
&&\\
\hline&&\\
$d_2(P_4, P_1)$ & $d_2(P_4, P_2)$ & $d_2(P_4, P_3)$ \\
&&\\
&&\\
\hline
\end{tabularx}

\vfill

{\it Extra example:} A new movie is released, and $P_1$ and $P_2$ watch it before $P_3$, and give it
ratings; $P_1$ gives \cmark~and $P_2$ gives \xmark.
Should this movie be recommended to $P_3$? Why or why not?

{\it Extra example:} Define the new functions that would be used to compare the $4$-tuples of ratings encoding
movie preferences now that there are four movies in the database.
