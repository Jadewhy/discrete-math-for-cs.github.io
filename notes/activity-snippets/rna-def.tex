%! app: rna
%! outcome: recursive
%! small-outcomes: set-roster,set-builder

RNA is made up of strands of four different bases that match up in specific ways.\\
The bases are elements of the set 
$B  = \{\A, \C, \G, \U \}$.


{\bf Definition} The set of RNA strands $S$ is defined (recursively) by:
\[
\begin{array}{ll}
\textrm{Basis Step: } & \A \in S, \C \in S, \U \in S, \G \in S \\
\textrm{Recursive Step: } & \textrm{If } s \in S\textrm{ and }b \in B \textrm{, then }sb \in S
\end{array}
\]
where $sb$ is string concatenation.

Examples: 

\vfill

\fbox{\parbox{\textwidth}{%
To define a set we can use the {\bf roster method}, the {\bf set builder notation}, and also \ldots


{\bf New! Recursive Definitions of Sets}: The set $S$ (pick a name) is defined by:
\[
\begin{array}{ll}
\textrm{Basis Step: } & \textrm{Specify finitely many elements of } S\\
\textrm{Recursive Step: } & \textrm{Give a rule for creating a new element of } S \textrm{ from known values existing in } S, \\
& \textrm{and potentially other values}. \\
\end{array}
\]
The set $S$ then consists of all and only elements that are put in $S$ by finitely many (a nonnegative integer number) of
applications of the recursive step after the basis step.
}}