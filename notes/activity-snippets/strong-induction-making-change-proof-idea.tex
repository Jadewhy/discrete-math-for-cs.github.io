%! app: TODOapp
%! outcome: TODOoutcome

Suppose we had postage stamps worth $5$ cents and $3$ cents.
Which number of cents can we form using these stamps?
In other words, which postage can we pay?

\vspace{20pt}

\begin{align*}
    &CanPay(0) \land \lnot CanPay(1) \land \lnot CanPay(2) \land \\
    &CanPay(3) \land \lnot CanPay(4) \land CanPay(5) \land CanPay(6) \\
    &\lnot CanPay(7) \land \forall n \in \mathbb{Z}^{\geq 8} CanPay(n)
\end{align*}

How is the predicate $CanPay$ formalized? 
$\phantom{\exists x \in \mathbb{N} \exists y \in \mathbb{N}  ( 5x+3y = n)}$


{\bf Proof} (idea): First, explicitly give witnesses or general arguments
for postages between $0$ and $7$. 
To prove the universal claim, we can use mathematical induction or strong induction.

{\it Approach 1, mathematical induction}: if we have
stamps that add up to $n$ cents, need to use them (and others)
to give $n+$ cents. How do we get $1$ cent with just $3$ cent
$5$ cent stamps?

Either take away a $5$-cent stamps and add two $3$-cent stamps,
or take away three $3$-cent stamps and add two $5$-cent stamps.

The details of this proof by mathematical induction
are making sure we have enough 
stamps to use one of these approaches.

{\it Approach 2, strong induction}: assuming we know how to make postage
for {\bf all} smaller values (greater than or equal to $8$), when
we need to make $n+1$ cents, add one $3$ cent stamp to 
however we make $(n+1) - 3$ cents.

The details of this proof by strong induction are making sure we 
stay in the domain of the universal when applying the induction hypothesis.
