{\it Application}: design a circuit given a desired input-output relationship.

\begin{center}
\begin{tabular}{cc||cc}
\multicolumn{2}{c||}{Input}  &\multicolumn{2}{c}{Output}\\
$p$ & $q$& $mystery_1$ & $mystery_2$\\
\hline
$T$ & $T$  & $T$ & $F$\\
$T$ & $F$  & $T$ & $F$\\
$F$ & $T$  & $F$ & $F$\\
$F$ & $F$  & $T$ & $T$\\
\end{tabular}
\qquad \qquad
\begin{tabular}{ccc||c}
\multicolumn{3}{c||}{Input}  & Output\\
$p$ & $q$ & $r$  &  ?\\
\hline
$T$ & $T$  & $T$ & $T$ \\
$T$ & $T$  & $F$ & $T$ \\
$T$ & $F$  & $T$ & $F$ \\
$T$ & $F$  & $F$ & $T$ \\
$F$ & $T$  & $T$ & $F$ \\
$F$ & $T$  & $F$ & $F$ \\
$F$ & $F$  & $T$ & $T$ \\
$F$ & $F$  & $F$ & $F$ \\
\end{tabular}

\end{center}


A compound proposition that  gives output $mystery_1$ is: \underline{\phantom{\hspace{3in}}}


\vfill


A compound proposition that  gives output $mystery_2$ is: \underline{\phantom{\hspace{3in}}}


\vfill