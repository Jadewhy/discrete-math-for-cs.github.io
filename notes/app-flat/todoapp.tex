\documentclass[12pt, oneside]{article}

\usepackage[letterpaper, scale=0.89, centering]{geometry}
\usepackage{fancyhdr}
\setlength{\parindent}{0em}
\setlength{\parskip}{1em}

\pagestyle{fancy}
\fancyhf{}
\renewcommand{\headrulewidth}{0pt}
\rfoot{\href{https://creativecommons.org/licenses/by-nc-sa/2.0/}{CC BY-NC-SA 2.0} Version \today~(\thepage)}

\usepackage{amssymb,amsmath,pifont,amsfonts,comment,enumerate,enumitem}
\usepackage{currfile,xstring,hyperref,tabularx,graphicx,wasysym}
\usepackage[labelformat=empty]{caption}
\usepackage[dvipsnames,table]{xcolor}
\usepackage{multicol,multirow,array,listings,tabularx,lastpage,textcomp,booktabs}

\lstnewenvironment{algorithm}[1][] {   
    \lstset{ mathescape=true,
        frame=tB,
        numbers=left, 
        numberstyle=\tiny,
        basicstyle=\rmfamily\scriptsize, 
        keywordstyle=\color{black}\bfseries,
        keywords={,procedure, div, for, to, input, output, return, datatype, function, in, if, else, foreach, while, begin, end, }
        numbers=left,
        xleftmargin=.04\textwidth,
        #1
    }
}
{}
\lstnewenvironment{java}[1][]
{   
    \lstset{
        language=java,
        mathescape=true,
        frame=tB,
        numbers=left, 
        numberstyle=\tiny,
        basicstyle=\ttfamily\scriptsize, 
        keywordstyle=\color{black}\bfseries,
        keywords={, int, double, for, return, if, else, while, }
        numbers=left,
        xleftmargin=.04\textwidth,
        #1
    }
}
{}

\newcommand\abs[1]{\lvert~#1~\rvert}
\newcommand{\st}{\mid}

\newcommand{\A}[0]{\texttt{A}}
\newcommand{\C}[0]{\texttt{C}}
\newcommand{\G}[0]{\texttt{G}}
\newcommand{\U}[0]{\texttt{U}}

\newcommand{\cmark}{\ding{51}}
\newcommand{\xmark}{\ding{55}}

 
\begin{document}
\begin{flushright}
    \StrBefore{\currfilename}{.}
\end{flushright} \section*{Netflix intro}


What data should we encode about each Netflix account holder to help us make effective recommendations?

\vfill
\vfill

In machine learning, clustering can be used to group similar data for prediction and recommendation.  For example,
each Netflix user's viewing history can be represented as a $n$-tuple indicating their preferences about
movies in the database, where $n$ is the number of movies in the database.  People with similar tastes in movies can then be clustered to provide recommendations
of movies for one another.  Mathematically, clustering is based on a notion of distance between pairs of $n$-tuples.
 \vfill
\section*{Data types}


\begin{center}
    \begin{tabular}{p{4.6in}p{2.6in}}
    {\bf  Term} & {\bf Examples}:\\
    &  (add additional examples from class)\\
    \hline 
    {\bf set} \newline
    unordered collection of elements & $7 \in \{43, 7, 9 \}$ \qquad $2 \notin \{43, 7, 9 \}$ \\
    {\it repetition doesn't matter} & \\
    {\it Equal sets agree on membership of all elements}& \\
    \hline
    {\bf $n$-tuple} \newline
    ordered sequence of elements with $n$ ``slots" ($n >0$) & \\
    {\it repetition matters, fixed length} &\\
    {\it Equal $n$-tuples have corresponding components equal}& \\
    \hline
    {\bf string} \newline
    ordered finite sequence of elements each from specified
    set & \\
    {\it repetition matters, arbitrary finite length} &\\
    {\it Equal strings have same length and corresponding characters equal}
    \end{tabular}
\end{center}

{\it Special cases}: 

When $n=2$, the 2-tuple is called an {\bf ordered pair}.

A string of length $0$ is called the {\bf empty string} and is denoted $\lambda$.

A set with no elements is called the {\bf empty set} and is denoted $\{\}$ or $\emptyset$. \vfill
\section*{Set operations}


\fbox{\parbox{\textwidth}{To define a set we can use the roster method, set builder notation, a recursive definition, 
and also we can apply a set operation to other sets. \\

{\bf New! Cartesian product of sets} and {\bf set-wise concatenation of sets of strings}\\


{\bf Definition}: Let $X$ and $Y$ be sets.  The {\bf Cartesian product} of $X$ and $Y$, denoted
$X \times Y$, is the set of all ordered pairs $(x,y)$ where $x \in X$ and $y \in Y$
\[
X \times Y = \{ (x,y) \mid x \in X \text{ and } y \in Y \}
\]
{\bf Definition}: Let $X$ and $Y$ be sets of strings over the same alphabet. The {\bf set-wise concatenation} 
of $X$ and $Y$, denoted $X \circ Y$, is the set of all results of string concatenation $xy$ where $x \in X$ 
and $y \in Y$
\[
X \circ Y = \{ xy \mid x \in X \text{ and } y \in Y \}
\]
}}

{\bf Pro-tip}: the meaning of writing one element next to another like $xy$ depends on the data-types of $x$ and 
$y$. When $x$ and $y$ are strings, the convention is that $xy$ is the result of string concatenation. 
When $x$ and $y$ are numbers, the convention is that $xy$ is the result of multiplication. This is 
(one of the many reasons) why is it very important to declare the data-type of variables before we use them.

{\it Fill in the missing entries in the table}:

\begin{center}
\begin{tabular}{cc}
{\bf  Set} & {\bf Example elements in this set}:\\
\hline 
& \\
$B$ &\A \qquad \C \qquad \G \qquad \U \\
& \\
\hline
& \\
\phantom{$B \times B$} & $(\A, \C)$ \qquad $(\U, \U)$\\
& \\
\hline
& \\
$B \times \{-1,0,1\}$ & \\
& \\
\hline
& \\
$\{-1,0,1\} \times B$ & \\
& \\
\hline
& \\
\phantom{$\{-1,0,1\} \times \{-1,0,1\}  \times \{-1,0,1\} $} & \qquad $(0,0,0)$ \\
& \\
\hline
& \\
$ \{\A, \C, \G, \U \} \circ  \{\A, \C, \G, \U \}$& \\
& \\
\hline
& \\
\phantom{$\{G\} \circ \{G\} \circ \{G\}$} & \qquad $\G\G\G\G$ \\
& \\
\hline

\end{tabular}
\end{center}

\vfill \vfill
\section*{Defining functions}


\fbox{\parbox{\textwidth}{{\bf New! Defining functions} A function is defined by its (1) domain, 
(2) codomain, and (3) rule assigning each 
element in the domain exactly one element in the codomain.\\

The domain and codomain are nonempty sets.

The rule can be depicted as a table, formula, or English description.

The notation is 
\begin{center}
    ``Let the function FUNCTION-NAME: DOMAIN $\to$ CODOMAIN be given by \\
FUNCTION-NAME(x) = \ldots for every $x \in DOMAIN$''.
\end{center}

or 
\begin{center}
    ``Consider the function FUNCTION-NAME: DOMAIN $\to$ CODOMAIN given by \\
FUNCTION-NAME(x) = \ldots for every $x \in DOMAIN$''.
\end{center}
}}

\vfill

Example: The absolute value function 

{\bf Domain}

{\bf Codomain}

{\bf Rule}

\vfill 
 \vfill
\section*{Defining functions recursively}


When the domain of a function is a {\it recursively defined set}, the rule assigning 
images to domain elements (outputs) can also be defined recursively.

Recall: The set of RNA strands $S$ is defined (recursively) by:
\[
\begin{array}{ll}
\textrm{Basis Step: } & \A \in S, \C \in S, \U \in S, \G \in S \\
\textrm{Recursive Step: } & \textrm{If } s \in S\textrm{ and }b \in B \textrm{, then }sb \in S
\end{array}
\]
where $sb$ is string concatenation.

{\bf Definition} (Of a function, recursively) A function \textit{rnalen} that computes the length of RNA strands in $S$ is defined by:
\[
\begin{array}{llll}
& & \textit{rnalen} : S & \to \mathbb{Z}^+ \\
\textrm{Basis Step:} & \textrm{If } b \in B\textrm{ then } & \textit{rnalen}(b) & = 1 \\
\textrm{Recursive Step:} & \textrm{If } s \in S\textrm{ and }b \in B\textrm{, then  } & \textit{rnalen}(sb) & = 1 + \textit{rnalen}(s)
\end{array}
\]

The domain of \textit{rnalen} is \phantom{$S$}\\

The codomain of \textit{rnalen} is \phantom{$\mathbb{Z}^+$}\\

Example function application:
\[
rnalen(\A\C\U) = \phantom{1+ rnalen(\A\C) = 1 + (1 + rnalen(\A) ) = 1 + ( 1 + 1) = 3}
\]

\vfill

{\it Extra example}: A function \textit{basecount} that computes the number of a given base $b$ appearing in a RNA strand $s$ is defined recursively:  {\it fill in codomain and sample function
applications}
\[
\begin{array}{llll}
& & \textit{basecount} : S \times B & \to \phantom{\mathbb{N}} \\
\textrm{Basis Step:} &  \textrm{If } b_1 \in B, b_2 \in B & \textit{basecount}(b_1, b_2) & =
        \begin{cases}
            1 & \textrm{when } b_1 = b_2 \\
            0 & \textrm{when } b_1 \neq b_2 \\
        \end{cases} \\
\textrm{Recursive Step:} & \textrm{If } s \in S, b_1 \in B, b_2 \in B &\textit{basecount}(s b_1, b_2) & =
        \begin{cases}
            1 + \textit{basecount}(s, b_2) & \textrm{when } b_1 = b_2 \\
            \textit{basecount}(s, b_2) & \textrm{when } b_1 \neq b_2 \\
        \end{cases}
\end{array}
\]

$basecount(\A\C\U,\A) = \hspace{5in}$\\


$basecount(\A\C\U,\G) = \hspace{5in}$\\


\vfill
{\it Extra example}: The function which outputs $2^n$ when given a nonnegative integer $n$ can be defined recursively, 
because its domain is the set of nonnegative integers.

\vfill
 \vfill
\section*{Why represent numbers}


Modeling uses data-types that are 
encoded in a computer.

The details of the encoding impact the efficiency of algorithms
we use to understand the systems we are modeling and the 
impacts of these algorithms on the people using the systems.

Case study: how to encode numbers?

\phantom{
Positional representation with familiar (decimal) number encodings
\vspace{30pt}
}
\vfill \vfill
\section*{Base expansion definition}


{\bf Definition} For $b$ an integer greater than $1$ and $n$ a positive integer, 
the {\bf base $b$ expansion of $n$}  is
\[
(a_{k-1} \cdots a_1 a_0)_b
\]
where $k$ is a positive integer, $a_0, a_1, \ldots, a_{k-1}$ 
are nonnegative integers less than $b$, $a_{k-1} \neq  0$, and
\[
n =  a_{k-1} b^{k-1} + \cdots + a_1b + a_0
\]

Notice: {\it The base $b$ expansion of a positive integer $n$ is a string over the alphabet 
$\{x \in \mathbb{N} \st x < b\}$
whose leftmost character is nonzero.}

\begin{center}
\begin{tabular}{|c|c|}
\hline
Base $b$ & Collection of possible coefficients in base $b$ expansion of  a positive integer \\
\hline
& \\
Binary ($b=2$) & $\{0,1\}$ \\
\hline
& \\
Ternary ($b=3$) & $\{0,1, 2\}$ \\
\hline
& \\
Octal ($b=8$) & $\{0,1, 2, 3, 4, 5, 6, 7\}$\\
\hline
& \\
Decimal ($b=10$) & $\{0,1, 2, 3, 4, 5, 6, 7, 8, 9\}$\\
\hline
& \\
Hexadecimal ($b=16$) &  $\{0,1, 2, 3, 4, 5, 6, 7, 8, 9, A, B, C, D, E, F\}$\\
& letter coefficient symbols represent numerical values $(A)_{16} = (10)_{10}$\\
&$(B)_{16} = (11)_{10} ~~(C)_{16} = (12)_{10} ~~
 (D)_{16} = (13)_{10} ~~ (E)_{16} = (14)_{10} ~~ (F)_{16} = (15)_{10} $\\
\hline
\end{tabular}
\end{center}

\vfill
 \vfill
\section*{Base expansion examples}


\fbox{\parbox{\textwidth}{{\bf Common bases}: \hfill Binary  $b=2$ \qquad Octal $b=8$ \qquad Decimal $b=10$ \qquad Hexadecimal $b=16$
\hfill }}

{\it Examples}:

$(1401)_{2}$

\vfill

$(1401)_{10}$

\vfill
\vfill
\vfill


$(1401)_{16}$

\vfill
\vfill
\vfill
 \vfill
\section*{Algorithm definition}


\fbox{\parbox{\textwidth}{{\bf New!} An algorithm is a finite sequence of precise instructions for solving a problem.
\hfill
}} \vfill
\section*{Algorithm half}


\begin{algorithm}[caption={Algorithm for calculating integer part of half the input}]
    procedure $\textit{half}$($n$: a positive integer)
    $r$ := $0$
    while $n$ > $1$
      $r$ := $r + 1$
      $n$ := $n - 2$
    return $r$ $\{ r~\textrm{holds the result of the operation}\} $
    \end{algorithm}

 \begin{multicols}{2}
  \begin{center} 
    \begin{tabular}{c|c|c}
    $n$ & $r$  & $n > 1$?\\
    \hline 
    ~$6$~ & \phantom{~$0$~} & \phantom{~T~}\\
    \phantom{$4$} & \phantom{$1$} & \phantom{T}\\
    \phantom{$2$} & \phantom{$2$} & \phantom{T}\\
    \phantom{$0$} & \phantom{$3$} & \phantom{F}\\
    &\\
    \end{tabular}
    \end{center}
    \begin{center}
      \begin{tabular}{c|c|c}
      $n$ & $r$  & $n > 1$?\\
      \hline 
      ~$5$~ & \phantom{~$0$~} & \phantom{~T~}\\
      \phantom{$3$} & \phantom{$1$} & \phantom{T}\\
      \phantom{$1$} & \phantom{$2$} & \phantom{F}\\
      &\\
      \end{tabular}
      \end{center}    
\end{multicols}

\vfill \vfill
\section*{Algorithm log}


 \begin{algorithm}[caption={Algorithm for calculating integer part of $\log$}]
    procedure $\textit{log}$($n$: a positive integer)
    $r$ := $0$
    while $n$ > $1$
      $r$ := $r + 1$
      $n$ := $half(n)$
    return $r$ $\{ r~\textrm{holds the result of the}~\log~\textrm{operation}\} $
\end{algorithm}

\begin{multicols}{2}
  \begin{center}
    \begin{tabular}{c|c|c}
    $n$ & $r$  & $n > 1$?\\
    \hline 
    ~$8$~ & \phantom{~$0$~} & \phantom{~T~}\\
    \phantom{$4$} & \phantom{$1$} & \phantom{T}\\
    \phantom{$2$} & \phantom{$2$} & \phantom{T}\\
    \phantom{$1$} & \phantom{$3$} & \phantom{F}\\
    &\\
    \end{tabular}
    \end{center}
  \begin{center}
    \begin{tabular}{c|c|c}
    $n$ & $r$  & $n > 1$?\\
    \hline 
    ~$6$~ & \phantom{~$0$~} & \phantom{~T~}\\
    \phantom{$3$} & \phantom{$1$} & \phantom{T}\\
    \phantom{$1$} & \phantom{$2$} & \phantom{F}\\
    &\\
    \end{tabular}
    \end{center}
\end{multicols}

\vfill

\fbox{\parbox{\textwidth}{
$2^0 = 1$~~\hfill $2^1=2$~~\hfill $2^2=4$~~\hfill $2^3=8$~~
\hfill $2^4=16$~~\hfill $2^5=32$~~
\hfill $2^6=64$~~\hfill $2^7=128$~~
\hfill $2^8=256$~~\hfill $2^9=512$~~
\hfill $2^{10}=1024$}} \vfill
\section*{Division algorithm}


{\bf Integer division and remainders} (aka The Division Algorithm) Let $n$ be an integer 
and $d$ a positive integer. There are unique integers $q$ and $r$, with $0 \leq r < d$, such that 
$n = dq + r$. In this case, $d$ is called the divisor, $n$ is called the dividend, 
$q$ is called the quotient, 
and $r$ is called the remainder. We write $q=n \textbf{ div } d$ and $r=n \textbf{ mod } d$.

\textit{Extra example}: How do $\textbf{ div }$ and $\textbf{ mod }$ compare to $/$ and $\%$ in Java and python?

\vfill
 \vfill
\section*{Base expansion algorithms}


{\bf Two algorithms for constructing base $b$ expansion from decimal representation}

{\bf Most significant first}: Start with left-most coefficient of expansion
\begin{multicols}{2}
\begin{algorithm}[caption={Calculating integer part of $\log_b$}]
procedure $\textit{logb}$($n, b$: positive integers with $b > 1$)
$r$ := $0$
while $n$ > $1$
  $r$ := $r + 1$
  $n$ := $n$ div $b$
return $r$ $\{ r~\textrm{holds the result of the}~\log_b~\textrm{operation}\}$
\end{algorithm}
\columnbreak
\begin{algorithm}[caption={Calculating base $b$ expansion, from left}]
procedure $\textit{baseb1}$($n, b$: positive integers with $b > 1$)
$v$ := $n$
$k$ := $logb(n,b) + 1$
for $i$ := $1$ to $k$
  $a_{k-i}$ := $0$
  while $v \geq b^{k-i}$
    $a_{k-i}$ := $a_{k-i} + 1$
    $v$ := $v -  b^{k-i}$
return $(a_{k-1}, \ldots, a_0) \{(a_{k-1} \ldots a_0)_b~\textrm{ is the base } b \textrm{ expansion of } n \}$
\end{algorithm}
\end{multicols}

\vfill
\vfill

{\bf Least significant first}: Start with right-most coefficient of expansion

\begin{multicols}{2}
  \begin{minipage}{3.2in}
    Idea: {\tiny(when $k > 1$)} 
    \begin{align*}
      n &= a_{k-1} b^{k-1} + \cdots + a_1 b + a_0 \\
        &= b ( a_{k-1} b^{k-2} + \cdots + a_1) + a_0\end{align*}
    so $a_0 = n \textbf{ mod } b$ and $a_{k-1} b^{k-2} + \cdots + a_1 = n \textbf{ div } b$.

\end{minipage}
\columnbreak
\begin{algorithm}[caption={Calculating base $b$ expansion, from right}]
procedure $\textit{baseb2}$($n, b$: positive integers with $b > 1$)
$q$ := $n$
$k$ := $0$
while $q  \neq 0$
  $a_{k}$ := $q$ mod $b$
  $q$ := $q$ div $b$
  $k$ := $k+1$
return $(a_{k-1}, \ldots, a_0) \{(a_{k-1} \ldots a_0)_b~\textrm{ is the base } b \textrm{ expansion of } n \}$
\end{algorithm}
\end{multicols}

\vfill
\vfill \vfill
\end{document}