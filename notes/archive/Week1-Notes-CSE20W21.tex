\documentclass[12pt, oneside]{article}

\usepackage{amssymb,amsmath,pifont,amsfonts,comment,enumerate,enumitem}
\usepackage{currfile,xstring,hyperref,tabularx,graphicx,wasysym}
\usepackage[labelformat=empty]{caption}
\usepackage[dvipsnames,table]{xcolor}
\usepackage{multicol,multirow,array,listings,tabularx,lastpage,textcomp,booktabs}

% NOTE(joe): This environment is credit @pnpo (https://tex.stackexchange.com/a/218450)
\lstnewenvironment{algorithm}[1][] %defines the algorithm listing environment
{   
    \lstset{ %this is the stype
        mathescape=true,
        frame=tB,
        numbers=left, 
        numberstyle=\tiny,
        basicstyle=\rmfamily\scriptsize, 
        keywordstyle=\color{black}\bfseries,
        keywords={,procedure, div, for, to, input, output, return, datatype, function, in, if, else, foreach, while, begin, end, }
        numbers=left,
        xleftmargin=.04\textwidth,
        #1
    }
}
{}
\lstnewenvironment{java}[1][]
{   
    \lstset{
        language=java,
        mathescape=true,
        frame=tB,
        numbers=left, 
        numberstyle=\tiny,
        basicstyle=\ttfamily\scriptsize, 
        keywordstyle=\color{black}\bfseries,
        keywords={, int, double, for, return, if, else, while, }
        numbers=left,
        xleftmargin=.04\textwidth,
        #1
    }
}
{}

\newcommand\abs[1]{\lvert~#1~\rvert}
\newcommand{\st}{\mid}

\newcommand{\A}[0]{\texttt{A}}
\newcommand{\C}[0]{\texttt{C}}
\newcommand{\G}[0]{\texttt{G}}
\newcommand{\U}[0]{\texttt{U}}

\newcommand{\cmark}{\ding{51}}
\newcommand{\xmark}{\ding{55}}


\begin{document}
\begin{flushright}
\StrBefore{\currfilename}{.}
\end{flushright}

\newpage
\section*{Themes for CSE 20}
\begin{itemize}
\item Technical skepticism
\item Multiple representations
\end{itemize}

\section*{Recurring examples in CSE 20}
\begin{itemize}
\item Clustering and recommendation systems (machine learning, Netflix)
\item Genomics and bioinformatics (DNA and RNA)
\item Codes and information (secret message sharing and error correction)
\item ``Under the hood" of computers (circuits, pixel color representation, data structures)
\end{itemize}

\section*{This week's highlights}
\begin{itemize}
\item Use and apply definitions and notation
\item Explore mathematical definitions related to a specific application (Netflix)
\item Define data types: set, $n$-tuple, string (over specific alphabet)
\item Define sets and functions in multiple ways
\item Trace an algorithm specified in pseudocode
\item Define the base expansion of a positive integer, specifically decimal, binary, hexadecimal, and octal.
\item Convert between expansions in different bases of a positive integer.
\item Define and use the div and mod operators.
\end{itemize}

% \section*{Lecture videos}
% Week 1 Day 1
% \href{https://youtube.com/playlist?list=PLML4QilACLk7gzYrukE78l8mPniy1ieP2}{YouTube playlist}\footnote{
% \href{https://youtube.com/playlist?list=PLML4QilACLk7gzYrukE78l8mPniy1ieP2}{https://youtube.com/playlist?list=PLML4QilACLk7gzYrukE78l8mPniy1ieP2}}

% Week 1 Day 2
% \href{https://youtube.com/playlist?list=PLML4QilACLk5UEuC7vC3KVh4-c-4omZN8}{YouTube playlist}\footnote{
% \href{https://youtube.com/playlist?list=PLML4QilACLk5UEuC7vC3KVh4-c-4omZN8}{https://youtube.com/playlist?list=PLML4QilACLk5UEuC7vC3KVh4-c-4omZN8}}

% Week 1 Day 3
% \href{https://youtube.com/playlist?list=PLML4QilACLk6J5h3Jg-m71pMXWKvPwXDI}{YouTube playlist}\footnote{
% \href{https://youtube.com/playlist?list=PLML4QilACLk6J5h3Jg-m71pMXWKvPwXDI}{https://youtube.com/playlist?list=PLML4QilACLk6J5h3Jg-m71pMXWKvPwXDI}}

\newpage
\section*{Monday January 4}

What data should we encode about each Netflix account holder to help us make effective recommendations?

\vfill

In machine learning, clustering can be used to group similar data for prediction and recommendation.  For example,
each Netflix user's viewing history can be represented as a $n$-tuple indicating their preferences about
movies in the database, where $n$ is the number of movies in the database.  People with similar tastes in movies can then be clustered to provide recommendations
of movies for one another.  Mathematically, clustering is based on a notion of distance between pairs of $n$-tuples.

\vfill


In the table  below,  each row represents a user's ratings of movies: 
\cmark~(check) indicates the person liked the movie, \xmark~(x)
that they didn't, and $\bullet$ (dot) that they didn't rate it one way or another (neutral rating or didn't watch).

\begin{center}
\begin{tabular}{c|ccc||c}
Person & Fyre & Frozen II & Picard & Ratings written as a  $3$-tuple\\
\hline
$P_1$     & \xmark & $\bullet$ & \cmark & $(-1, 0, 1)$ \\
$P_2$     & \cmark & \cmark & \xmark & $(1, 1, -1)$ \\
$P_3$     & \cmark & \cmark & \cmark & $(1, 1, 1)$ \\
$P_4$     & $\bullet$ & \xmark & \cmark &  \\
\end{tabular}
\end{center}

Which of $P_1$, $P_2$, $P_3$ has movie preferences most similar to $P_4$?

One approach to answer this question: use {\bf functions} to define distance between user preferences.

\begin{center}
\begin{tabular}{|c|c|}
\hline
\multicolumn{2}{|l|}{
Define the following functions whose inputs are ordered pairs of $3$-tuples each of whose components}\\
\multicolumn{2}{|l|}{
 comes from the set $\{-1,0,1\}$
}
\\
\hline
&\\
$\displaystyle d_{1}(~ (x_1, x_2, x_3) , (y_1, y_2, y_3) ~) =  \sum_{i=1}^3\left( (\abs{x_i-y_i} + 1) \textbf{ div } 2 \right)$
&
$\displaystyle d_{2}(~ (x_1, x_2, x_3) , (y_1, y_2, y_3) ~) =  \sqrt{ \sum_{i=1}^3 (x_i - y_i)^2}$ \\
&\\
\hline
\end{tabular}
\end{center}

\begin{tabularx}{\textwidth}{|X|X|X|}
\hline &&\\
$d_1(P_4, P_1)$ & $d_1(P_4, P_2)$ & $d_1(P_4, P_3)$ \\
&&\\
&&\\
\hline&&\\
$d_2(P_4, P_1)$ & $d_2(P_4, P_2)$ & $d_2(P_4, P_3)$ \\
&&\\
&&\\
\hline
\end{tabularx}

\vfill


{\it Extra example:} A new movie is released, and $P_1$ and $P_2$ watch it before $P_3$, and give it
ratings; $P_1$ gives \cmark~and $P_2$ gives \xmark.
Should this movie be recommended to $P_3$? Why or why not?

{\it Extra example:} Define the new functions that would be used to compare the $4$-tuples of ratings encoding
movie preferences now that there are four movies in the database.


\section*{Wednesday January 6}

\begin{center}
\begin{tabular}{p{4.4in}p{2.8in}}
{\bf  Term} & {\bf Examples}:\\
&  (add additional examples from class)\\
\hline 
{\bf set} \newline
unordered collection of elements & $7 \in \{43, 7, 9 \}$ \qquad $2 \notin \{43, 7, 9 \}$ \\
{\it Equal means agree on membership of all elements}& \\
\hline
{\bf $n$-tuple} \newline
ordered sequence of elements with $n$ ``slots" & \\
{\it Equal means corresponding components equal}& \\
\hline
{\bf string} \newline
ordered finite sequence of elements each from specified
set & \\
{\it Equal means same length and corresponding characters equal}
\end{tabular}
\end{center}
\[
\{ -1, 1\} \qquad 
\{0, 0 \} \qquad
\{-1, 0, 1 \} \qquad
\mathbb{Z} \qquad
\mathbb{N} = \{ x \in \mathbb{Z} \mid x \geq 0 \} \qquad
\emptyset \qquad
\mathbb{Z}^+ = \{ x \in \mathbb{Z}  \mid x > 0 \}
\]

\vfill

{\it Which of the sets above are defined using the roster method? Which are defined using set builder notation?}

{\it Which of the sets above have $0$ as an element?}

{\it Can you write any of the sets above more simply?}

\vfill

RNA is made up of strands of four different bases that match up in specific ways.\\
The bases are elements of the set 
$B  = \{\A, \C, \G, \U \}$.


{\bf Definition} The set of RNA strands $S$ is defined (recursively) by:
\[
\begin{array}{ll}
\textrm{Basis Step: } & \A \in S, \C \in S, \U \in S, \G \in S \\
\textrm{Recursive Step: } & \textrm{If } s \in S\textrm{ and }b \in B \textrm{, then }sb \in S
\end{array}
\]
where $sb$ is string concatenation.

Examples: 


\vfill


\fbox{\parbox{\textwidth}{%
To define a set we can use the {\bf roster method}, the {\bf set builder notation}, and also \ldots


{\bf New! Recursive Definitions of Sets}: The set $S$ (pick a name) is defined by:
\[
\begin{array}{ll}
\textrm{Basis Step: } & \textrm{Specify finitely many elements of } S\\
\textrm{Recursive Step: } & \textrm{Give a rule for creating a new element of } S \textrm{ from known values existing in } S, \\
& \textrm{and potentially other values}. \\
\end{array}
\]
The set $S$ then consists of all and only elements that are put in $S$ by finitely many (a nonnegative integer number) of
applications of the recursive step after the basis step.
}}


\newpage
{\it Extra example}:  The set of binary strings, denoted $\{0,1\}^*$ which we read as ``zero one star", 
is defined (recursively) by:
\[
\begin{array}{ll}
\textrm{Basis Step: } & \lambda \in \{0,1\}^* \\
\textrm{Recursive Step: } & \textrm{If } s \in \{0,1\}^*\textrm{ then } s0 \in  \{0,1\}^* \textrm{ and } s1 \in  \{0,1\}^*
\end{array}
\]
where $s0$ and $s1$ are the results of string concatenation. The symbol $\lambda$, pronounced ``lambda" is used to denote the empty string
and has the property that $\lambda x = x \lambda = x$ for each string $x$.

Examples: 



\vfill 

\fbox{\parbox{\textwidth}{%
To define a set we can use the roster method, the set builder notation, a recursive definition, 
and also we can apply a set operation to other sets. \\

{\bf New! Cartesian product of sets} and {\bf set-wise concatenation of sets of strings}
}}


{\bf Definition} (Rosen p.\ 123) Let $A$ and $B$ be sets.  The {\bf Cartesian product} of $A$ and $B$, denoted
$A \times B$, is the set of all ordered pairs $(a,b)$ where $a \in A$ and $b \in B$
\[
A \times B = \{ (a,b) \mid a \in A \text{ and } b \in B \}
\]
{\bf Definition}: Let $A$ and $B$ be sets of strings over the same alphabet. The {\bf set-wise concatenation} 
of $A$ and $B$, denoted $A \circ B$, is the set of all results of string concatenation $ab$ where $a \in A$ and $b \in B$
\[
A \circ B = \{ ab \mid a \in A \text{ and } b \in B \}
\]



\begin{center}
\begin{tabular}{cc}
{\bf  Set} & {\bf Example elements in this set}:\\
\hline 
$B$ &\A \qquad \C \qquad \G \qquad \U \\
& \\
\hline
\phantom{$B \times B$} & $(\A, \C)$ \qquad $(\U, \U)$\\
& \\
\hline
$B \times \{-1,0,1\}$ & \\
& \\
\hline
$\{-1,0,1\} \times B$ & \\
& \\
\hline
\phantom{$\{-1,0,1\} \times \{-1,0,1\}  \times \{-1,0,1\} $} & \qquad $(0,0,0)$ \\
& \\
\hline
$ \{\A, \C, \G, \U \} \circ  \{\A, \C, \G, \U \}$& \\
& \\
\hline
\phantom{$\{G\} \circ \{G\} \circ \{G\}$} & \qquad $\G\G\G\G$ \\
& \\
\hline

\end{tabular}
\end{center}

\vfill
\vfill

\fbox{\parbox{\textwidth}{%
{\bf New! Defining functions} A function is defined by its (1) domain, (2) codomain, and (3) rule assigning each 
element in the domain exactly one element in the codomain.\\

The domain and codomain are nonempty sets.

The rule can be depicted as a table, formula, English description, etc.
}}


Examples: 



\vfill 


{\bf Definition} (Of a function, recursively) A function \textit{rnalen} that computes the length of RNA strands in $S$ is defined by:
\[
\begin{array}{llll}
& & \textit{rnalen} : S & \to \mathbb{Z}^+ \\
\textrm{Basis Step:} & \textrm{If } b \in B\textrm{ then } & \textit{rnalen}(b) & = 1 \\
\textrm{Recursive Step:} & \textrm{If } s \in S\textrm{ and }b \in B\textrm{, then  } & \textit{rnalen}(sb) & = 1 + \textit{rnalen}(s)
\end{array}
\]

The domain of \textit{rnalen} is \underline{\phantom{$S$\hspace{1.5in}}}.
The codomain of \textit{rnalen} is \underline{\phantom{$\mathbb{Z}^+$\hspace{1.5in}}}.
\[
rnalen(\A\C\U) = \underline{\phantom{\hspace{5in}}}
\]

\vfill

{\it Extra example}: A function \textit{basecount} that computes the number of a given base $b$ appearing in a RNA strand $s$ is defined recursively:  {\it fill in codomain and sample function
applications}
\[
\begin{array}{llll}
& & \textit{basecount} : S \times B & \to \phantom{\mathbb{N}} \\
\textrm{Basis Step:} &  \textrm{If } b_1 \in B, b_2 \in B & \textit{basecount}(b_1, b_2) & =
        \begin{cases}
            1 & \textrm{when } b_1 = b_2 \\
            0 & \textrm{when } b_1 \neq b_2 \\
        \end{cases} \\
\textrm{Recursive Step:} & \textrm{If } s \in S, b_1 \in B, b_2 \in B &\textit{basecount}(s b_1, b_2) & =
        \begin{cases}
            1 + \textit{basecount}(s, b_2) & \textrm{when } b_1 = b_2 \\
            \textit{basecount}(s, b_2) & \textrm{when } b_1 \neq b_2 \\
        \end{cases}
\end{array}
\]
\[
basecount(\A\C\U,\A) = \underline{\phantom{\hspace{5in}}}
\]
\[
basecount(\A\C\U,\G) = \underline{\phantom{\hspace{5in}}}
\]

\section*{Friday January 8}


{\bf Definition} (Rosen p.\ 246) For $b$ an integer greater than $1$ and $n$ a positive integer, 
the {\bf base $b$ expansion of $n$}  is
\[
(a_{k-1} \cdots a_1 a_0)_b
\]
where $k$ is a positive integer, $a_0, a_1, \ldots, a_{k-1}$ are nonnegative integers less than $b$, $a_{k-1} \neq  0$, and
\[
n =  a_{k-1} b^{k-1} + \cdots + a_1b + a_0
\]

{\it The base $b$ expansion of a positive integer $n$ is a string over the alphabet $\{x \in \mathbb{N} \st x < b\}$
whose leftmost character is nonzero.}

\begin{center}
\begin{tabular}{|c|c|}
\hline
Base $b$ & Collection of possible coefficients in base $b$ expansion of  a positive integer \\
\hline
& \\
Binary ($b=2$) & $\{0,1\}$ \\
\hline
& \\
Ternary ($b=3$) & $\{0,1, 2\}$ \\
\hline
& \\
Octal ($b=8$) & $\{0,1, 2, 3, 4, 5, 6, 7\}$\\
\hline
& \\
Decimal ($b=10$) & $\{0,1, 2, 3, 4, 5, 6, 7, 8, 9\}$\\
\hline
& \\
Hexadecimal ($b=16$) &  $\{0,1, 2, 3, 4, 5, 6, 7, 8, 9, A, B, C, D, E, F\}$\\
& letter coefficient symbols represent numerical values $(A)_{16} = (10)_{10}$\\
&$(B)_{16} = (11)_{10} ~~(C)_{16} = (12)_{10} ~~
 (D)_{16} = (13)_{10} ~~ (E)_{16} = (14)_{10} ~~ (F)_{16} = (15)_{10} $\\
\hline
\end{tabular}
\end{center}



\begin{center}
\begin{tabular}{|p{1.75in}|p{1.75in}|p{1.75in}|p{1.75in}|}
\hline
Binary  $b=2$ & Octal $b=8$ & Decimal $b=10$ & Hexadecimal $b=16$ \\
\hline 
&&&  \\
$(1401)_{2}$&&&\\
&&&  \\
&&&  \\
\hline 
&&&  \\
& $(1401)_{8}$&&\\
&&&\\
&&&  \\
\hline
&&&\\
&&$(1401)_{10}$&\\
&&&  \\
&&&\\
\hline
&&&\\
&&& $(1401)_{16}$\\
&&&  \\
&&&\\
\hline
\end{tabular}
\end{center}

\fbox{\parbox{\textwidth}{%
{\bf New!} An algorithm is a finite sequence of precise instructions for solving a problem.
}}


\begin{multicols}{2}
\begin{minipage}{4in}
\begin{algorithm}[caption={Algorithm for calculating integer part of $\log$}]
procedure $\textit{log}$($n$: a positive integer)
$r$ := $0$
while $n$ > $1$
  $r$ := $r + 1$
  $n$ := $n$ div $2$
return $r$ $\{ r~\textrm{holds the result of the}~\log~\textrm{operation}\} $
\end{algorithm}
\end{minipage}
\begin{center}
\begin{tabular}{c|c|c}
$n$ & $r$  & $n > 1$?\\
\hline 
~$6$~ & \phantom{~$0$~} & \phantom{~T~}\\
\phantom{$3$} & \phantom{$1$} & \phantom{T}\\
\phantom{$1$} & \phantom{$2$} & \phantom{F}\\
&\\
\end{tabular}
\end{center}
\end{multicols}

{\bf Two algorithms for constructing base $b$ expansion from decimal representation}

{\bf Algorithm 1}: Start with highest power of $b$, i.e. at left-most coefficient of expansion
\begin{multicols}{2}
\begin{algorithm}[caption={Calculating integer part of $\log_b$}]
procedure $\textit{logb}$($n, b$: positive integers with $b > 1$)
$r$ := $0$
while $n$ > $1$
  $r$ := $r + 1$
  $n$ := $n$ div $b$
return $r$ $\{ r~\textrm{holds the result of the}~\log~\textrm{operation}\}$
\end{algorithm}
\columnbreak
\begin{algorithm}[caption={Calculating base $b$ expansion, from left}]
procedure $\textit{baseb1}$($n, b$: positive integers with $b > 1$)
$v$ := $n$
$k$ := $logb(n,b) + 1$
for $i$ := $1$ to $k$
  $a_{k-i}$ := $0$
  while $v \geq b^{k-i}$
    $a_{k-i}$ := $a_{k-i} + 1$
    $v$ := $v -  b^{k-i}$
return $(a_{k-1}, \ldots, a_0) \{(a_{k-1} \ldots a_0)_b~\textrm{ is the base } b \textrm{ expansion of } n \}$
\end{algorithm}
\end{multicols}


 {\bf The Division Algorithm} (Rosen 4.1 Theorem 2, p. 239) Let $n$ be an integer 
and $d$ a positive integer. There are unique integers $q$ and $r$, with $0 \leq r < d$, such that 
$n = dq + r$. In this case, $d$ is called the divisor, $n$ is called the dividend, $q$ is called the quotient, 
and $r$ is called the remainder. We write $q=n \textbf{ div } d$ and $r=n \textbf{ mod } d$.

\textit{Extra example}: How do $\textbf{ div }$ and $\textbf{ mod }$ compare to $/$ and $\%$ in Java and python?


{\bf Algorithm 2}: Start with right-most coefficient of expansion

\begin{multicols}{2}
\begin{minipage}{3.2in}
\begin{tabular}{c|c|c|c|c|c}
$n$ & $b$  & $q$ & $k$ & $a_k$ & $q \neq 0$?\\
\hline 
\phantom{~$17$~} & \phantom{~$3$~} & \phantom{~$17$~} & \phantom{~$0$~} & \phantom{~~~} & \phantom{~T~}\\
\phantom{~$17$~} & \phantom{~$3$~} & \phantom{~$5$~} & \phantom{~$1$~} & \phantom{~$a_0 = 2$~} & \phantom{~T~}\\
\phantom{~$17$~} & \phantom{~$3$~} & \phantom{~$1$~} & \phantom{~$2$~} & \phantom{~$a_1 = 2$~} & \phantom{~T~}\\
\phantom{~$17$~} & \phantom{~$3$~} & \phantom{~$0$~} & \phantom{~$3$~} & \phantom{~$a_2 = 1$~} & \phantom{~F~}\\
&&&&&\\
\end{tabular}
\end{minipage}
\columnbreak
\begin{algorithm}[caption={Calculating base $b$ expansion, from right}]
procedure $\textit{baseb2}$($n, b$: positive integers with $b > 1$)
$q$ := $n$
$k$ := $0$
while $q  \neq 0$
  $a_{k}$ := $q$ mod $b$
  $q$ := $q$ div $b$
  $k$ := $k+1$
return $(a_{k-1}, \ldots, a_0) \{(a_{k-1} \ldots a_0)_b~\textrm{ is the base } b \textrm{ expansion of } n \}$
\end{algorithm}

Idea: {\tiny(when $k > 1$)} $n = a_{k-1} b^{k-1} + \cdot+ a_1 b + a_0 = b ( a_{k-1} b^{k-2} + \cdots + a_1) + a_0$
so $a_0 = n \textbf{ mod } b$ and $a_{k-1} b^{k-2} + \cdots + a_1 = n \textbf{ div } b$.
\end{multicols}


\begin{center}
\begin{tabular}{p{3.3in}|p{3.3in}}
Algorithm 1 for the base 3 expansion of $17$ &Algorithm 2 for the base 3 expansion of $17$  \\
\hline
 & \\
 & \\
 & \\
 & \\
  & \\
 & \\
\end{tabular}
\end{center}




\section*{Review quiz questions}
\begin{enumerate}
\item Please complete the beginning of the quarter survey 
\href{https://forms.gle/8qnCRC26muASsXFi8}{https://forms.gle/8qnCRC26muASsXFi8}.

\item Consider $n=5$.  Define the following functions whose inputs are ordered pairs of $5$-tuples each of whose components
comes from the set $\{-1,0,1\}$
\[
d_{1,5}(~ (x_1, x_2, x_3, x_4, x_5) , (y_1, y_2, y_3, y_4, y_5) ~) =  \sum_{i=1}^5\left( (\abs{x_i-y_i} + 1) \textbf{ div } 2 \right)
\]
\[
d_{2,5}(~ (x_1, x_2, x_3, x_4, x_5) , (y_1, y_2, y_3, y_4, y_5) ~) =  \sqrt{ \sum_{i=1}^5 (x_i - y_i)^2}
\]

\begin{enumerate}
\item Consider the function application $d_{1,5} (~ (1,1,1,1,1), (1,1,1,1,1) ~)$
\begin{enumerate}
\item What is the input? 
\item What is the output?
\end{enumerate}

\item Consider the function application $d_{2,5} (~ (1,0,1,1,1), (0, 0,1,1,1) ~)$
\begin{enumerate}
\item What is the input? 
\item What is the output?
\end{enumerate}
\item What is the domain of the function $d_{1,5}$?
\begin{enumerate}
\item $\{-1,0,1\}$
\item $\{-1,0,1\} \times \{-1,0,1\}$
\item $\{-1,0,1\} \times \{-1,0,1\}\times \{-1,0,1\}\times \{-1,0,1\}\times \{-1,0,1\}$
\item None of the above.
\end{enumerate}

\item {\bf True} or {\bf False}: The functions $d_{1,5}$ and $d_{2,5}$ have the same domain.
\end{enumerate}

\item RNA is made up of strands of four different bases that match up in
specific ways. The bases are elements of the set 
$B  = \{\A, \C, \G, \U \}$.

{\bf Definition} The set of RNA strands $S$ is defined (recursively) by:

\[
\begin{array}{ll}
\textrm{Basis Step: } & \A \in S, \C \in S, \U \in S, \G \in S \\
\textrm{Recursive Step: } & \textrm{If } s \in S\textrm{ and }b \in B \textrm{, then }sb \in S
\end{array}
\]

A function \textit{rnalen} that computes the length of RNA strands in $S$ is defined by:
\[
\begin{array}{llll}
& & \textit{rnalen} : S & \to \mathbb{Z}^+ \\
\textrm{Basis Step:} & \textrm{If } b \in B\textrm{ then } & \textit{rnalen}(b) & = 1 \\
\textrm{Recursive Step:} & \textrm{If } s \in S\textrm{ and }b \in B\textrm{, then  } & \textit{rnalen}(sb) & = 1 + \textit{rnalen}(s)
\end{array}
\]

\begin{enumerate}
\item How many distinct elements are in the set described using set builder notation as 
\[
\{ x \in S \mid rnalen(x) = 1\} \qquad ?
\]

\item How many distinct elements are in the set described using set builder notation as 
\[
\{ x \in S \mid rnalen(x) = 2\} \qquad ?
\]

\item How many distinct elements are in the set described using set builder notation as 
\[
\{ rnalen(x) \mid x \in S \text{ and } rnalen(x) = 2\} \qquad ?
\]


\item How many distinct elements are in the set obtained as the result
of the set-wise concatenation $\{ \A\A, \A\C \} \circ \{\U, \A\A \}$?

\item How many distinct elements are in the set obtained as the result
of the Cartesian product $\{ \A\A, \A\C \} \times \{\U, \A\A \}$?

\item {\bf True} or {\bf False}: There is an example of an RNA strand that is both in the set obtained as the result
of the set-wise concatenation $\{ \A\A, \A\C \} \circ \{\U, \A\A \}$ and in the set obtained as the result of the 
Cartesian product $\{ \A\A, \A\C \} \times \{\U\A, \A\A \}$

\end{enumerate}
{\it Bonus - not for credit: Describe each of the sets above using roster method.}


\item For many applications in cryptography and random number generation,
dividing very large integers efficiently is critical.  Recall {\bf The Division Algorithm} (Rosen 4.1 Theorem 2, p. 239):
 Let $n$ be an integer 
and $d$ a positive integer. There are unique integers $q$ and $r$, with $0 \leq r < d$, such that 
$n = dq + r$. In this case, $d$ is called the divisor, $n$ is called the dividend, $q$ is called the quotient, 
and $r$ is called the remainder. We write $q=n \textbf{ div } d$ and $r=n \textbf{ mod } d$.

One application of the Division Algorithm is in computing the integer part of the logarithm.
When we discuss algorithms in this class, we will usually write them in 
pseudocode or English. Sometimes we will find it useful to relate the pseudocode to
runnable code in a programming language. We will typically use Java for this.

\begin{multicols}{2}
\begin{algorithm}[caption={Calculating log in pseudocode}]
procedure $\textit{log}$($n$: a positive integer)
$r$ := $0$
while $n$ > $1$
  $r$ := $r + 1$
  $n$ := $n$ div $2$
return $r$ $\{ r~\textrm{holds the result of the}~\log~\textrm{operation}\} $
\end{algorithm}
\columnbreak
\begin{java}[caption={Calculating log in Java}]
int log(int n) {
  if (n < 1) { 
    throw new IllegalArgumentException(); 
  }
  int result = 0;
  while(n > 1) {
    result = result + 1;
    n = n / 2;
  }
  return result;
}
\end{java}
\end{multicols}


\begin{enumerate}
\item Calculate $2021 \textbf{ div } 20$.  {\it You may use a calculator if you like.}
\item Calculate $2021 \textbf{ mod } 20$.  {\it You may use a calculator if you like.}
\item How many different possible values of $r$ (results of taking $n \textbf{ mod } d$) are there when 
$n$ is any positive integer and $d$ is $20$?
\item What is the smallest positive integer $n$ which can be written as $16q+7$ for $q$ an integer?
\item What is the return value of \textit{log$(457)$}?
{\it You can run the Java version in order to calculate it.}
\end{enumerate}


\item Colors can be described as amounts of red, green, and blue mixed together\footnote{This RGB representation
is common in web applications.  Many online tools are available to play around with mixing these colors, e.g. \url{https://www.w3schools.com/colors/colors_rgb.asp}}.  Mathematically, a color can be represented as a $3$-tuple $(r, g, b)$ where $r$
represents the red component, $g$ the green component, $b$ the blue component and where each of $r$, $g$, $b$ must
be a value from this collection of numbers:
\begin{quote}
$\{$0, 1, 2, 3, 4, 5, 6, 7, 8, 9, 10, 11, 12, 13, 14, 15, 16, 17, 18, 19, 20, 21, 22, 23, 24, 25, 26, 27, 28, 29, 30, 31, 32, 33, 34, 35, 36, 37, 38, 39, 40, 41, 42, 43, 44, 45, 46, 47, 48, 49, 50, 51, 52, 53, 54, 55, 56, 57, 58, 59, 60, 61, 62, 63, 64, 65, 66, 67, 68, 69, 70, 71, 72, 73, 74, 75, 76, 77, 78, 79, 80, 81, 82, 83, 84, 85, 86, 87, 88, 89, 90, 91, 92, 93, 94, 95, 96, 97, 98, 99, 100, 101, 102, 103, 104, 105, 106, 107, 108, 109, 110, 111, 112, 113, 114, 115, 116, 117, 118, 119, 120, 121, 122, 123, 124, 125, 126, 127, 128, 129, 130, 131, 132, 133, 134, 135, 136, 137, 138, 139, 140, 141, 142, 143, 144, 145, 146, 147, 148, 149, 150, 151, 152, 153, 154, 155, 156, 157, 158, 159, 160, 161, 162, 163, 164, 165, 166, 167, 168, 169, 170, 171, 172, 173, 174, 175, 176, 177, 178, 179, 180, 181, 182, 183, 184, 185, 186, 187, 188, 189, 190, 191, 192, 193, 194, 195, 196, 197, 198, 199, 200, 201, 202, 203, 204, 205, 206, 207, 208, 209, 210, 211, 212, 213, 214, 215, 216, 217, 218, 219, 220, 221, 222, 223, 224, 225, 226, 227, 228, 229, 230, 231, 232, 233, 234, 235, 236, 237, 238, 239, 240, 241, 242, 243, 244, 245, 246, 247, 248, 249, 250, 251, 252, 253, 254, 255$\}$
\end{quote}

\begin{enumerate}
\item \textbf{True} or \textbf{False}: $(1, 3, 4)$ fits the definition of a color above.
\item \textbf{True} or \textbf{False}: $(1, 100, 200, 0)$ fits the definition of a color above.
\item \textbf{True} or \textbf{False}: $(510, 255)$ fits the definition of a color above.
\item \textbf{True} or \textbf{False}: There is a color $(r_1, g_1, b_1)$ where $r_1 + g_1 + b_1$ is greater than $765$.
\item \textbf{True} or \textbf{False}: There is a color $(r_2, g_2, b_2)$ where $r_2 + g_2 + b_2$ is equal to $1$.
\item \textbf{True} or \textbf{False}: Another way to write the collection of allowed values for red, green, and blue components is $$\{x \in \mathbb{N}\mid 0 \leq x \leq 255 \}$$.
\item \textbf{True} or \textbf{False}: Another way to write the collection of allowed values for red, green, and blue components is $$\{n \in \mathbb{Z}\mid 0 \leq n \leq 255 \}$$.
\item \textbf{True} or \textbf{False}: Another way to write the collection of allowed values for red, green, and blue components is $$\{y \in \mathbb{Z}\mid -1 < y \leq 255 \}$$.
\end{enumerate}

\end{enumerate}


\end{document}


\item Consider the $5$-tuples $x = (1, 1, 0, 1, 1)$, $y = (0, 0, 1, 0, 1)$, $z= (0, 1, 0, 0, 1)$.  
Is $x$ closer to $y$ or $z$ using $d_{1,5}$?
\item Consider the $5$-tuples $x = (1, 1, 0, 1, 1)$, $y = (0, 0, 1, 0, 1)$, $z= (0, 1, 0, 0, 1)$.  
Is $x$ closer to $y$ or $z$ using $d_{2,5}$?

{\bf Definition} (Rosen p.\ 123) Let $A$ and $B$ be sets.  The {\bf Cartesian product} of $A$ and $B$, denoted
$A \times B$, is the set of all ordered pairs $(a,b)$ where $a \in A$ and $b \in B$
\[
A \times B = \{ (a,b) \mid a \in A \text{ and } b \in B \}
\]
{\bf Definition}: Let $A$ and $B$ be sets of strings over the same alphabet. The {\bf set-wise concatenation} 
of $A$ and $B$, denoted $A \circ B$, is the set of all results of string concatenation $ab$ where $a \in A$ and $b \in B$
\[
A \circ B = \{ ab \mid a \in A \text{ and } b \in B \}
\]

\end{document}  
