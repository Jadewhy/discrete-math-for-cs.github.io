\documentclass[12pt, oneside]{article}

\usepackage{amssymb,amsmath,pifont,amsfonts,comment,enumerate,enumitem}
\usepackage{currfile,xstring,hyperref,tabularx,graphicx,wasysym}
\usepackage[labelformat=empty]{caption}
\usepackage[dvipsnames,table]{xcolor}
\usepackage{multicol,multirow,array,listings,tabularx,lastpage,textcomp,booktabs}

% NOTE(joe): This environment is credit @pnpo (https://tex.stackexchange.com/a/218450)
\lstnewenvironment{algorithm}[1][] %defines the algorithm listing environment
{   
    \lstset{ %this is the stype
        mathescape=true,
        frame=tB,
        numbers=left, 
        numberstyle=\tiny,
        basicstyle=\rmfamily\scriptsize, 
        keywordstyle=\color{black}\bfseries,
        keywords={,procedure, div, for, to, input, output, return, datatype, function, in, if, else, foreach, while, begin, end, }
        numbers=left,
        xleftmargin=.04\textwidth,
        #1
    }
}
{}
\lstnewenvironment{java}[1][]
{   
    \lstset{
        language=java,
        mathescape=true,
        frame=tB,
        numbers=left, 
        numberstyle=\tiny,
        basicstyle=\ttfamily\scriptsize, 
        keywordstyle=\color{black}\bfseries,
        keywords={, int, double, for, return, if, else, while, }
        numbers=left,
        xleftmargin=.04\textwidth,
        #1
    }
}
{}

\newcommand\abs[1]{\lvert~#1~\rvert}
\newcommand{\st}{\mid}

\newcommand{\A}[0]{\texttt{A}}
\newcommand{\C}[0]{\texttt{C}}
\newcommand{\G}[0]{\texttt{G}}
\newcommand{\U}[0]{\texttt{U}}

\newcommand{\cmark}{\ding{51}}
\newcommand{\xmark}{\ding{55}}



\title{HW1 Definitions and Notation}
\author{CSE20W21}
\date{Due: Tuesday, January 12, 2021 at 11:00PM on Gradescope}

\begin{document}
\maketitle

{\bf In this assignment,}

You will practice reading and
applying definitions to get comfortable working with mathematical language. As
a result, you can expect to spend more time reading the questions and looking
up notation than doing calculations.


{\bf For all HW assignments:}

Weekly homework may be done individually or in groups of up to 4 students. You may switch HW partners for different HW assignments. The lowest HW score will not be included in your overall HW average. Please ensure your name(s) and PID(s) are clearly visible on the first page of your homework submission.

All submitted homework for this class must be typed. Diagrams may be hand-drawn and scanned and included in the typed document. You can use a word processing editor if you like (Microsoft Word, Open Office, Notepad, Vim, Google Docs, etc.) but you might find it useful to take this opportunity to learn LaTeX. LaTeX is a markup language used widely in computer science and mathematics. The homework assignments are typed using LaTeX and you can use the source files as templates for typesetting your solutions\footnote{To use this template, you will need to copy both the source file (extension \texttt{.tex})  you'll be editing
and the file containing all the ``shortcut" commands we've defined for this class \href{https://drive.google.com/file/d/1-iFMqMK-_PvjW1czijoV8kuOP02alWw6/view?usp=sharing}{CSE20packages.tex})} .


{\bf Integrity reminders}
\begin{itemize}
\item Problems should be solved together, not divided up between the partners. The homework is
designed to give you practice with the main concepts and techniques of the course, while getting to know and learn from your classmates.
\item You may not collaborate on homework with anyone other than your group members.
You may ask questions about the homework in office hours (of the instructor, TAs, and/or tutors) and 
on Piazza.  You \emph{cannot} use any online resources about the course content other than the text
book and class material from this quarter -- this is primarily to ensure that we all use consistent notation and
definitions we will use this quarter.
\item Do not share written solutions or partial solutions for homework with other students in the class who are not in your group. Doing so would dilute their learning experience and detract from their success in the class.
\end{itemize}


You will submit this assignment via Gradescope
(\href{https://www.gradescope.com}{https://www.gradescope.com}) in the assignment called ``HW1-definitions-and-notation''.


{\bf Resources}

To review the topics you are working with for this assignment, see the class material from 
Week 1\footnote{Week 1 material \href{https://drive.google.com/drive/folders/1ldVPHpuoxXIwIIy-tJ4flm-w9L6vjlC_?usp=sharing}
{Google Drive folder}}.

Relevant examples in the textbook (all numbers and pages refer to the 7th edition) include: 
Examples 17-18 on page 123. Section 2.1 exercises 1, 2, 27. Examples 2-3 on page 346, Example 5 on page 349, Examples 6-7 on page 350. Section 5.3 exercises 23, 25, 39. Examples 3-4 on pages 239-240, Examples 1-6 on pages 246-248. Section 4.2 exercises 1, 3, 5, 9, 11.

We will post frequently asked questions and our answers to them in a 
shared FAQ doc linked below\footnote{FAQ \href{https://docs.google.com/document/d/1UompYSqscNBDO43VKvcO17HCwLJGP6d7XRa_poit4Cc}{Google doc}}.

{\bf Assigned questions}

\begin{enumerate}

\item  In machine learning, clustering can be used to group similar data for prediction and recommendation.  For example,
each Netflix user's viewing history can be represented as a $n$-tuple indicating their preferences about
movies in the database, where $n$ is the number of movies in the database.  People with similar tastes in movies can then be clustered to provide recommendations
of movies for one another.  Mathematically, clustering is based on a notion of distance between pairs of $n$-tuples.

Consider $n=5$.  Define the following functions whose inputs are ordered pairs of $5$-tuples each of whose components
comes from the set $\{-1,0,1\}$
\[
d_{1,5}(~ (x_1, x_2, x_3, x_4, x_5) , (y_1, y_2, y_3, y_4, y_5) ~) =  \sum_{i=1}^5\left( (\abs{x_i-y_i} + 1) \textbf{ div } 2 \right) 
\]
\[
d_{2,5}(~ (x_1, x_2, x_3, x_4, x_5) , (y_1, y_2, y_3, y_4, y_5) ~) =  \sqrt{ \sum_{i=1}^5 (x_i - y_i)^2}
\]


\begin{enumerate}
\item ({\it Graded for fair effort completeness}\footnote{This means you will get full credit so long as your submission demonstrates honest effort to answer the question. You will not be penalized for incorrect answers.}) 
Give precise definitions of the absolute value function and the summation function.
The input for the absolute value function is a real number and the output the magnitude
(value ignoring sign) of the input. The input for the summation function is a sequence of numbers and the output
is the sum of those numbers.

{\it Note}: A piecewise definition is appropriate for the absolute value function and a recursive definition is appropriate
for the summation function. Include a description of the domain and codomain of the function, along with the
basis step and the recursive step of the function definition.

\item ({\it Graded for correctness}\footnote{This means your solution will be
evaluated not only on the correctness of your answers, but on your ability to 
present your ideas clearly and logically. You should explain how you arrived at your conclusions, using 
mathematically sound reasoning. Whether you use formal proof techniques or write a more informal argument for why 
something is true, your answers should always be well-supported. Your goal should be to convince the reader that 
your results and methods are sound.}) Provide the following examples.  
Each of your answers should include  {\bf both} specific values for the example {\bf and} a justification 
of your examples with (clear, correct, complete) calculations and/or references to definitions, and connecting them with
the desired conclusion.

\rule{0.5\textwidth}{.4pt}

{\it Sample response that can be used as reference for the detail expected 
in your answer:} An example input to $d_{1,5}$ for which the output of the function is $0$ is 
$( (1,1,1,1,1), (1,1,1,1,1) )$.  We can use the definition of this function to verify that the output is what we said:
\begin{align*}
d_{1,5} (~ (1,1,1,1,1), (1,1,1,1,1) ~) &=   \sum_{i=1}^5\left( (\abs{1-1} + 1) \textbf{ div } 2 \right) \\
&=5(~(\abs{0}+1) \textbf{ div } 2~) = 5 (1 \textbf{ div } 2) = 5 (0) = 0.
\end{align*}

\rule{0.5\textwidth}{.4pt}

\begin{enumerate}
\item Give one example input to $d_{1,5}$ for which the output of the function is $3$.
\item Give one example input to $d_{2,5}$ for which the output of the function is $1$.
\item Give one example input for which the output of $d_{1,5}$ is $1$ and the output of 
$d_{2,5}$ is not $1$. (Include justifications for both function applications.)
\end{enumerate}

\end{enumerate}


\item RNA is made up of strands of four different bases that match up in
specific ways. The bases are elements of the set 
$B  = \{\A, \C, \G, \U \}$.

{\bf Definition} The set of RNA strands $S$ is defined (recursively) by:

\[
\begin{array}{ll}
\textrm{Basis Step: } & \A \in S, \C \in S, \U \in S, \G \in S \\
\textrm{Recursive Step: } & \textrm{If } s \in S\textrm{ and }b \in B \textrm{, then }sb \in S
\end{array}
\]

A function \textit{rnalen} that computes the length of RNA strands in $S$ is defined by:
\[
\begin{array}{llll}
& & \textit{rnalen} : S & \to \mathbb{Z}^+ \\
\textrm{Basis Step:} & \textrm{If } b \in B\textrm{ then } & \textit{rnalen}(b) & = 1 \\
\textrm{Recursive Step:} & \textrm{If } s \in S\textrm{ and }b \in B\textrm{, then  } & \textit{rnalen}(sb) & = 1 + \textit{rnalen}(s)
\end{array}
\]Each of the sets below is described using set builder notation.  Rewrite them using the roster method.
For example, the set described in set builder notation as
\[
\{ s \in S \mid \text{the leftmost base in $s$ is $\A$ and $rnalen(s) = 2$} \} 
\]
is described using the roster method by
\[
\{ \A\A , \A\C , \A\G, \A\U\}.
\]
Justifications aren't required for credit for this question,
but it's good practice to think about how you would explain why your answer is correct.

\begin{enumerate}
\item ({\it Graded for correctness})\begin{align*}
\{ s \in S \mid &\text{ the leftmost three bases in $s$ are $\A\U\G$ (in order),}\\
&\text{the rightmost base of $s$ is not the same as its leftmost base, and $rnalen(s) = 4$} \}
\end{align*}

\item ({\it Graded for correctness})
\[
\{ s \in S \mid \text{ $s$ has at least two $\C$s, and there are twice as many $\G$s as $\A$s in $s$, and $rnalen(s) = 3$} \}
\]
\end{enumerate}

\item Recall {\bf The Division Algorithm} (Rosen 4.1 Theorem 2, p. 239):
 Let $n$ be an integer 
and $d$ a positive integer. There are unique integers $q$ and $r$, with $0 \leq r < d$, such that 
$n = dq + r$. In this case, $d$ is called the divisor, $n$ is called the dividend, $q$ is called the quotient, 
and $r$ is called the remainder. We write $q=n \textbf{ div } d$ and $r=n \textbf{ mod } d$.

Mathematically, a color can be represented as a $3$-tuple $(r, g, b)$ where $r$
represents the red component, $g$ the green component, $b$ the blue component and where each of $r$, $g$, $b$ must be from the collection $\{x \in \mathbb{N}\mid 0 \leq x \leq 255 \}$.

\begin{enumerate}
\item  ({\it Graded for correctness}) Convert the number expressed in hexadecimal as $(FF)_{16}$ to binary {\bf and} to decimal.
Justify your answer
with (clear, correct, complete) calculations and/or references to definitions, and connecting these
calculations and/or definitions with
the desired conclusion.
What is the color $(x,x,x)$ where $x = (F0)_{16}$? Describe how you know what this color is 
(if you use a web color tool, include its URL in your submission writeup; if not, describe your reasoning).


\item  ({\it Graded for correctness}) Suppose you were told that the colors that will work best for your web app are $(r,g,b)$ where $r \textbf{ mod } 16 = 8$ and $g \textbf{ mod } 16 = 8$
and $b \textbf{ mod } 16 = 8$.  Give three distinct examples of such colors.
For each example, describe it informally in English (or include a sample image of the color) and then specify its 
red, green, and blue components both in decimal and in hexadecimal.  
Justify each of your examples
with (clear, correct, complete) calculations and/or references to definitions, and connecting these
calculations and/or definitions with
the desired properties.
\end{enumerate}


\item Upload your work to Gradescope.  
You can choose to upload your assignment as a single PDF\footnote{Gradescope tutorial for 
uploading PDF to an assignment \url{https://help.gradescope.com/article/ccbpppziu9-student-submit-work}; search for ``Submitting a PDF''}
or as individual images\footnote{Gradescope tutorial for uploading images to an assignment \url{https://help.gradescope.com/article/ccbpppziu9-student-submit-work}; search for ``submitting individual images''}.

You must assign your pages to questions in Gradescope assignment. This helps the graders see all of your work.

If you worked on this assignment in a group 
{\bf only one group member should upload your entire group's work}.  After they upload the submission,
they then need to {\bf add group member(s)} so that the submission is associated with 
all group members. Gradescope provides a video tutorial\footnote{Adding group members 
\url{https://help.gradescope.com/article/m5qz2xsnjy-student-add-group-members} for adding group members.}


\end{enumerate}

\end{document}
