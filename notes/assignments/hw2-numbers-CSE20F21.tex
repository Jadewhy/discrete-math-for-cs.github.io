\documentclass[12pt, oneside]{article}

\usepackage[letterpaper, scale=0.89, centering]{geometry}
\usepackage{fancyhdr}
\setlength{\parindent}{0em}
\setlength{\parskip}{1em}

\pagestyle{fancy}
\fancyhf{}
\renewcommand{\headrulewidth}{0pt}
\rfoot{\href{https://creativecommons.org/licenses/by-nc-sa/2.0/}{CC BY-NC-SA 2.0} Version \today~(\thepage)}

\author{CSE20F21}

\newcommand{\instructions}{{\bf For all HW assignments:}

Weekly homework may be done individually or in groups of up to 3 students. 
You may switch HW partners for different HW assignments. 
The lowest HW score will not be included in your overall HW average. 
Please ensure your name(s) and PID(s) are clearly visible on the first page of your homework submission.

All submitted homework for this class must be typed. 
Diagrams may be hand-drawn and scanned and included in the typed document. 
You can use a word processing editor if you like (Microsoft Word, Open Office, Notepad, Vim, Google Docs, etc.) 
but you might find it useful to take this opportunity to learn LaTeX. 
LaTeX is a markup language used widely in computer science and mathematics. 
The homework assignments are typed using LaTeX and you can use the source files 
as templates for typesetting your solutions\footnote{To use this template, copy the source file (extension \texttt{.tex}) 
to your working directory or upload to Overleaf.}.


{\bf Integrity reminders}
\begin{itemize}
\item Problems should be solved together, not divided up between the partners. The homework is
designed to give you practice with the main concepts and techniques of the course, 
while getting to know and learn from your classmates.
\item You may not collaborate on homework with anyone other than your group members.
You may ask questions about the homework in office hours (of the instructor, TAs, and/or tutors) and 
on Piazza (as private notes viewable only to the Instructors).  
You \emph{cannot} use any online resources about the course content other than the class material 
from this quarter -- this is primarily to ensure that we all use consistent notation and
definitions we will use this quarter.
\item Do not share written solutions or partial solutions for homework with 
other students in the class who are not in your group. Doing so would dilute their learning 
experience and detract from their success in the class.
\end{itemize}

}
\usepackage{amssymb,amsmath,pifont,amsfonts,comment,enumerate,enumitem}
\usepackage{currfile,xstring,hyperref,tabularx,graphicx,wasysym}
\usepackage[labelformat=empty]{caption}
\usepackage[dvipsnames,table]{xcolor}
\usepackage{multicol,multirow,array,listings,tabularx,lastpage,textcomp,booktabs}

% NOTE(joe): This environment is credit @pnpo (https://tex.stackexchange.com/a/218450)
\lstnewenvironment{algorithm}[1][] %defines the algorithm listing environment
{   
    \lstset{ %this is the stype
        mathescape=true,
        frame=tB,
        numbers=left, 
        numberstyle=\tiny,
        basicstyle=\rmfamily\scriptsize, 
        keywordstyle=\color{black}\bfseries,
        keywords={,procedure, div, for, to, input, output, return, datatype, function, in, if, else, foreach, while, begin, end, }
        numbers=left,
        xleftmargin=.04\textwidth,
        #1
    }
}
{}
\lstnewenvironment{java}[1][]
{   
    \lstset{
        language=java,
        mathescape=true,
        frame=tB,
        numbers=left, 
        numberstyle=\tiny,
        basicstyle=\ttfamily\scriptsize, 
        keywordstyle=\color{black}\bfseries,
        keywords={, int, double, for, return, if, else, while, }
        numbers=left,
        xleftmargin=.04\textwidth,
        #1
    }
}
{}

\newcommand\abs[1]{\lvert~#1~\rvert}
\newcommand{\st}{\mid}

\newcommand{\A}[0]{\texttt{A}}
\newcommand{\C}[0]{\texttt{C}}
\newcommand{\G}[0]{\texttt{G}}
\newcommand{\U}[0]{\texttt{U}}

\newcommand{\cmark}{\ding{51}}
\newcommand{\xmark}{\ding{55}}





\title{HW2 Numbers}
\date{Due: Tuesday, October 12, 2021 at 11:00PM on Gradescope}

\begin{document}
\maketitle
\thispagestyle{fancy}


{\bf In this assignment,}

You will consider multiple number representations and how they connect 
to applications in computer science. You will also practice tracing
and working with algorithms.

Instructions and academic integrity reminders for all homework assignments in 
CSE20 this quarter are on the class website and on the hw1-definitions-and-notations
assignment.


You will submit this assignment via Gradescope
(\href{https://www.gradescope.com}{https://www.gradescope.com}) 
in the assignment called ``hw2-numbers''.


{\bf Resources}: To review the topics you are working with 
for this assignment, see the class material from Friday of Week 1 and Monday
 and Wednesday of Week 2.
We will post frequently asked questions and our answers to them in a 
pinned Piazza post.

{\bf Assigned questions}

\begin{enumerate}
    \item ({\it Graded for fair effort completeness}\footnote{This means 
    you will get full credit so long as your submission demonstrates honest 
    effort to answer the question. You will not be penalized for incorrect answers.})
    Pick a nonnegative integer between 100 and 1000 (inclusive) and express it using at 
    least three different representations, 
    at least two of which must be ones discussed in class. Include at least one
    trace of using procedure $baseb1$ from page 16 of the Week 0 and 1 notes
    to calculate the base $b_1$ expansion of your
    number (for your choice of $b_1$) and include at least one trace of using 
    procedure $baseb2$ from page 16 of the Week 0 and 1 notes
    to calculate the base $b_2$ expansion of your number
    (for your choice of $b_2$). 
    
    {\it Extra; not for credit:} Consider choosing representations that might be useful in 
    some way; what would make one representation more useful than another?

    \newpage
    \item In this question, we will use recursive definitions to give a precise description 
    of the set of strings that form octal (base $8$) expansions and the values
    of those expansions. Let's start by defining the set of possible coefficients
    \[
        C_8 = \{ 0, 1, 2, 3, 4, 5, 6, 7\}
    \]

    \begin{enumerate}
    \item ({\it Graded for fair effort completeness}) Fill in the recursive definition of the 
    set of all strings that form octal expansions, $S_8$: 
    
    {\bf Definition} The set $S_8$ is defined (recursively) by:
    \[
    \begin{array}{ll}
    \textrm{Basis Step: } & \textrm{If } x \in \underline{\phantom{\hspace{2in}}} \textrm{, then } x \in S_8\\
    \textrm{Recursive Step: } & \textrm{If } s \in S_8 \textrm{ and } x \in C_8 \textrm{, then }
    \underline{\phantom{\hspace{2in}}} \in S_8
    \end{array}
    \]
    
    \item  ({\it Graded for correctness}\footnote{This means your solution will be
    evaluated not only on the correctness of your answers, but on your ability to 
    present your ideas clearly and logically. You should explain how you arrived at 
    your conclusions, using 
    mathematically sound reasoning. Whether you use formal proof techniques or 
    write a more informal argument for why 
    something is true, your answers should always be well-supported. Your goal 
    should be to convince the reader that 
    your results and methods are sound.})
    Consider the function $v_8: S_8 \to \mathbb{Z}^+$ defined recursively by 
    
    \begin{quote}
    Basis Step: If $x \in C_8$, then $v_8 (x) = x$.
    
    Recursive Step: If $s \in S_8$ and $x \in C_8$, then $v_8 (sx) = 8v_8(s) + x$, where 
    the input $sx$ is the result of string concatenation and the output $8 v_8 (s)$
    is the result of integer multiplication.
    \end{quote}

    Calculate $v_8(104)$, including all steps in your calculation and justifications for them.

    \item ({\it Graded for fair effort completeness})
    It turns out\footnote{We'll be able to prove this in Week 7 or so, once
    we've talked about induction.} that for any string $u$ in $S_8$, the value of the 
    octal expansion $(u)_8$ equals $v_8(u)$. Using this fact, write an expression 
    relating the value of $(u000)_8$ to the value of $(u)_8$ and justify it. 
    \end{enumerate}
    
    
    \item ({\it Graded for correctness}) 
    Recall that, mathematically, a color can be represented as a 
    $3$-tuple $(r, g, b)$ where $r$
    represents the red component, $g$ the green component, $b$ the blue component 
    and where each of $r$, $g$, $b$ must be from the collection 
    $\{x \in \mathbb{N}\mid 0 \leq x \leq 255 \}$.
    
    As an {\bf alternative} representation, in this assignment
    we'll use base $16$ fixed-width expansions to represent colors
    as single numbers.
    
    {\bf Definition}: A {\bf hex color} is a nonnegative
    integer less than or equal to $16777215$. 
    For $n$ a hex color, we define its red, green, and blue components
    by first writing its base $16$ fixed-width $6$ expansion
    $$n = (r_1r_2g_1g_2b_1b_2)_{16,6}$$ and 
    then defining
    $(r_1r_2)_{16,2}$ is the red
    component, $(g_1g_2)_{16,2}$ is the green component, and $(b_1b_2)_{16,2}$ is the
    blue component.
    
    \newpage
    \rule{0.5\textwidth}{.4pt}

    {\it Sample response that can be used as reference for the detail expected 
    in your answer:} 
    
    In RGB codes\footnote{You can use online tools to visualize the colors associated
    with different values for the red, green, and blue components, 
    e.g. \url{https://www.w3schools.com/colors/colors_rgb.asp}. }
    white is represented as maximum red, maximum green, and maximum blue 
    and so has
    $(FF)_{16,2}$ as each of these components. This means that the hex color for white
    is $(FFFFFF)_{16,6}$ which is the value 
    \[
        15\cdot 16^5 + 15 \cdot 16^4 + 15 \cdot 16^3 + 15 \cdot 16^2 + 15 \cdot 16^1 + 15 \cdot 16^0 
        = 16^6 - 1 = 16777215
    \]
    \rule{0.5\textwidth}{.4pt}

    \begin{enumerate}
    \item  Write the hex color representing red (with no green or blue) 
    in base $16$ fixed-width $6$ and also calculate its value 
    (using usual mathematical conventions).
    Include your (clear, correct, complete) calculations.
    \item  Write the hex color representing green (with no red or blue) 
    in base $16$ fixed-width $6$ and also calculate its value
    (using usual mathematical conventions).
    Include your (clear, correct, complete) calculations.
    \item  Write the hex color representing blue (with no red or green) 
    in base $16$ fixed-width $6$ and also calculate its values
    (using usual mathematical conventions).
    Include your (clear, correct, complete) calculations.
    \item The human eye can't distinguish between some hex colors because of 
    physical limitations. Give an example of two hex colors $c_1$ and $c_2$ such that
    they look indistinguishable (the colors they represent are very very similar) but
        \[
            c_1 - c_2 > 50000
        \]
    Justify your choice 
    with (clear, correct, complete) calculations and/or references to definitions, 
    and connecting these
    calculations and/or definitions with
    the desired properties.  Include squares with each of your two colors so 
    that we can see how indistinguishable they are. 

    Pro tip: To show a color, you can use the following LaTeX source code:
    
        \verb|\definecolor{UCSDaccent}{RGB}{0,198,215}|

        \verb|\textcolor{UCSDaccent}{\rule{1cm}{1cm}}|

    which produces 
    \definecolor{UCSDaccent}{RGB}{0,198,215}
    \textcolor{UCSDaccent}{\rule{1cm}{1cm}}

    Notice that the code to define the color uses the decimal-like values 
    for each of the red, green, and blue components. For the UCSD accent color we defined, 
    the base $16$ fixed-width $2$ values are: red is $(00)_{16,2}$, green is $(C6)_{16,2}$,
    blue is $(D7)_{16,2}$.

    {\it Extra; not for credit:} What does this mean about the choice of hex color for
    representing colors? What are advantages and disadvantages of this representation?

    \end{enumerate}

    \item ({\it Graded for correctness}) In class (Week 2 notes page 7), we discussed fixed-width addition. In this
    question we will look at fixed-width multiplication. The algorithm for fixed-width 
    multiplication is to multiply using the usual long-multiplication algorithm 
    (column-by-column and carry), and dropping all leftmost columns so the result is the same 
    width as the input terms. For each of the examples below, consider whether 
    this algorithm gives the correct value for the product of the two numbers, based on
    the way the bitstrings are interpreted.

    \rule{0.5\textwidth}{.4pt}

    {\it Sample response that can be used as reference for the detail expected 
    in your answer:} 
    
    The fixed-width $5$ multiplication of $[00101]_{2c,5}$ and $[00101]_{2c,5}$ 
    does not give the correct
    value for the product, as we can see from the following calculation.
    
    First, we calculate the values: 
    \[
        [00101]_{2c,5} = 0\cdot (-2^4) + 0\cdot 2^3 + 1 \cdot 2^2 + 0\cdot 2^1 + 1 \cdot 2^0 = 4 + 1 = 5
    \]
    so the correct value for the product is $5 \cdot 5 = 25$, which cannot be represented
    in 2s complement width 5 (the largest positive number that can be represented in 
    2s complement width 5 is $[01111]_{2c,5} = 8 + 4 + 2 + 1 = 15$).
    
    When we perform the fixed-width $5$ multiplication algorithm:
    \begin{align*}
            & 0~ 0~ 1~ 0~ 1\\
     \times & 0~ 0~ 1~ 0~ 1\\
     &\overline{0~ 0~ 1~ 0~ 1}\\
     + {0~} & {0~0~0~0~~}\\
     + {0~0~} & {1~0~1~~}\\
     \overline{\phantom{0~0~}}&\overline{1~ 1~ 0~ 0~ 1}\\
    \end{align*}

    we get $[11001]_{2c,5} = 1\cdot (-2^4) + 1 \cdot 2^3 + 0 \cdot 2^2 + 0 \cdot 2^1 + 1 \cdot 2^0 = -16 + 8 + 1 = -7$, 
    which is not the required value.

    \rule{0.5\textwidth}{.4pt}


    \begin{enumerate}
        \item Does the fixed-width $5$ multiplication of $[11101]_{2c,5}$ and 
        $[11011]_{2c,5}$ give the correct value for the product?
        Justify your answer  
        with (clear, correct, complete) calculations and/or references to definitions, 
        and connecting these
        calculations and/or definitions with
        your answer.
        
        \item Does the fixed-width $5$ multiplication of $[00100]_{2c,5}$ and 
        $[11100]_{2c,5}$ give the correct value for the product?
        Justify your answer  
        with (clear, correct, complete) calculations and/or references to definitions, 
        and connecting these
        calculations and/or definitions with
        your answer.
    \end{enumerate}
    
\end{enumerate}
\end{document}