\documentclass[12pt, oneside]{article}

\usepackage[letterpaper, scale=0.89, centering]{geometry}
\usepackage{fancyhdr}
\setlength{\parindent}{0em}
\setlength{\parskip}{1em}

\pagestyle{fancy}
\fancyhf{}
\renewcommand{\headrulewidth}{0pt}
\rfoot{\href{https://creativecommons.org/licenses/by-nc-sa/2.0/}{CC BY-NC-SA 2.0} Version \today~(\thepage)}

\author{CSE20F21}

\newcommand{\instructions}{{\bf For all HW assignments:}

Weekly homework may be done individually or in groups of up to 3 students. 
You may switch HW partners for different HW assignments. 
The lowest HW score will not be included in your overall HW average. 
Please ensure your name(s) and PID(s) are clearly visible on the first page of your homework submission.

All submitted homework for this class must be typed. 
Diagrams may be hand-drawn and scanned and included in the typed document. 
You can use a word processing editor if you like (Microsoft Word, Open Office, Notepad, Vim, Google Docs, etc.) 
but you might find it useful to take this opportunity to learn LaTeX. 
LaTeX is a markup language used widely in computer science and mathematics. 
The homework assignments are typed using LaTeX and you can use the source files 
as templates for typesetting your solutions\footnote{To use this template, copy the source file (extension \texttt{.tex}) 
to your working directory or upload to Overleaf.}.


{\bf Integrity reminders}
\begin{itemize}
\item Problems should be solved together, not divided up between the partners. The homework is
designed to give you practice with the main concepts and techniques of the course, 
while getting to know and learn from your classmates.
\item You may not collaborate on homework with anyone other than your group members.
You may ask questions about the homework in office hours (of the instructor, TAs, and/or tutors) and 
on Piazza (as private notes viewable only to the Instructors).  
You \emph{cannot} use any online resources about the course content other than the class material 
from this quarter -- this is primarily to ensure that we all use consistent notation and
definitions we will use this quarter.
\item Do not share written solutions or partial solutions for homework with 
other students in the class who are not in your group. Doing so would dilute their learning 
experience and detract from their success in the class.
\end{itemize}

}
\usepackage{amssymb,amsmath,pifont,amsfonts,comment,enumerate,enumitem}
\usepackage{currfile,xstring,hyperref,tabularx,graphicx,wasysym}
\usepackage[labelformat=empty]{caption}
\usepackage[dvipsnames,table]{xcolor}
\usepackage{multicol,multirow,array,listings,tabularx,lastpage,textcomp,booktabs}

% NOTE(joe): This environment is credit @pnpo (https://tex.stackexchange.com/a/218450)
\lstnewenvironment{algorithm}[1][] %defines the algorithm listing environment
{   
    \lstset{ %this is the stype
        mathescape=true,
        frame=tB,
        numbers=left, 
        numberstyle=\tiny,
        basicstyle=\rmfamily\scriptsize, 
        keywordstyle=\color{black}\bfseries,
        keywords={,procedure, div, for, to, input, output, return, datatype, function, in, if, else, foreach, while, begin, end, }
        numbers=left,
        xleftmargin=.04\textwidth,
        #1
    }
}
{}
\lstnewenvironment{java}[1][]
{   
    \lstset{
        language=java,
        mathescape=true,
        frame=tB,
        numbers=left, 
        numberstyle=\tiny,
        basicstyle=\ttfamily\scriptsize, 
        keywordstyle=\color{black}\bfseries,
        keywords={, int, double, for, return, if, else, while, }
        numbers=left,
        xleftmargin=.04\textwidth,
        #1
    }
}
{}

\newcommand\abs[1]{\lvert~#1~\rvert}
\newcommand{\st}{\mid}

\newcommand{\A}[0]{\texttt{A}}
\newcommand{\C}[0]{\texttt{C}}
\newcommand{\G}[0]{\texttt{G}}
\newcommand{\U}[0]{\texttt{U}}

\newcommand{\cmark}{\ding{51}}
\newcommand{\xmark}{\ding{55}}





\title{HW5 Proofs and Induction}
\date{Due: Tuesday, November 16, 2021 at 11:00PM on Gradescope}

\begin{document}
\maketitle
\thispagestyle{fancy}

{\bf In this assignment,}

You will work with recursively defined sets and functions and prove 
properties about them, practicing induction and other proof strategies.

Instructions and academic integrity reminders for all homework assignments in 
CSE20 this quarter are on the class website and on the hw1-definitions-and-notations
assignment.

You will submit this assignment via Gradescope
(\href{https://www.gradescope.com}{https://www.gradescope.com}) 
in the assignment called ``hw5-proofs-and-induction''.

{\bf Resources}: To review the topics you are working with 
for this assignment, see the class material from Weeks 5 through 7.
We will post frequently asked questions and our answers to them in a 
pinned Piazza post.


In your proofs and disproofs of statements below, justify each  step
by reference to  a component of the  following proof  strategies
we  have discussed so far, and/or to relevant definitions and calculations.
\begin{itemize}
    \item A counterexample can be used to prove that  $\forall x P(x)$ is {\bf false}.
    \item  A witness can be used to prove that  $\exists x P(x)$ is {\bf true}.
    \item {\bf Proof of universal by exhaustion}: To prove that $\forall x \, P(x)$
is true when $P$ has a finite domain, evaluate the predicate at {\bf each} domain element to confirm that it is always T.
    \item  {\bf Proof by universal generalization}: To prove that $\forall x \, P(x)$
is true, we can take an arbitrary element $e$ from the domain and show that $P(e)$ is true, without making any assumptions 
about $e$ other than that it comes from the domain.
    \item To  prove  that $\exists x P(x)$ is {\bf false}, write the universal statement that is 
    logically equivalent to its negation and then prove it true using universal generalization.
    \item {\bf Strategies for conjunction}: To prove that $p \land q$ is true, have two subgoals: 
    subgoal (1) prove $p$ 
is  true; and, subgoal (2) prove $q$ is true. To prove that $p \land q$ is false, it's enough to prove that $p$ is false.
 To prove that $p \land q$ is false, it's enough to prove that $q$ is false.
    \item {\bf Proof of Conditional by Direct Proof}: To prove that the implication $p \to q$ is true, 
    we can assume $p$ is true and use that assumption to show $q$ is true.
    \item {\bf Proof of Conditional by Contrapositive Proof}: To prove that the implication $p \to q$ is true, 
    we can assume $\neg q$ is true and use that assumption to show $\neg p$ is true.
    \item {\bf Proof of disjuction using equivalent conditional}: To prove that the 
    disjunction $p \lor q$ is true, we can rewrite it equivalently as $\lnot p \to q$ and
    then use direct proof or contrapositive proof.
    \item {\bf Proof by Cases}: To prove $q$ when we know $p_1 \lor p_2$, show that $p_1 \to q$ and $p_2 \to q$.
    \item
    {\bf Proof by Structural Induction}: To prove that $\forall x \in X \, P(x)$ where $X$ is a recursively defined set, prove two cases:
        
        \begin{tabularx}{\textwidth}{l X}
        Basis Step: & Show the statement holds for elements specified in the basis step of the definition. \\
        Recursive Step: & Show that if the statement is true for each of the elements used to construct
    new elements in the recursive step of the definition, the result holds for these new elements.
    \end{tabularx}
    
    \item {\bf Proof by Mathematical Induction}: To prove a universal quantification over the set of  all integers greater than  or  equal to some base integer $b$:
    
    \begin{tabularx}{\textwidth}{l X}
        Basis Step: & Show the statement holds for $b$. \\
        Recursive Step: & Consider an arbitrary integer $n$ greater than or  equal to  $b$, assume
        (as the {\bf induction hypothesis})  that the property holds  for $n$, and use  this and
        other facts to  prove that  the property holds for $n+1$.
    \end{tabularx}
    
    \item {\bf Proof by Strong Induction} To prove that a universal quantification over the set of all integers greater than or equal to some  base integer $b$ holds,  pick a  fixed nonnegative integer  $j$ and then: \hfill 
    
    \begin{tabularx}{\textwidth}{l X}
        Basis Step: & Show the statement holds for $b$, $b+1$, \ldots, $b+j$. \\
        Recursive Step: & Consider an arbitrary integer $n$ greater than or  equal to  $b+j$, assume
        (as the {\bf strong  induction hypothesis})  that the property holds  for {\bf each of} $b$, $b+1$, \ldots, $n$, 	
        and use  this and
        other facts to  prove that  the property holds for $n+1$.
    \end{tabularx}

    \item {\bf Proof by Contradiction} 

    To prove that a statement $p$ is true, pick another statement $r$ and once we show
    that $\neg p  \to (r \wedge  \neg r)$ then  we can conclude that  $p$ is  true.
    
    {\it Informally} The statement we care about can't possibly be false, so it must be true.
\end{itemize}

\newpage
{\bf Assigned questions}

\begin{enumerate}
   \item Recall the definitions from class about factoring and divisibility:
   when $a$ and $b$ are integers and $a$ is nonzero, 
   {\bf $a$ divides $b$} means there is an integer $c$ such that $b = ac$ . 
   In this case, we say $a$ is a {\bf factor} of $b$, $a$ is $a$ {\bf divisor} of $b$, 
   $b$ is a {\bf multiple} of $a$, 
   $a | b$.  We define the function 
   $PosFactors: \mathbb{Z}^+ \to \mathcal{P}(\mathbb{Z}^+)$ by 
   \[
        PosFactors (n) = \{ x \in \mathbb{Z}^+ \mid x\text{ is a factor of } n\}
   \]

   \rule{0.5\textwidth}{.4pt}

   {\it Sample calculation that can be used as reference for the detail expected 
   in your answer when working with this function:} 
   
   The function application $PosFactors(4)$ evaluates to 
   \[
       PosFactors (4) = \{ 1,2,4\}
   \]
   because the only possible positive factors of $4$ are $1,2,3,4$ (the positive integers less than 
   or equal to $4$) and when we divide we get:
   \begin{align*}
        4 &= 4 \cdot 1 + 0 \qquad \text{so $4$ is a factor of $4$}\\
        4 &= 3 \cdot 1 + 1 \qquad \text{so $3$ is not a factor of $4$}\\
        4 &= 2 \cdot 2 + 0 \qquad \text{so $2$ is a factor of $4$}\\
        4 &= 1 \cdot 4 + 0 \qquad \text{so $1$ is a factor of $4$}
   \end{align*}

   \rule{0.5\textwidth}{.4pt}

   \begin{enumerate}
     \item ({\it Graded for correctness}\footnote{Graded for correctness means your solution will be
     evaluated not only on the correctness of your answers, but on your ability to 
     present your ideas clearly and logically. You should explain how you arrived at your conclusions, using 
     mathematically sound reasoning. Whether you use formal proof techniques or write a more informal argument for why 
     something is true, your answers should always be well-supported. Your goal should be to convince the reader that 
     your results and methods are sound.}) Give a witness that proves the statement 
     \[
         \exists x \in \mathbb{Z}^+ ~\forall y \in \mathbb{Z}^+ ~\left(~x \in PosFactors(y)~\right)
     \]
     Justify your choice of witness by explanations that include references to the relevant definitions.
     \item ({\it Graded for correctness}) Give a counterexample that disproves the statement 
     \[
         \forall n \in \mathbb{Z}^+ ~\left(~ PosFactors(n) \subseteq PosFactors(n+1)~\right)
     \]
     Justify your choice of counterexample by explanations that include references to the relevant definitions.
    
    \item ({\it Graded for fair effort completeness}) Consider the following attempted proof.
    \begin{quote}
    {\bf Attempted proof}: For arbitrary integers $a, b, c$, assume towards a direct proof that 
    $(a+b) | c$.  We need to show that $a|c$
    and $b | c$. Let $n$ be the integer $c \text{\bf ~div~} (a+b)$. 
    Since $(a+b) | c$, by definition of divides, $n | c$
    and $n$ is an integer.  Since $c = 1 \cdot n \cdot (a+b)$, $(n \cdot (a+b) ) | c$. 
    Rewriting by distributing 
    multiplication over addition, 
    we have $na | c$ and $nb | c$. Since $a | na$ and $na | c$, we have $a | c$.  
    Similarly, since $b | nb$ and 
    $nb | c$, we have $b | c$.  Thus, we have proved both conjuncts and the proof is complete.
    \end{quote}
    Select the statement below that the attempted proof is trying to prove.  
    \begin{enumerate}[label=(\roman*)]
    \item $\forall a \in \mathbb{Z} ~\forall b \in \mathbb{Z} ~\forall c \in \mathbb{Z} ~( ~( ~a|c ~\lor~b |c~) ~\to~ (a+b) |c~)$
    \item $\forall a \in \mathbb{Z} ~\forall b \in \mathbb{Z} ~\forall c \in \mathbb{Z} ~( ~( ~a|b ~\land~a |c~) ~\to~ a |(b+c)~)$
    \item $\forall a \in \mathbb{Z} ~\forall b \in \mathbb{Z} ~\forall c \in \mathbb{Z} ~(~ (a+b) | c ~\to ~( ~a|c ~\land~b |c~)~)$
    \end{enumerate}
    
    Identify the first major error in the attempted proof and explain why it is incorrect.
    
    Next, disprove the statement the attempted proof was attempting to prove.
    
    {\it Extra practice; not for credit}: prove or disprove the other two statements.
    \end{enumerate}

    \item In this question, we'll consider the function which calculates the sum of the first $n$ positive integers.
   \begin{enumerate}
        \item ({\it Graded for fair effort completeness}\footnote{Graded for fair effort completeness means 
        you will get full credit so long as your submission demonstrates honest 
        effort to answer the question. You will not be penalized for incorrect answers.}) 

        Give a recursive definition of this function, including domain, codomain and both the basis step
        and recursive step of the rule. That is, fill in the blanks 
        \[
            sumOfFirst: \underline{~~domain~~} \to \underline{~~codomain}
        \]
        given by 
        \begin{align*}
            &\textbf{Basis step}: \underline{\text{fill in basis step}} \\
            &\textbf{Recursive step}: \underline{\text{fill in recursive step}}
        \end{align*}

        {\it Notation}: Using summation, this function can be written $sumOfFirst(n) = \sum_{i=1}^n i$.\\


        \item ({\it Graded for fair effort completeness}) It turns out that the value of this function
        can also be calculated explicitly (without recursion)\footnote{When the value of a function 
        that is recursively defined can also be calculated without recursion, we call the formula 
        that we can use to calculate the value without recursion the ``closed form formula'' for the 
        function.}. You will 
        prove this by completing the proof of the identity
        \[
            \forall n \in \mathbb{Z}^+ ~\left(~sumOfFirst(n) = \frac{n(n+1)}{2} \right)
        \]
        
        Fill in the missing parts of the proof of this statement:\\

        {\bf Proof}: We proceed by mathematical induction on the set of positive integers.

        {\bf Basis Step}: Choose $n = 1$ as the basis step. 
        Using the Basis Step in the recursive definiton of $sumOfFirst$, 
        $sumOfFirst(1) = 1$. Plugging $n$ into the RHS of the desired formula,
        $\frac{1 (1+1)}{2} = \frac{2}{2} = 1$. Since LHS=RHS, the Basis step is complete.
   
        {\bf Recursive Step}: Consider an arbitrary $k \geq 1$.  
        We assume (as the induction hypothesis) that \underline{fill in the blank here}. 
   
        We want to show that $sumOfFirst(k+1) = \frac{(k + 1) \cdot ((k + 1) + 1)}{2}$.
   
        \underline{Fill in the rest of the proof here.} \\

        \item ({\it Graded for fair effort completeness})
        When calculating the runtime of an algorithm, nested for loops sometimes lead to program 
        runtimes that involve the sum of the first $n$ positive integers. To estimate
        the rate of growth of this runtime, it is useful to find an upper bound for this function 
        in terms of a simpler function.  Use the explicit formula from the earlier parts of this question
        and mathematical induction to prove
        \[
            \forall n \in \mathbb{Z}^+~ \left(~sumOfFirst(n) \leq n^2 \right)
        \]

   \end{enumerate}

   \item Recall the definition of linked lists that we discussed in class.


   Define the function $count$ which returns the number of occurrences of a datum
   in the list. Formally, $count: L \times \mathbb{N} \to \mathbb{N}$, where
   \begin{align*}
    \textbf{Basis Step:} \qquad &\textrm{If } m \in \mathbb{N},~~ count(~( [], m)~ ) = 0 \\ 
    \textbf{Recursive Step:} \qquad &\textrm{If } l \in L\textrm{ and }n \in \mathbb{N}
    \textrm{ and }m \in \mathbb{N} \textrm{, then  } \\
    &count(~(~(n, l),m~)~ ) = 
        \begin{cases}
            1 + count(~(l,m)~) &\text{if $n=m$} \\
            count(~(l,m)~) &\text{otherwise} \\
        \end{cases}
    \end{align*}

    A mystery function is defined by 
    $mystery : L \times \mathbb{N} \to L$, where

    \begin{align*}
    \textbf{Basis Step:} \qquad &\textrm{If } m \in \mathbb{N},~~ mystery(~( [], m)~ ) = [] \\ 
    \textbf{Recursive Step:} \qquad &\textrm{If } l \in L\textrm{ and }n \in \mathbb{N}
    \textrm{ and }m \in \mathbb{N} \textrm{, then  } \\
    &mystery(~(~(n, l),m~)~ ) = 
        \begin{cases}
            l &\text{if $n=m$} \\
            mystery(~(l,m)~) &\text{otherwise} \\
        \end{cases}
    \end{align*}

   \begin{enumerate}
        \item ({\it Graded for correctness}) Prove that
            \[
                \forall m \in \mathbb{N}~ \exists l \in L ~\left( count(~(l,20)~) = m~ \right)
            \]

        \item ({\it Graded for correctness}) Give an example input $x$ to the function such that 
            \[
               mystery( ~(~ x ~,~ 2~) ~) = []
            \]
           For full credit, include all intermediate steps of the function application
           that justifies your choice of $x$, with brief justifications for each.

        \item ({\it Graded for correctness}) Evaluate the function application
            \[
                mystery( ~(~ (2, (0, (2, []) ) ) ~,~ 2~) ~)
            \]
         For full credit, include all intermediate steps of the function application,
         with brief justifications for each.
   
        \item ({\it Graded for fair effort completeness for English statements and correctness in use 
        of syntax for symbolic statements}) Describe the rule of the function 
        $mystery$ in English. Then, write a true statement that describes an invariant using
        both the functions $mystery$ and $count$. Express this invariant both symbolically 
        and in English.
    \end{enumerate}

\end{enumerate}

\end{document}    
