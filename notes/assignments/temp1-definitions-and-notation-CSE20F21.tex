\documentclass[12pt, oneside]{article}

\usepackage[letterpaper, scale=0.89, centering]{geometry}
\usepackage{fancyhdr}
\setlength{\parindent}{0em}
\setlength{\parskip}{1em}

\pagestyle{fancy}
\fancyhf{}
\renewcommand{\headrulewidth}{0pt}
\rfoot{\href{https://creativecommons.org/licenses/by-nc-sa/2.0/}{CC BY-NC-SA 2.0} Version \today~(\thepage)}

\author{CSE20F21}

\newcommand{\instructions}{{\bf For all HW assignments:}

Weekly homework may be done individually or in groups of up to 3 students. 
You may switch HW partners for different HW assignments. 
The lowest HW score will not be included in your overall HW average. 
Please ensure your name(s) and PID(s) are clearly visible on the first page of your homework submission.

All submitted homework for this class must be typed. 
Diagrams may be hand-drawn and scanned and included in the typed document. 
You can use a word processing editor if you like (Microsoft Word, Open Office, Notepad, Vim, Google Docs, etc.) 
but you might find it useful to take this opportunity to learn LaTeX. 
LaTeX is a markup language used widely in computer science and mathematics. 
The homework assignments are typed using LaTeX and you can use the source files 
as templates for typesetting your solutions\footnote{To use this template, copy the source file (extension \texttt{.tex}) 
to your working directory or upload to Overleaf.}.


{\bf Integrity reminders}
\begin{itemize}
\item Problems should be solved together, not divided up between the partners. The homework is
designed to give you practice with the main concepts and techniques of the course, 
while getting to know and learn from your classmates.
\item You may not collaborate on homework with anyone other than your group members.
You may ask questions about the homework in office hours (of the instructor, TAs, and/or tutors) and 
on Piazza (as private notes viewable only to the Instructors).  
You \emph{cannot} use any online resources about the course content other than the class material 
from this quarter -- this is primarily to ensure that we all use consistent notation and
definitions we will use this quarter.
\item Do not share written solutions or partial solutions for homework with 
other students in the class who are not in your group. Doing so would dilute their learning 
experience and detract from their success in the class.
\end{itemize}

}
\usepackage{amssymb,amsmath,pifont,amsfonts,comment,enumerate,enumitem}
\usepackage{currfile,xstring,hyperref,tabularx,graphicx,wasysym}
\usepackage[labelformat=empty]{caption}
\usepackage[dvipsnames,table]{xcolor}
\usepackage{multicol,multirow,array,listings,tabularx,lastpage,textcomp,booktabs}

% NOTE(joe): This environment is credit @pnpo (https://tex.stackexchange.com/a/218450)
\lstnewenvironment{algorithm}[1][] %defines the algorithm listing environment
{   
    \lstset{ %this is the stype
        mathescape=true,
        frame=tB,
        numbers=left, 
        numberstyle=\tiny,
        basicstyle=\rmfamily\scriptsize, 
        keywordstyle=\color{black}\bfseries,
        keywords={,procedure, div, for, to, input, output, return, datatype, function, in, if, else, foreach, while, begin, end, }
        numbers=left,
        xleftmargin=.04\textwidth,
        #1
    }
}
{}
\lstnewenvironment{java}[1][]
{   
    \lstset{
        language=java,
        mathescape=true,
        frame=tB,
        numbers=left, 
        numberstyle=\tiny,
        basicstyle=\ttfamily\scriptsize, 
        keywordstyle=\color{black}\bfseries,
        keywords={, int, double, for, return, if, else, while, }
        numbers=left,
        xleftmargin=.04\textwidth,
        #1
    }
}
{}

\newcommand\abs[1]{\lvert~#1~\rvert}
\newcommand{\st}{\mid}

\newcommand{\A}[0]{\texttt{A}}
\newcommand{\C}[0]{\texttt{C}}
\newcommand{\G}[0]{\texttt{G}}
\newcommand{\U}[0]{\texttt{U}}

\newcommand{\cmark}{\ding{51}}
\newcommand{\xmark}{\ding{55}}





\title{HW1 Definitions and Notation}
\date{Due: Tuesday, October 4, 2021 at 11:00PM on Gradescope}

\begin{document}
\maketitle
\thispagestyle{fancy}


{\bf In this assignment,}

You will practice reading and
applying definitions to get comfortable working with mathematical language. As
a result, you can expect to spend more time reading the questions and looking
up notation than doing calculations.

\instructions

You will submit this assignment via Gradescope
(\href{https://www.gradescope.com}{https://www.gradescope.com}) 
in the assignment called ``HW1-definitions-and-notation''.


{\bf Resources}: To review the topics you are working with 
for this assignment, see the class material from  Week 0 and 1.
We will post frequently asked questions and our answers to them in a 
pinned Piazza post.

{\bf Assigned questions}

\begin{enumerate}

\item ({\it Graded for correctness}\footnote{This means your solution will be
evaluated not only on the correctness of your answers, but on your ability to 
present your ideas clearly and logically. You should explain how you arrived at 
your conclusions, using 
mathematically sound reasoning. Whether you use formal proof techniques or 
write a more informal argument for why 
something is true, your answers should always be well-supported. Your goal 
should be to convince the reader that 
your results and methods are sound.}) Each of the sets below is described 
using set builder notation or as a result of set operations
applied to other known sets.  Rewrite them using the roster method.

Remember our discussions of data-types: use clear notation that 
is consistent with our class notes and definitions 
to communicate the data-types of the elements in each set.


\rule{0.5\textwidth}{.4pt}

{\it Sample response that can be used as reference for the detail expected 
in your answer:} 

The set $\{ \A \} \circ \{ \A\U, \A\C, \A\G\}$ can be written using
the roster method as 
\[
\{ \A\A\U, \A\A\C, \A\A\G \}
\]
because set-wise concatenation gives a set whose elements are 
all possible results of concatenating an element of the 
left set with an 
element of the right set. Since the left set in this example only
has one element ($\A$), each of the elements of the set we 
described starts with $\A$. There are three elements of this set, 
one for each of the distinct elements of the right set.

\rule{0.5\textwidth}{.4pt}

\begin{enumerate}
\item $$\{ x \in S \mid rnalen(x) = 1 \} \circ \{x \in S \mid rnalen(x) = 1 \}$$
where $S$ is the set of RNA strands and $rnalen$ is the recursively defined
function that we discussed in class.
\item $$\{ (r,g,b) \in C \mid r+g+b = 1\}$$ where 
$C = \{ (r,g,b) \mid 0 \leq r \leq 255, 0 \leq g \leq 255, 0 \leq b \leq 255, r \in \mathbb{N}, g \in \mathbb{N}, b \in \mathbb{N} \}$
is the set that you worked with in Monday's review quiz.
\item $$\{ a \in \mathbb{Z} \mid  a \textbf{ div } 2 = a \textbf{ mod } 2\}$$
\end{enumerate}

\item ({\it Graded for fair effort completeness}\footnote{This means 
you will get full credit so long as your submission demonstrates honest 
effort to answer the question. You will not be penalized for incorrect answers.}) 

\begin{enumerate}
    \item In Wednesday's review quiz, you considered some attempted 
    recursive definitions for the function
    with domain $\mathbb{N}$ and with codomain $\mathbb{Z}$
    which gives $2^n$ for each $n$. 
    Write out a correct recursive definition of this function.
    \item How would your answer to part (a) change if we consider
    a new function with the same domain and rule but whose codomain 
    is $\mathbb{R}$?
    \item How would your answer to part (a) change if we consider
    a new function with the same codomain and rule but whose domain 
    is $\mathbb{R}$?
    \item Write a recursive definition of the function with domain $\mathbb{Z}^+$,
    codomain $\mathbb{Z}^+$ and which gives $n!$ for each $n$. The $!$ symbol
    is the ``factorial'' symbol and means that we need to multiple $n$ by each of the integers
    between it and $1$ inclusive. For example, $5! = 5 \cdot 4 \cdot 3 \cdot 2 \cdot 1 = 120$.
\end{enumerate}

\item ({\it Graded for correctness}) Recall the function
$d_0$ which takes an ordered pair of ratings $3$-tuples and returns a measure
of the distance between them 
given by
\[
d_0 (~(~ (x_1, x_2, x_3), (y_1, y_2, y_3) ~) ~) = \sqrt{ (x_1 - y_1)^2 + (x_2 - y_2)^2 + (x_3 -y_3)^2}
\]

\rule{0.5\textwidth}{.4pt}

{\it Sample response that can be used as reference for the detail expected 
in your answers for this question: } 

To give an example of two $3$-tuples that are $d_0$ distance $1$ from each other, 
consider the $3$-tuples $(1,0,0)$ and $(0,0,0)$. We calculate the function application:
\begin{align*}
    d_0 (~(~ (1, 0,0), (0,0,0) ~) ~) &= \sqrt{ (1 - 0)^2 + (0 - 0)^2 + (0 -0)^2} = \sqrt{1^2 + 0^2 + 0^2} = \sqrt{1} = 1,
\end{align*}
which is the result required for this example.

\rule{0.5\textwidth}{.4pt}

\begin{enumerate}
    \item Give an example of three $3$-tuples 
    \begin{align*}
    &(x_{1,1}, x_{1,2}, x_{1,3}) \\
    &(x_{2,1}, x_{2,2}, x_{2,3}) \\
    &(x_{3,1}, x_{3,2}, x_{3,3}) \\
    \end{align*}
    that are all $d_0$ distance {\bf greater than} $1$ from each other.  
    In other words, for each $i$ and $j$ between $1$ and 
    $3$ (with $i \neq j$), 
    \[
    d_{0}(~(~(x_{i,1}, x_{i,2}, x_{i,3})  , (x_{j,1}, x_{j,2}, x_{j,3})~)~) > 1
    \]
    
    Your answer should include  {\bf both} specific values for each example $3$-tuple {\bf and} a justification 
    of your examples with (clear, correct, complete) calculations and/or references to definitions and connecting them with
    the desired conclusion.\\
    
    {\it To think about}: 
    how will you justify that the square root of a number is greater than $1$?
    Are calculations from a calculator accurate enough to help us?

    \item What is the range of values that results from applying the function $d_0$
    to ordered pairs of $3$-tuple ratings? That is, what are the smallest and largest
    possible results?

    Your answer should include  {\bf both} specific values for the smallest and largest 
    possible results {\bf and} a justification 
    of your answers with (clear, correct, complete) calculations and/or references to definitions and connecting them with
    the desired conclusion.
\end{enumerate}
\end{enumerate}
\end{document}