\documentclass[12pt, oneside]{article}

\usepackage[letterpaper, scale=0.89, centering]{geometry}
\usepackage{fancyhdr}
\setlength{\parindent}{0em}
\setlength{\parskip}{1em}

\pagestyle{fancy}
\fancyhf{}
\renewcommand{\headrulewidth}{0pt}
\rfoot{\href{https://creativecommons.org/licenses/by-nc-sa/2.0/}{CC BY-NC-SA 2.0} Version \today~(\thepage.)}

\usepackage{amssymb,amsmath,pifont,amsfonts,comment,enumerate}
\usepackage{currfile,xstring,hyperref,tabularx,graphicx,wasysym}
\usepackage[labelformat=empty]{caption}
\usepackage[dvipsnames,table]{xcolor}
\usepackage{multicol,multirow,array,listings,tabularx,lastpage,textcomp,booktabs}

% NOTE(joe): This environment is credit @pnpo (https://tex.stackexchange.com/a/218450)
\lstnewenvironment{algorithm}[1][] %defines the algorithm listing environment
{   
    \lstset{ %this is the stype
        mathescape=true,
        frame=tB,
        numbers=left, 
        numberstyle=\tiny,
        basicstyle=\rmfamily\scriptsize, 
        keywordstyle=\color{black}\bfseries,
        keywords={,procedure, div, for, to, input, output, return, datatype, function, in, if, else, foreach, while, begin, end, }
        numbers=left,
        xleftmargin=.04\textwidth,
        #1
    }
}
{}
\lstnewenvironment{java}[1][]
{   
    \lstset{
        language=java,
        mathescape=true,
        frame=tB,
        numbers=left, 
        numberstyle=\tiny,
        basicstyle=\ttfamily\scriptsize, 
        keywordstyle=\color{black}\bfseries,
        keywords={, int, double, for, return, if, else, while, }
        numbers=left,
        xleftmargin=.04\textwidth,
        #1
    }
}
{}

\newcommand\abs[1]{\lvert~#1~\rvert}
\newcommand{\st}{\mid}

\newcommand{\A}[0]{\texttt{A}}
\newcommand{\C}[0]{\texttt{C}}
\newcommand{\G}[0]{\texttt{G}}
\newcommand{\U}[0]{\texttt{U}}

\newcommand{\cmark}{\ding{51}}
\newcommand{\xmark}{\ding{55}}



\begin{document}
\begin{flushright}
    \StrBefore{\currfilename}{.}
\end{flushright}

Welcome to the CSE 20: Discrete Math for Computer Science

\section*{Themes for CSE 20}
\begin{itemize}
\item Technical skepticism
\item Multiple representations
\end{itemize}

Why are we here?
\begin{itemize}
\item ... for discrete math
\item ... in Galbraith Hall
\item ... together
\end{itemize}

\section*{Introductions}
Class website: \href{http://cseweb.ucsd.edu/classes/fa21/cse20-a}{http://cseweb.ucsd.edu/classes/fa21/cse20-a}

Notice: URL structure

Instructor: Prof. Mia Minnes \tiny{"Minnes" rhymes with Guinness}


\section*{Recurring applications in CSE 20}
\begin{itemize}
\item Clustering and recommendation systems (machine learning, Netflix)
\item Genomics and bioinformatics (DNA and RNA)
\item Codes and information (secret message sharing and error correction)
\item ``Under the hood" of computers (circuits, pixel color representation, data structures)
\end{itemize}

%%% This is where the build drops all the small outcomes associated with snippets
%%%\input{learningoutcomes.tex}
% \section*{This week's highlights}
% \begin{itemize}
% \item Use and apply definitions and notation
% \item Explore mathematical definitions related to a specific application (Netflix)
% \item Define data types: set, $n$-tuple, string (over specific alphabet)
% \item Define sets and functions in multiple ways
% \item Trace an algorithm specified in pseudocode
% \item Define the base expansion of a positive integer, specifically decimal, binary, hexadecimal, and octal.
% \item Convert between expansions in different bases of a positive integer.
% \item Define and use the div and mod operators.
% \end{itemize}

\newpage
\section*{Friday September 24}
%! app: 
%! outcome: data types, translating, important sets, write set definition, function and relation definitions

\begin{center}
\begin{tabular}{|llp{9.8cm}|}
\hline
{\bf Term} & {\bf Notation Example(s)} & {\bf We say in English \ldots } \\
\hline
%$n$-tuple & $(x_1, x_2, x_3)$ & The 3-tuple of $x_1$, $x_2$, and $x_3$ \\
%          & $(3, 4)$ & The 2-tuple or {\bf ordered pair} of $3$ and $4$ \\
sequence & $x_1, \ldots, x_n$ & A sequence $x_1$ to $x_n$ \\
%         & $x_1, \ldots, x_n$ where $n = 0$ & An empty sequence \\
%         & $x_1, \ldots, x_n$ where $n = 1$ & A sequence containing just $x_1$ \\
%         & $x_1, \ldots, x_n$ where $n = 2$ & A sequence containing just $x_1$ and $x_2$ in order \\
%         & $x_1, x_2$ & A sequence containing just $x_1$ and $x_2$ in order \\
summation & $\sum_{i=1}^n x_i$ or $\displaystyle{\sum_{i=1}^n x_i}$ & The sum of the terms of the sequence $x_1$ to $x_n$ \\
&&\\
%maximum & $\displaystyle \max(x, y)$ & The max of $x$ and $y$, when they are numbers \\ % Note that this is different than summation!
%        & $\displaystyle \max_{1 \leq i \leq n} x_i$ & The max of $x_1$ to $x_n$, when they are numbers \\ % Also different from display
%&&\\
%set & & Unordered collection of objects. The set of \ldots \\
all reals & $\mathbb{R}$ & The (set of all) real numbers (numbers on the number line)\\
all integers & $\mathbb{Z}$ & The (set of all) integers (whole numbers including negatives, zero, and positives) \\
all positive integers & $\mathbb{Z}^+$ & The (set of all) strictly positive integers \\
all natural numbers & $\mathbb{N}$ & The (set of all) natural numbers. {\bf Note}: we use the convention that $0$ is a natural number. \\
%roster method & $\{43, 7, 9\}$ & The set whose elements are $43$, $7$, and $9$\\
%              & $\{9, \mathbb{N}\}$ & The set whose elements are $9$ and $\mathbb{N}$\\
%&&\\
%set builder notation & $\{ x \in \mathbb{Z} \mid x > 0\}$ & The set of all $x$ from the integers such that $x$ is greater than $0$ \\
%                     & $\{ 3x  \mid x \in \mathbb{Z} \}$ & The set of all integer multiples of $3$. {\bf Note}: we use the convention that writing two numbers next to each other means multiplication. \\
&&\\
%function rule definition & $f(x) = x + 4$ & Define $f$ of $x$ to be $x + 4$ \\
piecewise rule definition & $f(x) = \begin{cases} x & \text{if~}x \geq 0 \\ -x & \text{if~}x<0\end{cases}$ &
Define $f$ of $x$ to be $x$ when $x$ is nonnegative and to be $-x$ when $x$ is negative\\
function application & $f(7)$ & $f$ of $7$ {\bf or} $f$ applied to $7$ {\bf or} the image of $7$ under $f$\\
                     & $f(z)$ & $f$ of $z$ {\bf or} $f$ applied to $z$ {\bf or} the image of $z$ under $f$\\
                     & $f(g(z))$ & $f$ of $g$ of $z$ {\bf or} $f$ applied to the result of $g$ applied to $z$ \\
&&\\
absolute value & $\lvert -3 \rvert$ & The absolute value of $-3$ \\
square root & $\sqrt{9}$ & The non-negative square root of $9$ \\
&&\\
%summation notation & $\displaystyle \sum_{i=1}^n i$ & The sum of the integers from $1$ to $n$, inclusive \\
%                    & $\displaystyle \sum_{i=1}^n i^2 - 1$ & The sum of $i^2 - 1$ ($i$ squared minus $1$) for each $i$ from $1$ to $n$, inclusive \\
%&&\\
%quotient, integer division & $n~\textbf{div}~m$ & The (integer) quotient upon dividing $n$ by $m$; informally: divide and then 
%drop the fractional part\\
%modulo, remainder & $n~\textbf{mod}~m$ & The remainder upon dividing $n$ by $m$ \\

\hline
\end{tabular}
\end{center}
%! app: netflix
%! outcome: modeling
%! small-outcomes: 

What data should we encode about each Netflix account holder to help us make effective recommendations?

\vfill

In machine learning, clustering can be used to group similar data for prediction and recommendation.  For example,
each Netflix user's viewing history can be represented as a $n$-tuple indicating their preferences about
movies in the database, where $n$ is the number of movies in the database.  People with similar tastes in movies can then be clustered to provide recommendations
of movies for one another.  Mathematically, clustering is based on a notion of distance between pairs of $n$-tuples.

\vfill

%! app: Recommendation Systems
%! outcome: function and relation definitions, data types

In the table  below,  each row represents a user's ratings of movies: 
\cmark~(check) indicates the person liked the movie, \xmark~(x)
that they didn't, and $\bullet$ (dot) that they didn't rate it one way or another (neutral rating or didn't watch).

\begin{center}
\begin{tabular}{c|ccc||c}
Person & Fyre & Frozen II & Picard & Ratings written as a  $3$-tuple\\
\hline
$P_1$     & \xmark & $\bullet$ & \cmark & $(-1, 0, 1)$ \\
$P_2$     & \cmark & \cmark & \xmark & $(1, 1, -1)$ \\
$P_3$     & \cmark & \cmark & \cmark & $(1, 1, 1)$ \\
$P_4$     & $\bullet$ & \xmark & \cmark &  \\
\end{tabular}
\end{center}

Which of $P_1$, $P_2$, $P_3$ has movie preferences most similar to $P_4$?

One approach to answer this question: use {\bf functions} to define distance between user preferences.

\begin{center}
\begin{tabular}{|c|c|}
\hline
\multicolumn{2}{|l|}{
Define the following functions whose inputs are ordered pairs of $3$-tuples each of whose components}\\
\multicolumn{2}{|l|}{
 comes from the set $\{-1,0,1\}$
}
\\
\hline
&\\
$\displaystyle d_{1}(~ (x_1, x_2, x_3) , (y_1, y_2, y_3) ~) =  \sum_{i=1}^3\left( (\abs{x_i-y_i} + 1) \textbf{ div } 2 \right)$
&
$\displaystyle d_{2}(~ (x_1, x_2, x_3) , (y_1, y_2, y_3) ~) =  \sqrt{ \sum_{i=1}^3 (x_i - y_i)^2}$ \\
&\\
\hline
\end{tabular}
\end{center}

\begin{tabularx}{\textwidth}{|X|X|X|}
\hline &&\\
$d_1(P_4, P_1)$ & $d_1(P_4, P_2)$ & $d_1(P_4, P_3)$ \\
&&\\
&&\\
\hline&&\\
$d_2(P_4, P_1)$ & $d_2(P_4, P_2)$ & $d_2(P_4, P_3)$ \\
&&\\
&&\\
\hline
\end{tabularx}

\vfill

{\it Extra example:} A new movie is released, and $P_1$ and $P_2$ watch it before $P_3$, and give it
ratings; $P_1$ gives \cmark~and $P_2$ gives \xmark.
Should this movie be recommended to $P_3$? Why or why not?

{\it Extra example:} Define the new functions that would be used to compare the $4$-tuples of ratings encoding
movie preferences now that there are four movies in the database.

\newpage
\section*{Monday September 27}
\documentclass[12pt, oneside]{article}

\usepackage[letterpaper, scale=0.89, centering]{geometry}
\usepackage{fancyhdr}
\setlength{\parindent}{0em}
\setlength{\parskip}{1em}

\pagestyle{fancy}
\fancyhf{}
\renewcommand{\headrulewidth}{0pt}
\rfoot{\href{https://creativecommons.org/licenses/by-nc-sa/2.0/}{CC BY-NC-SA 2.0} Version \today~(\thepage.)}

\usepackage{amssymb,amsmath,pifont,amsfonts,comment,enumerate,enumitem}
\usepackage{currfile,xstring,hyperref,tabularx,graphicx,wasysym}
\usepackage[labelformat=empty]{caption}
\usepackage[dvipsnames,table]{xcolor}
\usepackage{multicol,multirow,array,listings,tabularx,lastpage,textcomp,booktabs}

\lstnewenvironment{algorithm}[1][] {   
    \lstset{ mathescape=true,
        frame=tB,
        numbers=left, 
        numberstyle=\tiny,
        basicstyle=\rmfamily\scriptsize, 
        keywordstyle=\color{black}\bfseries,
        keywords={,procedure, div, for, to, input, output, return, datatype, function, in, if, else, foreach, while, begin, end, }
        numbers=left,
        xleftmargin=.04\textwidth,
        #1
    }
}
{}
\lstnewenvironment{java}[1][]
{   
    \lstset{
        language=java,
        mathescape=true,
        frame=tB,
        numbers=left, 
        numberstyle=\tiny,
        basicstyle=\ttfamily\scriptsize, 
        keywordstyle=\color{black}\bfseries,
        keywords={, int, double, for, return, if, else, while, }
        numbers=left,
        xleftmargin=.04\textwidth,
        #1
    }
}
{}

\newcommand\abs[1]{\lvert~#1~\rvert}
\newcommand{\st}{\mid}

\newcommand{\A}[0]{\texttt{A}}
\newcommand{\C}[0]{\texttt{C}}
\newcommand{\G}[0]{\texttt{G}}
\newcommand{\U}[0]{\texttt{U}}

\newcommand{\cmark}{\ding{51}}
\newcommand{\xmark}{\ding{55}}

 
\begin{document}
\begin{flushright}
    \StrBefore{\currfilename}{.}
\end{flushright} \section*{Ratings examples}


In the table  below,  each row represents a user's ratings of movies: 
\cmark~(check) indicates the person liked the movie, \xmark~(x)
that they didn't, and $\bullet$ (dot) that they didn't rate it one way or another (neutral rating or didn't watch).

\begin{center}
\begin{tabular}{c|ccc||c}
Person & Fyre & Frozen II & Picard & Ratings written as a  $3$-tuple\\
\hline
$P_1$     & \xmark & $\bullet$ & \cmark & $(-1, 0, 1)$ \\
$P_2$     & \cmark & \cmark & \xmark & $(1, 1, -1)$ \\
$P_3$     & \cmark & \cmark & \cmark & $(1, 1, 1)$ \\
$P_4$     & $\bullet$ & \xmark & \cmark &  \\
\end{tabular}
\end{center}

Which of $P_1$, $P_2$, $P_3$ has movie preferences most similar to $P_4$?

One approach to answer this question: use {\bf functions} to define distance between user preferences.

\begin{center}
\begin{tabular}{|c|c|}
\hline
\multicolumn{2}{|l|}{
Define the following functions whose inputs are ordered pairs of $3$-tuples each of whose components}\\
\multicolumn{2}{|l|}{
 comes from the set $\{-1,0,1\}$
}
\\
\hline
&\\
$\displaystyle d_{1}(~ (x_1, x_2, x_3) , (y_1, y_2, y_3) ~) =  \sum_{i=1}^3\left( (\abs{x_i-y_i} + 1) \textbf{ div } 2 \right)$
&
$\displaystyle d_{2}(~ (x_1, x_2, x_3) , (y_1, y_2, y_3) ~) =  \sqrt{ \sum_{i=1}^3 (x_i - y_i)^2}$ \\
&\\
\hline
\end{tabular}
\end{center}

\begin{tabularx}{\textwidth}{|X|X|X|}
\hline &&\\
$d_1(P_4, P_1)$ & $d_1(P_4, P_2)$ & $d_1(P_4, P_3)$ \\
&&\\
&&\\
\hline&&\\
$d_2(P_4, P_1)$ & $d_2(P_4, P_2)$ & $d_2(P_4, P_3)$ \\
&&\\
&&\\
\hline
\end{tabularx}

\vfill

{\it Extra example:} A new movie is released, and $P_1$ and $P_2$ watch it before $P_3$, and give it
ratings; $P_1$ gives \cmark~and $P_2$ gives \xmark.
Should this movie be recommended to $P_3$? Why or why not?

{\it Extra example:} Define the new functions that would be used to compare the $4$-tuples of ratings encoding
movie preferences now that there are four movies in the database.
 \vfill
\end{document}
%! app: Bioinformatics, Numbers
%! outcome: recursive definitions

RNA is made up of strands of four different bases that encode genomic information
in specific ways.\\
The bases are elements of the set 
$B  = \{\A, \C, \U, \G \}$.


Formally, to define the set of all RNA strands, we need more than roster
method or set builder descriptions. 

\fbox{\parbox{\textwidth}{%

{\bf New! Recursive Definitions of Sets}: The set $S$ (pick a name) is defined by:
\[
\begin{array}{ll}
\textrm{Basis Step: } & \textrm{Specify finitely many elements of } S\\
\textrm{Recursive Step: } & \textrm{Give rule(s) for creating a new element of } S \textrm{ from known values existing in } S, \\
& \textrm{and potentially other values}. \\
\end{array}
\]
The set $S$ then consists of all and only elements that are put in $S$ by finitely many (a nonnegative integer number) of
applications of the recursive step after the basis step.
}}

{\bf Definition} The set of nonnegative integers $\mathbb{N}$ is defined (recursively) by: 
\[
\begin{array}{ll}
\textrm{Basis Step: } & \phantom{0 \in \mathbb{N}} \\
\textrm{Recursive Step: } & \phantom{\textrm{If } n \in \mathbb{N} \textrm{, then } n+1 \in \mathbb{N}}
\end{array}
\]

Examples: 

{\bf Definition} The set of all integers $\mathbb{Z}$ is defined (recursively) by: 
\[
\begin{array}{ll}
\textrm{Basis Step: } & \phantom{0 \in \mathbb{Z}} \\
\textrm{Recursive Step: } & \phantom{\textrm{If } x \in \mathbb{Z} \textrm{, then } x+1 \in \mathbb{Z}
\textrm{ and } x-1 \in \mathbb{Z}}
\end{array}
\]

Examples: 

\vfill

{\bf Definition} The set of RNA strands $S$ is defined (recursively) by:
\[
\begin{array}{ll}
\textrm{Basis Step: } & \A \in S, \C \in S, \U \in S, \G \in S \\
\textrm{Recursive Step: } & \textrm{If } s \in S\textrm{ and }b \in B \textrm{, then }sb \in S
\end{array}
\]
where $sb$ is string concatenation.

Examples: 

\vfill

{\bf Definition} The set of bitstrings (strings of 0s and 1s) is defined (recursively) by:
\[
\begin{array}{ll}
\textrm{Basis Step: } & \phantom{\lambda \in X} \\
\textrm{Recursive Step: } & \phantom{\textrm{If } s \in X \textrm{, then } s0 \in X \text{ and } s1 \in X}
\end{array}
\]

{\it Notation:} We call the set of bitstrings $\{0,1\}^*$.

Examples: 

\vfill
\newpage
\section*{Wednesday September 29}

\end{document}
