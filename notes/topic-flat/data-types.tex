\documentclass[12pt, oneside]{article}

\usepackage[letterpaper, scale=0.89, centering]{geometry}
\usepackage{fancyhdr}
\setlength{\parindent}{0em}
\setlength{\parskip}{1em}

\pagestyle{fancy}
\fancyhf{}
\renewcommand{\headrulewidth}{0pt}
\rfoot{\href{https://creativecommons.org/licenses/by-nc-sa/2.0/}{CC BY-NC-SA 2.0} Version \today~(\thepage.)}

\usepackage{amssymb,amsmath,pifont,amsfonts,comment,enumerate,enumitem}
\usepackage{currfile,xstring,hyperref,tabularx,graphicx,wasysym}
\usepackage[labelformat=empty]{caption}
\usepackage[dvipsnames,table]{xcolor}
\usepackage{multicol,multirow,array,listings,tabularx,lastpage,textcomp,booktabs}

\lstnewenvironment{algorithm}[1][] {   
    \lstset{ mathescape=true,
        frame=tB,
        numbers=left, 
        numberstyle=\tiny,
        basicstyle=\rmfamily\scriptsize, 
        keywordstyle=\color{black}\bfseries,
        keywords={,procedure, div, for, to, input, output, return, datatype, function, in, if, else, foreach, while, begin, end, }
        numbers=left,
        xleftmargin=.04\textwidth,
        #1
    }
}
{}
\lstnewenvironment{java}[1][]
{   
    \lstset{
        language=java,
        mathescape=true,
        frame=tB,
        numbers=left, 
        numberstyle=\tiny,
        basicstyle=\ttfamily\scriptsize, 
        keywordstyle=\color{black}\bfseries,
        keywords={, int, double, for, return, if, else, while, }
        numbers=left,
        xleftmargin=.04\textwidth,
        #1
    }
}
{}

\newcommand\abs[1]{\lvert~#1~\rvert}
\newcommand{\st}{\mid}

\newcommand{\A}[0]{\texttt{A}}
\newcommand{\C}[0]{\texttt{C}}
\newcommand{\G}[0]{\texttt{G}}
\newcommand{\U}[0]{\texttt{U}}

\newcommand{\cmark}{\ding{51}}
\newcommand{\xmark}{\ding{55}}

 
\begin{document}
\begin{flushright}
    \StrBefore{\currfilename}{.}
\end{flushright} \section*{Ratings examples}


In the table  below,  each row represents a user's ratings of movies: 
\cmark~(check) indicates the person liked the movie, \xmark~(x)
that they didn't, and $\bullet$ (dot) that they didn't rate it one way or another (neutral rating or didn't watch).

\begin{center}
\begin{tabular}{c|ccc||c}
Person & Fyre & Frozen II & Picard & Ratings written as a  $3$-tuple\\
\hline
$P_1$     & \xmark & $\bullet$ & \cmark & $(-1, 0, 1)$ \\
$P_2$     & \cmark & \cmark & \xmark & $(1, 1, -1)$ \\
$P_3$     & \cmark & \cmark & \cmark & $(1, 1, 1)$ \\
$P_4$     & $\bullet$ & \xmark & \cmark &  \\
\end{tabular}
\end{center}

Which of $P_1$, $P_2$, $P_3$ has movie preferences most similar to $P_4$?

One approach to answer this question: use {\bf functions} to define distance between user preferences.

\begin{center}
\begin{tabular}{|c|c|}
\hline
\multicolumn{2}{|l|}{
Define the following functions whose inputs are ordered pairs of $3$-tuples each of whose components}\\
\multicolumn{2}{|l|}{
 comes from the set $\{-1,0,1\}$
}
\\
\hline
&\\
$\displaystyle d_{1}(~ (x_1, x_2, x_3) , (y_1, y_2, y_3) ~) =  \sum_{i=1}^3\left( (\abs{x_i-y_i} + 1) \textbf{ div } 2 \right)$
&
$\displaystyle d_{2}(~ (x_1, x_2, x_3) , (y_1, y_2, y_3) ~) =  \sqrt{ \sum_{i=1}^3 (x_i - y_i)^2}$ \\
&\\
\hline
\end{tabular}
\end{center}

\begin{tabularx}{\textwidth}{|X|X|X|}
\hline &&\\
$d_1(P_4, P_1)$ & $d_1(P_4, P_2)$ & $d_1(P_4, P_3)$ \\
&&\\
&&\\
\hline&&\\
$d_2(P_4, P_1)$ & $d_2(P_4, P_2)$ & $d_2(P_4, P_3)$ \\
&&\\
&&\\
\hline
\end{tabularx}

\vfill

{\it Extra example:} A new movie is released, and $P_1$ and $P_2$ watch it before $P_3$, and give it
ratings; $P_1$ gives \cmark~and $P_2$ gives \xmark.
Should this movie be recommended to $P_3$? Why or why not?

{\it Extra example:} Define the new functions that would be used to compare the $4$-tuples of ratings encoding
movie preferences now that there are four movies in the database.
 \vfill
\end{document}